\subsection{Preparation}
The organization distinguish between security breaches and incidents. Security breaches are events that violate the organization's security principles caused intentionally by employees. Examples include employees visiting illegal websites or sharing their password with others. Incidents on the other hand are events threatening information security, but not necessarily caused by disloyal employees. Security breaches and incidents are often seen as deviations to the normal situation and that require some sort of structural change. The organization has a deviation system that keep track of all kinds of deviations, whether it is physical shortcomings or things related to information security.

When asked about what they consider to be the worst security incident the organization could possibly experience they answer that their biggest fear is disloyal employees leaking information to outsiders without being discovered. Other causes for compromised or disclosed information would also be considered very serious, but most likely these incidents are easier to detect. Information is the most important asset to protect, most other things can be restored or repaired.

Another severe consequence of incidents are service interruption. Ensuring that services are available and that business continuity is maintained are considered high priority. If services are unavailable, employees are left idle which potentially implies a major financial loss. Availability and access to files are therefore extremely important. Redundancy of equipment and other mechanisms are implemented to ensure availability. Investments to ensure availability are justified by the severe financial costs of service unavailability.

In addition to financial losses the organization sees reputation damage as a serious consequence of incidents. 
\begin{quote}
\textit{``If sensitive information concerning customers are leaked or lost, it is both a contractual breach as well as a violation of trust that would imply risk of reputation damage."}
\end{quote}

The supply manager says a worst case scenario is proliferation of either a virus or an operational fault caused by an employee leading to loss or disclosure of information.

The security policy are the main governing documents for activities conducted within the organization. These documents are intended to say something about senior management's intentions with the IT-security work and does not include detailed routines and activities. Its intention is to define senior management's position concerning IT security, and give an overall picture of where the organization stand. They have no policy that specifically address incident management, only routines and tools to do it.

The IT-manager or the supply manager are always available. To ensure communication during a serious incident, they have instant messenger as well as SIM-cards with various network suppliers. 

\paragraph{\acl{IRT}}
The \ac{IRT} is dynamic and changes with the type of incident the organization is dealing with. The main supplier of IT-operations act as response team and handle many minor incidents as part of their normal activities. The central contingency plan states whether incidents are categorizes as crisis or catastrophes to ensure they are handled correctly. The responsible person and the response team depends on the characteristics of the incident.   

For minor incidents, the \ac{IRT} is fully outsourced to the IT-operations supplier, but is gradually insourced as the severity of the incident increases. Thus, members of the \ac{IRT} may vary. The IT-manager says:
\begin{quote}
\textit{``We try to scale the organization in response to the specific incident we are dealing with."}
\end{quote}

The sequence of the scaling is described in the IT-contingency plan, where particular roles and activities are described. The permanent members of the \ac{IRT} are the IT-manager and the supply manager as they handle most incidents. Team members have other tasks beside incident management, but if there is need for them to respond to an incident, all other tasks are put on hold.

For incidents categorized as a crisis, a team composed of the IT-manager, the supply manager and people from the supplier will be formed to handle the incident. For the most serious incidents, a central manager in the organization will be in charge of handling the situation.

\paragraph{Standards and Guidelines}
All IT-managers have had training in ISO/IEC 27001 and ISO/IEC 27002 as well as ISO 27035. Additionally, some employees are certified in ITIL security. 
The supply manager says that standards are used as a basis for the fundamental thinking of how they work with security.  

The IT-manager says that just as important as being familiar with standards are being familiar with the internal documents describing how the organization perform incident management. 

The IT-manager emphasize that:
\begin{quote}
\textit{``The most important thing when a crisis occur is knowing what to do, not knowing what the standard says."}
\end{quote}
It is important that employees are familiar with internal routines, hence rehearsals are conducted regularly.

\paragraph{Awareness and Training}

All employees participate in an introductory course where routines for reporting incidents are explained. 

Employees are informed through the intranet if they need to be aware of new trends or specific spam e-mails. To raise awareness of IT security, the organization used a ``questionnaire" for employees regularly, addressing various security related topics such as secure use of memory sticks, viruses, social engineering and spam. This lead to discussions and rose awareness among employees. 

It is difficult to measure the effect of such measures. The IT-manager emphasize that it is not possible to track all incidents that did not occur. He says that:
\begin{quote}
\textit{``the level of security incidents has decreased the last three years. One may wonder whether this is due to us detecting fewer incidents or employees getting better at security. I don't think we'll get the answer."}
\end{quote}

The supply manager say that there are underreporting of incidents, but does not see it as a problem for their incident management. The potential high degree of underreporting does not imply that they miss seeing what the main threats are. He says that even though there are some incidents not being reported, the overall trend is still very apparent in their statistics. 
 
The organization conduct rehearsals regularly, talk about internal routines with employees as well as involving external suppliers in their training to best prepare for incident management in practise. Previously, exercises have been set up to expand and change as employees discuss what to do in given scenarios.

One thing revealed by training is the organizations ability to handle a real crisis. 

\subsection{Detection and Analysis}

The network supplier or other partners can sometimes give notifications if there is abuse or someone (one of the IPs in the network) contributing in a botnet.


\paragraph{Initial Detection}

As the organization has outsourced most of their IT operations, an external organization is responsible for running their internal IDS solution as well as the URL filtering towards the internet. For instance, Trojans can be detected if they have abnormal behaviour. Then organization 1(one of the main contractors for IT operations) is notified. They contact users and are responsible for re-installation of infected computers. In addition, all computers are obliged to have anti-virus installed, so the anti-virus system can in some cases notify the central monitoring agent, which again contacts users with infected computers for re-installation. 

The organization receive monthly reports from both their main IT contractors, where all incidents are recorded. 


\paragraph{Categorization}

\subsection{Incident Response}
\paragraph{Workflow}
Infected computers happen quite often and are thus seen as part of normal operations. If there is risk of escalation, users' network port are shut down before they are contacted. Sometimes situations arise indicating a serious security breach and the IT-manager is contacted. In addition to information security, the employee involved has to be dealt with which might need involvement from the management. 



\paragraph{Escalation}
\paragraph{Electronic Evidence}

\subsection{Lessons Learned}
The organization has external revision every other year.
