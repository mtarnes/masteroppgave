\subsection{Preparation}
The organization sees security breaches as events that violate the organization's security principles caused intentionally by employees. Examples include employees visiting illegal websites or sharing their password with others. Security incidents on the other hand, are events threatening information security, but not necessarily caused by disloyal employees. Security breaches and incidents are often seen as deviations from normal activity and often require some sort of structural change. The organization has a deviation system that keeps track of all kinds of deviations, whether it involves physical shortcomings or things related to information security.

The security policy is the main governing document for activities conducted within the organization. This document is intended to say something about senior management's intentions with the IT-security work and does not include detailed routines or activities. Its intention is to define senior management's position concerning IT security, and give an overall picture of where the organization stands with regards to information security. They have no policy that specifically address incident management, only practical routines and supporting tools.

When the interviewees were asked to consider what the worst possible security incident the organization could experience, they said that their biggest fear is disloyal employees leaking information to outsiders without being discovered. Compromised or disclosed information are considered very serious, but in most cases where this is caused by an error it is easier to detect than disloyal employees. Information is the most important asset to protect for the organization as most other things can be restored or repaired.

In addition to information disclosure, they see service interruption as another severe consequence of incidents. Ensuring service availability and business continuity are thus high priorities. Unavailable services could lead to employees not getting any work done which may imply a major financial loss for the organization. Availability and access to files are therefore extremely important and thus investments to ensure availability are justified by the severe financial costs of unavailable services. Redundant equipment and other mechanisms are implemented to ensure the availability of services. 

In addition to financial losses the organization sees reputation damage as a serious consequence of incidents. 
\begin{quote}
\textit{``If sensitive information concerning customers are leaked or lost, it is both a contractual breach as well as a violation of trust that would imply risk of reputation damage."}
\end{quote}

To avoid vulnerabilities in legacy software being exploited, the organization's computers are scanned regularly and software that are potential security holes or not in use are removed. 

The IT-manager says that one of the most challenging parts of incident management is constructing a holistic plan that includes everything. More often than not, things no one had thought of happen that needs to be handled. Hence, ensuring a solid information gathering during incidents and making correct decisions at the right time are more important than having a detailed plan to follow. Their experience indicate that the most important thing is to scale correctly, understand the situation and put relevant measures into action. That is also why they believe it is important that not too many have the authority to make important decisions during incidents, which they tried to limit in their contingency plan.

The organization has developed plans for communication. In the contingency plan it is described who is responsible for communication, both internally and externally. The IT-manager or the supply manager are always available in case of emergency. To ensure communication during serious incidents, the organization has an instant messenger application as well as SIM-cards with several network suppliers. 

\paragraph{Standards and Guidelines}
The organization bases many of their processes on the ITIL framework. Standards are mainly used as a basis for the fundamental thinking around how the organization works with security. All IT-managers have had training in ISO/IEC 27001 and ISO/IEC 27002 as well as ISO 27035. Additionally, some employees are certified in ITIL security.   

The IT-manager says that just as important as being familiar with standards is being familiar with the internal documents describing how the organization performs incident management. 

He emphasizes that:
\begin{quote}
\textit{``The most important thing when a crisis occurs is knowing what to do, not knowing what the standard says."}
\end{quote}
It is important that employees are familiar with internal routines, hence rehearsals are conducted regularly.

\paragraph{Awareness and Training}
All employees participate in an introductory course where routines for reporting incidents are explained. Additionally, employees are informed through the intranet in cases where they need to be aware of new trends or specific spam e-mails. To raise awareness around IT security, the organization has previously conducted a ``questionnaire" for employees via the intranet, addressing various security-related topics such as secure use of memory sticks, viruses, social engineering and spam. 

It is difficult to measure the effect of awareness campaigns. The IT-manager emphasizes that it is impossible to track incidents that did not occur. Nevertheless, the level of security incidents has decreased the last three years. He says that:
\begin{quote}
\textit{``One may wonder whether this is due to employees getting better at security or us detecting fewer incidents. I don't think we'll get the answer."}
\end{quote}
 
To raise awareness and best prepare for incident response in practise the organization conducts rehearsals regularly, discusses internal routines with employees as well as includes external suppliers in their training. Previously, exercises have been set up such that incidents escalate and change as employees discuss what to do in given scenarios.%

Notification, information, communication and crisis communication were identified as areas of improvement after training. The organization looked at their ability to handle a real crisis and found they needed a way to communicate to employees during emergency situations that is not dependent on e-mail or the intranet, when this is not necessarily available during a major crisis.

The organization has outsourced services with several external suppliers and thus collaboration and coordination are extremely important for their incident management. Having effective and sufficient coordination as well as distinct and well-established roles during incident response are highlighted by the IT-manager as important things to train for.

\subsection{Detection and Analysis}
The organization's network supplier or other partners may detect potential incidents. The supply manager is convinced that underreporting of incidents exist, but does not see it as a problem. The potentially high degree of underreporting does not imply that they miss seeing what the main threats are. He says that even though there are some incidents not being reported, the overall trend is still very apparent in their statistics and thus the right decisions can be made to avoid serious incidents. 

\paragraph{Initial Detection}
There are several ways the organization detects incidents. They have an \ac{IDS} reporting individual users or computers that for instance are involved in file sharing. Anti-virus detects and reports viruses, and in some cases employees themselves report that their computer is not working as expected. Those monitoring the network may also detect changes is traffic indicating security incidents. The IT-manager says that most of these initial detections are handled automatically, and that very few incidents require manual responses. 

\paragraph{Categorization}
The organization has developed their own framework for categorizing incidents. Inputs used for categorization of incidents are how many persons and departments are affected and the severity level. The IT-manager emphasizes that this categorization is quite similar to standards and that it is often the availability aspect that is in focus for incident categorization. The central contingency plan states whether incidents are categorized as a crisis or a catastrophe to ensure they are handled correctly.

\subsection{Incident Response}

\paragraph{\acl{IRT}}
The \ac{IRT} is dynamic and changes with the type of incident they are dealing with. The main supplier of IT-operations acts as a response team and handles most minor incidents as part of their normal activities. The responsible person and the response team depend on the characteristics of the incident.   

For minor incidents, the \ac{IRT} is fully outsourced to supplier 1, but is gradually insourced as the severity of the incident increases. Thus, members of the \ac{IRT} may vary. The IT-manager says:
\begin{quote}
\textit{``We try to scale the organization in response to the specific incident we are dealing with."}
\end{quote}

The sequence of the scaling is found in the IT-contingency plan, where particular roles and activities are described. The permanent members of the \ac{IRT} are the IT-manager and the supply manager as they handle most incidents. Team members have other tasks beside incident management, but if there is need for them to respond to an incident, all other tasks are put on hold.

For incidents categorized as a crisis, a team composed of the IT-manager, the supply manager and people from supplier 1 will be formed to respond to the incident. For the most serious incidents, a central manager in the organization will be in charge of the situation.

\paragraph{Workflow}
Minor incidents such as infected computers happen quite often and are seen as part of normal operations. If there is risk of escalation, users' network port can be shut down before users are contacted. Sometimes situations arise indicating a serious security breach and the IT-manager is contacted. In addition to information security, the employee involved has to be dealt with which might need involvement from the management.



The organization uses a proactive method for collecting information during incidents. %beskrevet bakerst i den sentrale beredskapsplanen.

The organization does not have an established check-list to follow during incident response. There is a check-list in the central contingency plan, but not one that addresses IT specifically. This was one of the things that were discovered through a rehearsal and they are currently working on such a check-list. The IT-manager says this is an area they wish to further develop to improve their incident response capabilities. 

Lack of proper communication may lead to unnecessary hassle. The supply manager says that the implicit knowledge of responsibilities in minor cases is an example of routines that are difficult to document properly.
\begin{quote}
\textit{``Problems may arise even with minor incidents when everyone assumes that it is someone else's responsibility."}
\end{quote}

The supply manager highlights the challenge of deciding how and how much information should be given to employees during incident response. This is challenging since it is very individual how much information people both want to share and receive, he says. Further, the IT-manager means this might be one of the most challenging aspects of incident management; - correct communication. To know when to communicate, what to say and to whom. 

\paragraph{Escalation}
Whenever an incident occurs a team is put together. The IT-manager emphasizes that the team is scalable, but that it escalates whenever incidents escalate and reaches a high severity. Most team members are employees from the main supplier of IT-operations, supplier 1, and the organization itself. Whenever incidents are of such a severity that escalation is necessary, the IT-manager says that they primarily try to recruit extra internal employees to the team, but that in some cases external experts are needed to respond effectively to an incident.

\paragraph{Electronic Evidence}
In case of security breaches, i.e. employees violating policies deliberately, all logs are preserved in case of need for future investigations. In cases where it is suspected that an employee is acting disloyal, their user account is blocked such that potential evidence is not deleted. The supply manager says assessing what to do in each individual case is difficult since sometimes incidents may be more serious than predicted. 

\subsection{Lessons Learned}
The organization has external revision every other year.

Incident reports are most often constructed by the two main suppliers and include excerpts from logs, what happened, what was done and proposed measures to avoid similar incidents in the future.The organization receives monthly reports from both their main IT suppliers, where all incidents are recorded. All IT-managers participate in a monthly meeting in an IT manager's forum where the monthly incident reports from the organization's main suppliers are discussed. In these meetings incidents are discussed, whether they are part of the report or not and questions and answers are discussed.

The organization only conducts debriefing meetings after serious incidents, which might be a couple of times a year. Additionally, four times a year, incidents concerning intelligence, espionage, crime for profit and information security are discussed within the organizations security board and are reported to the management. 

The supply manager emphasizes that when they work on improving their IT-operations and infrastructure, security is always in the back of their heads. The IT-manager says that there is always room for improvements. 
\begin{quote}
\textit{``It is a moving target we are trying to protect".}
\end{quote}

The IT-manager says that they are interested in obtaining a high contingency and security in their operations and that learning from others is an important factor in achieving this. To benefit from others' experience they often think about how incidents in other organizations would have affected their own organization. In that matter, incidents that hit other organizations will contribute to development and improvement of contingency plans. The IT-manager says that learning from other's experience is important since the scenarios are real and could happen in their organization as well.