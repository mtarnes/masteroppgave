\section{Why We Need Incident Management}
\label{sec:threatLandscape}
Modern society shows an increasing use of digital solutions. Today, digital solutions are vital to most organizations' day-to-day operations and large amounts of sensitive data are stored digitally \cite{KriposTrender}. As the value and sensitivity of information increases, the number of potential threats increase accordingly. This suggests that organizations today are more exposed to attacks than before. This section discusses the current threat landscape and the need for plans in situations where systems have not been sufficiently secure. 

Organizations are increasingly using and depending on information technology in their operations. Attacks get more advanced and attackers choose their targets more strategically. A significant challenge arises when new and severe security threats evolve faster than corresponding measures. This leads to an increasing gap between threats and security measures in organizations. To avoid severe consequences such as disclosure of sensitive information, this gap must be closed.

Despite organizations' implementation of information security policies and controls, it is inevitable that new vulnerabilities and information security incidents occur occasionally. Thus, it is essential that organizations have a structured and planned approach to detect, report, assess, respond to and learn from information security incidents \cite{ISO/IEC27035}. Preventive actions are not sufficient and an incident management capability is therefore necessary.

\begin{newquote}{Roar Thon, Senior Adviser \acs{NSM}}
\textit{"Everybody should do what they can to protect themselves from being attacked, but the sad truth is that the most important thing you should plan and prepare for is how to behave when the attacker has succeeded"}
\end{newquote}

The information security threat landscape is continuously changing and new types of security-related incidents emerge frequently \cite{nist800-61}. %Preventive actions are not sufficient in order to be able to handle this and an incident response capability is therefore necessary. Even though incident prevention is not sufficient, it is an important complement to incident response. 

\acs{NSM} \acs{NorCERT}\footnote{\acs{NSM} is the Norwegian National Security Authority and is a cross-sectional professional and supervisory authority within the protective security services in Norway \cite{AboutNSM}. \acs{NorCERT} is the \acl{NorCERT} and is part of \acs{NSM}. \acs{NorCERT} coordinates preventive work and responses against IT security breaches aimed at vital infrastructure in Norway \cite{AboutNorCERT}} has registered a 30\% increase in cases each year for the past few years. They have seen an increase in cases of all impact levels. In addition they believe that there is a large number of incidents not reported or discovered \cite{NorCERT3Kvartal2012}. These findings are supported by Kripos\footnote{Kripos is the \ac{NCIS} in Norway and it is the unit for combating organized and other serious crime \cite{policeInNorway}.}, that reports ICT related crime to be expanding \cite{KriposTrender}. There is a large increase in targeted espionage operations directed towards Norwegian industry \cite{NSMRapport2012}. Attacks are mainly driven by \ac{ROI}, thus targets are chosen based on potential profit. Other incidents not necessarily motivated by money are strategic targets and domestic political monitoring as seen in China and Syria\cite{Morketall2012}.

\acs{NSM} states that the security condition in Norway for 2012 is not satisfactory \cite{samordnaVurdering}. This seems to be a continuing trend and the security condition for 2011 was summarized in the following way \cite{NSMRapport}: 

\begin{quote}
\textit{``The \textbf{values} we want to protect increase in amount, the \textbf{threats} are increasing, new \textbf{vulnerabilities} are constantly discovered, but \textbf{measures} to reduce these vulnerabilities are not developed at the same rate in addition to being inadequate."}
\end{quote}

Additionally, there is an increasing number of vulnerabilities discovered on smart phones and tablets, which represents a relatively new part of the threat landscape. There exist persistent vulnerabilities in organizations with classified information and these exist mainly due to lack of understanding of risks \cite{NSMRapport2012}.

In 2012 \acs{NorCERT} handled a large amount of serious cases related to espionage against Norwegian high-technology organizations \cite{NorCERT3Kvartal2012}. In a recent report PST\footnote{The Norwegian Police Security Service} expressed concerns related to Norwegian research and education environments being exploited to strengthen other nations' defences \cite{PSTvurdering2013}.   

Many of the current threats cannot be stopped by antivirus software. Attacks are increasingly becoming targeted to specific organizations in addition to becoming more advanced. Delay in updates and patches of computers is a big problem for many organizations \cite{NorCERT2Kvartal2012}. \acs{NSM} has observed a change in attacks from random and opportunistic attacks to advanced and focused attacks on specific targets of high economic or social value. In addition to technical means, attackers use social engineering\footnote{Social engineering involves manipulating people into performing certain actions such as disclosing sensitive information.} to obtain sensitive information or to obtain access to systems \cite{NSMRapport}. Another trend is attacks that compromise legitimate websites and infect all users that visit them \cite{NSMRapport2012}. Such attacks are called water-holing. They are particularly difficult to protect against as these exploit websites that users are normally allowed to visit.

There is great diversification in type of attackers. Attackers can belong to foreign intelligence, traditional military, global businesses, terrorist organizations, hacker groups or they can operate individually \cite{samordnaVurdering}. Criminals are organized in new ways, and various participants contribute with services, making attacks possible \cite{KriposTrender}. It is even possible to buy attacks like \ac{DDoS} attacks or spam distribution \cite{NorCERT2Kvartal2012}. Attacks can also come from the inside, either from an insider or by social engineering, and many organizations do not focus on this threat \cite{NSMRapport2012}. This expands the group of possible attackers, in practice it includes everyone. 

Several publications and recent reports highlight the need for incident management by pointing out deficiencies in organizations' information security. PST states that information security is given low priority in Norwegian government and private institutions \cite{PSTvurdering}. This is supported by \acs{NSM} that states that organizations seem to lack the ability and/or will to prioritize ICT-security \cite{NSMmelding}. Incident handling is often not prioritized and the severity of attacks are often not understood \cite{NorCERT2Kvartal2012}. Management's knowledge of information security is often insufficient, which is unfortunate as this is key to commitment of the rest of the organization \cite{NorCERT3Kvartal2012}. Systems for reporting security incidents to the management rarely exist and \acs{NSM} almost always discover that incidents have occurred or are occurring when they perform inspections\cite{NSMRapport}. Many organizations have inadequate contingency plans related to information security. Organizations also omit to conduct rehearsals related to preventive security and omit to rehearse their contingency plans \cite{NSMRapport2012}.

Several trends described here are not unique to Norway. Verizon Enterprise's RISK team published a report in cooperation with the \ac{USSS}, the Dutch \ac{NHTCU}, the \ac{AFP}, the \ac{IRISS} and the \ac{PCeU} of the London Metropolitan Police \cite{VerizonReport}. The report discusses data breaches in 2011 in 36 different countries. 855 incidents were analysed. It shows that 96\% of the (reported) attacks were not particularly advanced. It also shows that 85\% of the breaches took weeks or more to discover and that 92\% of incidents were discovered by a third party. 86\% of the breaches were caused organized crime. Based on the cases reported to the involved organizations, 2011 seems to be the year with the second highest number of data losses since 2004. The results of this report indicate that the overall international security condition is not satisfactory.

This shows a complex threat landscape with a large variety of attackers and with organizations that are not sufficiently prepared. It is not realistic to believe that all incidents can be prevented. In addition, it is not economically feasible. Hence, it is evident that organizations need plans and procedures to handle incidents \textit{when} they occur. The existence of an incident response capability in an organization can assist them in rapidly detecting incidents, minimizing loss and destruction, mitigating the weaknesses that were exploited and restoring computing services \cite{nist800-61}. 