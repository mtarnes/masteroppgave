\section{Why We Need Incident Management}
Modern society has an increasing use of digital solutions. This suggests that organizations today are more vulnerable to attacks. Today, digital solutions are vital to most organizations' day-to-day operations and large amounts of sensitive data are stored digitally \cite{KriposTrender}. It is obvious that secure systems are important, but it might not be a good idea to trust that systems are sufficiently secure.  This section discusses the current threat landscape and why we need plans for situations where our systems have not been secure enough.

\subsection{Threat Landscape}
The threat landscape is constantly in change and a large number of serious threats are emerging.

\acs{NSM} \acs{NorCERT}\footnote{\acs{NSM} is the Norwegian National Security Authority and is a cross-sectional professional and supervisory authority within the protective security services in Norway \cite{AboutNSM}. \acs{NorCERT} is the \acl{NorCERT} and is part of \acs{NSM}. \acs{NorCERT} coordinates preventive work and responses against IT security breaches aimed at vital infrastructure in Norway \cite{AboutNorCERT}} has had a 30\% increase in cases each year for the past few years. They have had an increase in both low, medium and high cases. Additionally they believe that there is a large number of incidents not reported or discovered \cite{NorCERT3Kvartal2012}. Kripos\footnote{Kripos is the \ac{NCIS} in Norway and it is the unit for combating organized and other serious crimes \cite{policeInNorway}.} reports that ICT-related crime is expanding \cite{KriposTrender}.

In 2012 \acs{NorCERT} handled a large amount of serious cases related to espionage against Norwegian high-technology organizations. PST\footnote{The Norwegian Police Security Service} states that information security is given low priority in Norwegian government and private institutions \cite{PSTvurdering}. In a recent report PST expresses concern related to Norwegian research- and education environments being exploited to strengthen other nations' defences \cite{PSTvurdering2013}.   

Many current threats cannot be stopped by antivirus. Attacks are increasingly becoming targeted to specific organizations in addition to becoming more advanced. Delay in updates on computers is a big problem for many organizations. Incident handling is often not prioritized and the severity of attacks are often not understood \cite{NorCERT2Kvartal2012}. Knowledge of information security within management in organization is often insufficient, which is unfortunate as this is key to commitment of the rest of the organization \cite{NorCERT3Kvartal2012}. 

There is great diversification in the type of attackers. Criminals are organized in new ways, making attacks possible \cite{KriposTrender}. It is even possible to buy attacks and vulnerabilities, like \ac{DoS} attacks or spam e-mail sending \cite{NorCERT2Kvartal2012}. This expands the group of possible attackers, in practice it includes everyone. 

The government in Norway seems to understand that something needs to be done and have granted \acs{NSM} \acs{NorCERT} more money than ever before \cite{NorCERT3Kvartal2012}. This shows a growing initiative related to preventive security work.

%datainnbrudd - computer intrusion?