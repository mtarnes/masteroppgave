\section{Why We Need Incident Management}
Modern society shows an increasing use of digital solutions. Today, digital solutions are vital to most organizations' day-to-day operations and large amounts of sensitive data are stored digitally \cite{KriposTrender}. This suggests that organizations today are more vulnerable to attacks than before. 

Despite organizations' implementation of information security policies and controls, it is inevitable that new vulnerabilities and thus information security incidents occur occasionally. Thus it is essential that organizations have a structured and planned approach to detect, report, assess, respond and learn from information security incidents. \cite{ISO/IEC27035}

This section discusses the current threat landscape and the need for plans in situations where systems have not been secure enough. 

\subsection{Threat Landscape}
The security-related threat level is continuously changing and new types of security-related incidents emerge frequently \cite{nist800-61}. %Preventive actions are not sufficient in order to be able to handle this and an incident response capability is therefore necessary. Even though incident prevention is not sufficient, it is an important complement to incident response. 

\acs{NSM} \acs{NorCERT}\footnote{\acs{NSM} is the Norwegian National Security Authority and is a cross-sectional professional and supervisory authority within the protective security services in Norway \cite{AboutNSM}. \acs{NorCERT} is the \acl{NorCERT} and is part of \acs{NSM}. \acs{NorCERT} coordinates preventive work and responses against IT security breaches aimed at vital infrastructure in Norway \cite{AboutNorCERT}} has had a 30\% increase in cases each year for the past few years. They have had an increase in both low-, medium- and high level cases. Additionally they believe that there is a large number of incidents not reported or discovered \cite{NorCERT3Kvartal2012}. These findings are supported by Kripos\footnote{Kripos is the \ac{NCIS} in Norway and it is the unit for combating organized and other serious crimes \cite{policeInNorway}.}, that reports ICT-related crime to be expanding \cite{KriposTrender}. Especially targeted attacks, known as advanced persistent threats, appears to be increasing\cite{Morketall2012}. Attacks are mainly driven by \ac{ROI}, thus targets change with where attacks are profitable. Other concerns not necessarily motivated by money are strategic targets and domestic political monitoring as seen in China and Syria\cite{Morketall2012}.

\acs{NSM} states that the security condition in Norway for 2012 is not satisfying \cite{samordnaVurdering}. This seems to be a continuing trend and the security condition for 2011 is summarized in the following way: The \textit{values} we want to protect increase in amount, the \textit{threats} are increasing, new \textit{vulnerabilities} are constantly discovered, but \textit{measures} to reduce these vulnerabilities are not developed at the same rate in addition to being inadequate \cite{NSMRapport}. Additionally there is an increasing number of vulnerabilities discovered on smart phones and tablets, which represents a relatively new part of the threat landscape.

In 2012 \acs{NorCERT} handled a large amount of serious cases related to espionage against Norwegian high-technology organizations \cite{NorCERT3Kvartal2012}. In a recent report PST expresses concern related to Norwegian research- and education environments being exploited to strengthen other nations' defences \cite{PSTvurdering2013}.   

Many current threats cannot be stopped by antivirus. Attacks are increasingly becoming targeted to specific organizations in addition to becoming more advanced. Delay in updates on computers is a big problem for many organizations \cite{NorCERT2Kvartal2012}. \acs{NSM} has observed a change in attacks from random and opportunistic attacks to advanced and focused attacks on specific targets of high economic or social value. In addition to technical means, attackers use social engineering to obtain sensitive information or to obtain access to systems \cite{NSMRapport}.  

There is great diversification in the type of attackers. Attackers can belong to foreign intelligence, traditional military, global businesses, terrorists, organized hacker groups or operate as individuals \cite{samordnaVurdering}. Criminals are organized in new ways, with various participants contributing with services, making attacks possible \cite{KriposTrender}. It is even possible to buy attacks and vulnerabilities, like \ac{DoS} attacks or spam sending \cite{NorCERT2Kvartal2012}. This expands the group of possible attackers, in practice it includes everyone. 

PST\footnote{The Norwegian Police Security Service} states that information security is given low priority in Norwegian government and private institutions \cite{PSTvurdering}. This is supported by \acs{NSM} that states that organizations seem to lack the ability and/or will to prioritize ICT-security \cite{NSMmelding}. Incident handling is often not prioritized and the severity of attacks are often not understood \cite{NorCERT2Kvartal2012}. Knowledge of information security within management in organizations is often insufficient, which is unfortunate as this is key to commitment of the rest of the organization \cite{NorCERT3Kvartal2012}. Systems for reporting security incidents to the management rarely exist and \acs{NSM} almost always discover that incidents have occurred or are occurring when they perform inspections. Many organizations have inadequate emergency plans related to information security \cite{NSMRapport}.

This shows a complex threat landscape with a large variety of attackers and with organizations that are not sufficiently prepared to handle it. It is not realistic to believe that all incidents can be prevented and it is evident that organizations need plans and procedures to handle incidents \textit{when} they occur. The existence of an incident response capability in an organization can assist them in rapidly detecting incidents, minimizing loss and destruction, mitigating the weaknesses that were exploited and restoring computing services \cite{nist800-61}. 

%datainnbrudd - computer intrusion?