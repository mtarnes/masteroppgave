\subsection{ITIL}
\ac{ITIL} is a framework and a source of good practice in service management. The \acs{ISO}/\acs{IEC} 27000 standard is aligned with \ac{ITIL}. This section gives a brief introduction to \ac{ITIL}, focusing on the parts related to incident management and the content is, unless specified otherwise, derived from \cite{itilbok}. The definitions presented in this section are taken directly from \cite{itilbok}.

To describe service management, the \ac{ITIL} framework uses the following definitions:

\textbf{Service:} A service is a means of delivering value to customers by facilitating outcomes that customers want to achieve without the ownership of specific costs and risks.

\textbf{Service Management:} Service management is a set of specialized organizational capabilities for providing value to customers in the form of services.

The specialized organizational capabilities include the processes, activities, functions and roles that a service provider uses in delivering services. The framework is generic and is meant to be useful for any type of organization. It describes a set of functions and processes that can be implemented in order to be able to perform service management. The terms function and process are defined in the following ways:

\textbf{Function:} A team or group of people and the tools they use to carry out one or more processes or activities.

\textbf{Process:} A process is a structured set of activities designed to accomplish a specific objective. A process takes one or more defined inputs and turns them into defined outputs. A process may include any of the roles, responsibilities, tools and management controls required to reliably deliver the outputs. A process may define policies, standards, guidelines, activities and work instructions if they are needed.

%Risk is defined as a possible event that could cause harm or loss, or affect the ability to achieve objectives. Risk can also be defined as the uncertainty of outcome.

This section describes processes and functions related to incident management.

\paragraph{Availability Management}
Availability management is essential for an organization and it is primarily a proactive process. In addition to activities such as preparing and maintaining an availability plan and monitoring availability levels this process includes assisting with investigation and resolution of availability-related incidents and problems. The latter is a reactive part of availability management. This process is related to other processes including IT service continuity, information security, event, incident and problem management. The availability manager is responsible for this process.

\paragraph{IT Service Continuity Management}
This process is concerned with key systems in the event of a failure. The purpose of the process is to ensure that IT resources, systems and services can be restored within agreed timescales in the event of a major incident. The process is related to availability and information security management. The IT service continuity manager is responsible for ensuring that the objectives of this process are met.

\paragraph{Information Security Management}

\paragraph{The Service Desk}
The service desk is a function.

\paragraph{Incident Management}

\paragraph{Problem Management}

\paragraph{Event Management}