\subsection{\acs{ISO}/\acs{IEC} 27001}
\label{sec:iso27001}
This standard provides a model for establishing, implementing, operating, reviewing, maintaining and improving an \ac{ISMS}. The standard adopts the ``Plan-Do-Check-Act" model to structure all \ac{ISMS} processes. The \ac{ISMS} takes information security requirements and expectations of interested parties as input and produces information security outcomes that meet those requirements. The standard encourages to emphasize the importance of:
\begin{itemize}\itemsep-0.2cm
\item understanding an organization's information security requirements and the need to establish policy and objectives for information security;
\item implementing and operating controls to manage an organization's information security risks in the context of the organization's overall business risks;
\item monitoring and reviewing the performing and effectiveness of the \ac{ISMS}; and
\item continual improvement based on objective measurement.
\end{itemize}

This section describes parts of the standard that are relevant to incident management and the content is, unless specified otherwise, derived from \cite{ISO/IEC27001}.

\textbf{4.2.1 Establish the \ac{ISMS} } \\
The organization shall do the following.
\begin{enumerate}[d)]
\item Identify the risks.
\begin{enumerate}[1)]\itemsep-0.2cm
\item Identify the assets within the scope of the \ac{ISMS}, and the owners of these assets.
\item Identify the threats to these assets.
\item Identify the vulnerabilities that might be exploited by the threats.
\item Identify the impacts that losses of confidentiality, integrity and availability may have on the assets.
\end{enumerate}
\end{enumerate}

This clause is not directly related to incident management, but it can be an important prerequisite. %NBNB finn kilde til dette utsagnet?

\textbf{4.2.2 Implement and operate the \ac{ISMS}} \\
The organization should do the following.
\begin{enumerate}[h)]
\item Implement procedures and other controls capable of enabling prompt detection of security events and response to security incidents.
\end{enumerate}

This clause specifies that organizations should be able to detect and handle security incidents.

\textbf{4.2.3 Monitor and review the ISMS}\\
The organization shall do the following.
\begin{enumerate}[a)]
\item Execute monitoring and reviewing procedures and other controls to:
\begin{enumerate}[2)]
\item promptly identify attempted and successful security breaches and incidents;
\end{enumerate}
\vspace{-0.2cm}
\begin{enumerate}[4)]
\item help detect security events and thereby prevent security incidents by the use of indicators;
\end{enumerate}
\vspace{-0.2cm}
\begin{enumerate}[5)]
\item determine whether the actions taken to resolve a breach of security were effective.
\end{enumerate}
\end{enumerate}

Common for all clauses in this standard is that they only specify that things should be done, and not how. \acs{ISO}/\acs{IEC} 27002 provides a code of practice for information security management and \acs{ISO}/\acs{IEC} 27035 provides guidelines for the establishment of information security management. These standards are further described in sections \ref{sec:iso27002} and \ref{sec:iso27035} and they can be used to aid the fulfilment of the clauses presented in \acs{ISO}/\acs{IEC} 27001.