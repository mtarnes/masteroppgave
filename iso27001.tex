\subsection{The \acs{ISO}/\acs{IEC} 27001 Standard}
\label{sec:iso27001}
This standard provides a model for establishing, implementing, operating, reviewing, maintaining and improving an \ac{ISMS}. It states that management shall provide evidence of its commitment to the ISMS. This section presents clauses relevant to incident management that are directly retrieved from the standard \cite{ISO/IEC27001}. 

\textbf{4.2.2 Implement and operate the \ac{ISMS}} \\
The organization should do the following.
\begin{enumerate}[h)]
\item Implement procedures and other controls capable of enabling prompt detection of security events and response to security incidents.
\end{enumerate}

This clause specifies that organizations should be able to detect and handle security incidents.

\textbf{4.2.3 Monitor and review the ISMS}\\
The organization shall do the following.
\begin{enumerate}[a)]
\item Execute monitoring and reviewing procedures and other controls to:
\begin{enumerate}[2)]
\item promptly identify attempted and successful security breaches and incidents;
\end{enumerate}
\vspace{-0.2cm}
\begin{enumerate}[4)]
\item help detect security events and thereby prevent security incidents by the use of indicators;
\end{enumerate}
\vspace{-0.2cm}
\begin{enumerate}[5)]
\item determine whether the actions taken to resolve a breach of security were effective.
\end{enumerate}
\item Undertake regular reviews of the effectiveness of the \ac{ISMS} (including meeting \ac{ISMS} policy and objectives, and review of security controls) taking into account results of security audits, incidents, results from effectiveness measurements, suggestions and feedback from all interested parties.
\end{enumerate}

\textbf{4.3.3 Control of records}\\
Records shall be kept of the performance of the process as outlined in 4.2 and of all occurrences of significant security incidents related to the \ac{ISMS}

Common for all clauses in this standard is that they only specify that things should be done, and not \textit{how} they should be done. \acs{ISO}/\acs{IEC} 27002\footnote{ \acs{ISO}/\acs{IEC} 27002 Information technology - Security techniques
- Code of practice for information security management} provides a code of practice for information security management and \acs{ISO}/\acs{IEC} 27035 provides guidelines for the establishment of information security incident management. These standards are further described in sections \ref{sec:iso27002} and \ref{sec:iso27035} and can be used as aids to fulfil the clauses presented in \acs{ISO}/\acs{IEC} 27001.