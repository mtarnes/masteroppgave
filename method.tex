\chapter{Method}
\label{chp:method}
This section describes the research method for this thesis as well as reasons for the choices made. Further, ethical considerations and challenges are discussed.
\section{Choice of Method}
\label{sec:choiceOfMethod}
Figure \ref{fig:methods} shows an overview of various research methods and three criteria that can be used to determine the appropriate research method. The criteria are: form of research question, whether the study requires control of behavioural events and if the study focuses on contemporary events. The defined research question for this study, as presented in section \ref{sec:objectives}, is a so-called ``how" question. As the goal of our study was to reveal current practices in organizations, we did not need control over behavioural events. This study's focus was mainly contemporary events. Some past events such as incidents that have occurred were relevant, but the main focus was on current practices. Based on this, case study emerged as the most suitable method for this study, as highlighted in the figure.

A case study is applicable to real-world organizations, which is what we wanted to study. An advantage is that it can deal with various kinds of evidence, such as documents, archival records, interviews and artefacts.

\begin{figure}[h]
\begin{center}
\includegraphics[scale=0.35]{methods.png}
\caption[Choice of Research Method]{Choice of Research Method, modified from \cite{CaseStudyResearch}}
\label{fig:methods}
\end{center}
\end{figure}



\section{Qualitative research}
\label{sec:qualitativeresearch}
A qualitative research method based on relatively few informants was used for this thesis. Unlike a quantitative approach where the use of questionnaires to gather information from a large number of participants is common, we wanted in-depth information from selected organizations. The qualitative research method enabled us to perform a rich and detailed analysis. 

The use of a quantitative method would have made statistical generalization possible, but it would have been more difficult to gather in-depth information. A survey may be easier to ignore, than a request for a face-to-face interview and a quantitative approach may not have given answers from the type of organizations we wanted. It may additionally be easier to get sincere answers in a face-to-face interview. In an interview the possibility to explain the questions is there. This is not the case for a survey, and it can be difficult to construct unambiguous questions that provides sufficient data for the analysis.

\pagebreak
Further, we used an inductive research approach which is defined as follows\cite{bhattacherjee2012social}: 

\textbf{Inductive research:} The objective is to infer theories and patterns from observed data. Also called \emph{theory-building} research.

In inductive research, researchers perform field studies followed by deriving theories from observations. This method is a contrast to deductive research where a theory is developed initially, followed by observations to evaluate it\cite{oates2005researching}.

\section{Case Study}
\label{sec:caseStudy}
\begin{figure}[H]
%\hspace*{-0.4cm}
\begin{center}
\includegraphics[scale=0.38]{caseProcess.png}
\caption[Case Study Research Process]{The linear, but iterative process of doing case study research \cite{CaseStudyResearch}}
\label{fig:caseProcess}
\end{center}
\end{figure}

\begin{figure}[H]
%\hspace*{-0.4cm}
\begin{center}
\includegraphics[scale=0.85]{caseTactics.png}
\caption[Case Study Tactics for Four Design Tests]{Case Study Tactics for Four Design Tests \cite{CaseStudyResearch}}
\label{fig:caseTactics}
\end{center}
\end{figure}

\subsection{Background Study}
\label{sec:background}
The first step in our research was a background study of incident management. We studied relevant literature such as standards and best practice guidelines to acquire sufficient knowledge. We focused on some of the well-established and internationally accepted ISO/IEC standards and documentation from the \ac{NIST} in addition to a few other guidelines. As part of the background study we reviewed related work. The background study was used to guide the data collection. It was also used in the data analysis, by comparing standards and previously developed theory to the findings of this study. This is consistent with the definition of a case study, as presented in the beginning of section \ref{sec:caseStudy}.

To gain additional knowledge and to get a realistic perspective on incident management, we attended two conferences addressing information security: one arranged by the \ac{NSM}\footnote{NSMs sikkerhetskonferanse 2013} and one arranged by The Norwegian Computer Society\footnote{Sikkerhet \& S\aa rbarhet 2013}.

\subsection{Qualitative Interviews}
\label{sec:interviews}
We chose to perform qualitative interviews as part of our research as they are a well-known and powerful tool for information collection in qualitative research\cite{myers2007qualitative}. The main objective of qualitative interviews is to see the research topic from the interviewees' perspective and understand why and how they got that particular perspective\cite{cassell2004essential}. To meet this objective, qualitative interviews are driven by open questions, a low degree of structure and often focus on specific situations and experiences made by the interviewee. 

We used what is referred to in literature as semi-structured interviews\cite{cassell2004essential}. To ensure we got all necessary information we used an interview guide. The interview guide worked as an incomplete script and states the main goals for our research as well as the main research questions and topics for the interview. The interview guide can be found in Appendix B (in Norwegian).

Questions were not asked in any pre-defined order during interviews. This enabled us to ask follow-up questions and ask for elaborations on certain topics. When using semi-structured interviews, interviewees can be seen as being participants in the research, rather than objects only answering pre-defined questions.

The interviews were performed face-to-face and voice recorded. We believe that conducting interviews face-to-face helped build trust with interviewees and thus gave better and more elaborative answers. It also gave us the opportunity to explain and elaborate questions that were unclear. As we recorded all of the interviews we could focus on listening and thus ask valuable follow-up questions instead of being distracted by writing down answers. Additionally, we could listen to the recordings several times as needed and clarify things that were unclear later. Challenges related to qualitative interviews are discussed in section \ref{sec:challenges}.

\subsection{Document Study}
\label{sec:documentStudy}
In case studies, documents are often used to verify or to question data obtained from other data collecting methods. To complement information gathered from interviews we studied relevant academic literature, standards and organization-specific documents such as policies and plans. This enabled us to compare standards, plans and current practice of incident management in the participating organizations.

When using documents in research one should be aware of possible bias and other elements that could compromise reliability\cite{oates2005researching}. In our case study we looked at both public and confidential documents.  We believe that by signing confidentiality agreements we were presented with authentic documents from the participating organizations. Nevertheless, we kept in mind that information could be outdated, not applicable or incorrect.   

\subsection{Employee Surveys}
\label{sec:employeeSurveys}
By studying documents and performing detailed interviews we got a thorough knowledge about routines related to incident handling. We found it interesting to examine how well these routines were established among the employees in the various organizations. To accomplish this we developed five main questions. These were asked to randomly selected employees who did not necessarily have any specific IT knowledge. The questions can be found in Appendix C (in Norwegian).
 
%\section{Action Research}

\subsection{Qualitative Data Analysis}
\label{sec:qualitativeAnalysis}
As discussed in section \ref{sec:qualitativeresearch} we have chosen a qualitative and inductive research method. For the data analysis we used a ``general inductive approach", as described by David R. Thomas \cite{thomas2006general}. He presents a systematic set of procedures for analysing qualitative data and explains a straightforward approach for deriving findings guided by research questions. We found this approach to be less complicated and more suitable than other approaches to qualitative data analysis such as grounded theory, phenomenology and ethnography research approaches\cite{thorne2000data}.

Inductive analysis is often guided by predefined research objectives. The use of research questions as guidance in data analysis undoubtedly sets constraints on the number of possible interpretations and outcomes as it draws attention to specific aspects of the data. However, using the general inductive approach rather than a stricter and more structured methodology, enabled findings to emerge from themes inherent in the raw data despite the pre-set research questions. Also, by using this approach, findings were not restricted by the methodology used. 

The main purposes of an inductive research approach are \cite{thomas2006general}: 
\begin{itemize}
\item to condense raw textual data into a brief, summary format;
\item to establish clear links between the evaluation or research objectives and the summary findings derived from the raw data and
\item to develop a framework of the underlying structure  of experiences or processes that are evident in the raw data.
\end{itemize}
The first one is fulfilled in chapter \ref{chp:findings} where data from each individual case are summarized. The last two are fulfilled in chapter \ref{chp:discussion}.

In chapter \ref{chp:discussion}, findings both directly linked to the research questions and findings that emerged independently from the data are discussed. This is compliant with the general inductive approach \cite{thomas2006general}:

\begin{quote}
\textit{``Although the findings are influenced by the evaluation objectives or
questions outlined by the researcher, the findings arise directly from the analysis of the raw data, not from a priori expectations or models."}
\end{quote}

As a first step in our analysis we perused the data and identified themes and categories that we found related to the research questions. The process from raw data to main findings and a conclusion can be outlined as follows:

\begin{enumerate}
\vspace{0,5cm}
\item Detailed readings of the qualitative data
\item Identifying specific themes that captured core messages given by participants.
\item Grouped themes into broader categories.
\item Primary findings are represented as a framework of themes.
\vspace{0,5cm}
\end{enumerate}

To verify credibility of our findings we sent summaries of the interviews to the participants. They were thereby given the opportunity to challenge our interpretations and comment on whether our findings were in compliance with their personal experience. 

\section{Participants}
The participating organizations in this study are all large Norwegian organizations. Their core activities belong to sectors identified by organizations such as \acs{NSM} to be especially exposed to attacks. Additionally, a study from Gj\o vik University College \cite{sand2010hendelseshaandtering} found large organizations to be better at establishing information security policies, defining information security incidents, conducting rehearsals based on their incident management plans and facilitating anonymous reporting. This could indicate that they are experienced and well equipped to handle information security incidents. We found it interesting to examine how such assumed experienced organizations perform incident management and what challenges they face. Additionally, the participating organizations have quite different organizational structures, which we believed could lead to interesting findings.

\section{Ethical Considerations}
\label{sec:ethical}
%Litt i forhold til at man behandler sensitiv information og saann
The main ethical concern related to our research is the potentially confidential information revealed during interviews. It is unlikely that organizations want details about their information security practices to be publicly known. Another important consideration was the privacy of the interviewees. Since a voice recorder was used during interviews, participants could potentially be identified later by voice recognition. To make sure that the participants knew exactly what they participated in, they were given information about how collected data was handled through a statement of consent. They were also given the right to withdraw from the study at any given time. This project was reported to the Norwegian Social Science Data services. The information sheet, including the statement of consent can be found in Appendix A (in Norwegian). 

As we got insight into confidential documents, we had to sign confidentiality statements beforehand.

The term anonymization means that any information that could directly identify individuals or individual organizations is deleted and that any information that indirectly could identify individuals or individual organizations is deleted or changed. No individuals or individual organizations are recognizable in this report. Participating organizations are given pseudonyms. All relations between individuals and individual organizations and results are anonymized at the end of the study, and only available to the students and partly their supervisors during the study. At the end of the study all recordings are deleted.

\section{Challenges}
\label{sec:challenges}
This case study relied on qualitative information and it was challenging and time consuming to report all findings correctly. Furthermore, as the interviews were conducted in Norwegian, correct translation enhanced this challenge. Additionally this type of research provides little basis for statistical generalization. \cite{CaseStudyResearch}

For quantitative data there exist clear conventions for analysis, but there are fewer guidelines for analysing qualitative data. As Allen S. Lee pointed out in \cite{lee1989scientific}, ``[...] the analyst faced with a bank of qualitative data has very few guidelines for protection against self delusion".

As most of the information collection was based on interviews, the challenges with this approach had to be considered. We had little or no experience in preparing and conducting qualitative interviews. We therefore tried to identify challenges and prepare the questions beforehand. Michael D. Myers and Michael Newman discusses potential challenges with qualitative interviews in\cite{myers2007qualitative}. They mention the artificiality of qualitative interviews where one interrogates a stranger that does not know or trust you. The lack of trust may cause the interviewee to withhold information that could be of value to the study. As an attempt to mitigate trust issues the procedure of handling data (anonymization) was presented and the fact that the project had been reported to the Norwegian Social Science Data Services was highlighted.   

Problems could arise if too little time is assigned for interviews. Time constraints could cause questions being rushed leading to interviewees giving inaccurate information or leaving out important information. To avoid time limitations being a concern, we assigned more time than estimated for each interview. We used the interview guide with predefined questions and topics as well as correcting any misunderstandings during the interview to avoid ambiguous questions.

When relying on qualitative interviews for information, one have to consider potential interviewee bias as for instance incident management knowledge vary greatly among employees in an organization. In addition, Myers and Newman mention the possibility for interviewees to construct knowledge to appear knowledgeable and rational. By giving interviewees enough time to answer questions and carry out interviews as a dialogue, we hope to have avoided these problem.

One challenge of using qualitative data is that the interpretation of information is somewhat based on researchers' background. Both are master students in communication technology with specialization in information security. As students with similar backgrounds and limited experience we believe that choosing an inductive qualitative research approach gave less bias in our results since we did not aim at proving a specific theory, but rather aimed at starting our information collection with open minds. Our similar backgrounds could lead to limitations when analysing data due to a potentially narrow perspective. However, this was somewhat mitigated by discussions with our supervisor.

A challenge related to empirical research is that it relies on other people. We experienced that it was at times difficult to make contact with people and this led to at times slower progress than desired.


