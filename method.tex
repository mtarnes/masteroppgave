\chapter{Method}
\section{Choice of Method}
\label{sec:choiceOfMethod}
Figure \ref{fig:methods} shows an overview of various research methods and three criteria that can be used to determine the appropriate research method. The criteria are: form of research question, whether the study requires control of behavioural events and if the study focuses on contemporary events. The defined research question for this study, as presented in section \ref{sec:objectives}, is a so-called ``how" question. As the goal of our study was to reveal current practices in organizations, we did not need control over behavioural events. This study's focus was mainly contemporary events. Some past events such as incidents that have occurred were relevant, but the main focus was on current practices. Based on this, case study emerged as the most suitable method for this study, as highlighted in the figure.

A case study is applicable to real-world organizations, which is what we wanted to study. An advantage is that it can deal with various kinds of evidence, such as documents, archival records, interviews and artefacts.

\begin{figure}[h]
\begin{center}
\includegraphics[scale=0.35]{methods.png}
\caption[Choice of Research Method]{Choice of Research Method, modified from \cite{CaseStudyResearch}}
\label{fig:methods}
\end{center}
\end{figure}



\section{Case Study}
\begin{figure}[H]
%\hspace*{-0.4cm}
\begin{center}
\includegraphics[scale=0.38]{caseProcess.png}
\caption[Case Study Research Process]{The linear, but iterative process of doing case study research \cite{CaseStudyResearch}}
\label{fig:caseProcess}
\end{center}
\end{figure}

\begin{figure}[H]
%\hspace*{-0.4cm}
\begin{center}
\includegraphics[scale=0.85]{caseTactics.png}
\caption[Case Study Tactics for Four Design Tests]{Case Study Tactics for Four Design Tests \cite{CaseStudyResearch}}
\label{fig:caseTactics}
\end{center}
\end{figure}

\subsection{Background Study}
\label{sec:caseStudy}
\label{sec:background}
The first step in our research were a background study of security incident management. In order to find appropriate questions for the interviews we found it necessary to study relevant literature such as standards and best practice guidelines. We focused on the well-established and internationally accepted ISO/IEC standards in addition to documentation from the \ac{NIST}.

We also looked at related work and what have been studied earlier in the field of incident management. In this way the background study was used to guide the data collection. It was also used in the data analysis, by comparing standards and prior developed theory tot he findings of this study. This is consistent with the definition of a case study, as presented in section \ref{sec:caseStudy}.

Also had to study ``case study" techniques.

\subsection{Qualitative research}
We used a qualitative research method where we focused on relatively few informants. Unlike a quantitative approach where using questionnaires to gather information from a large number of participants is common, we wanted in-depth information from selected organizations. This eliminated the possibility to generalize results, but gave us more exhaustive answers and thus better suited data for our research.

\subsection{Qualitative Interviews}
\label{sec:interviews}
We chose to perform qualitative interviews in our research as they are common and powerful tools to gather information in qualitative research\cite{myers2007qualitative}. The goal of qualitative interviews is to see the research topic from the interviewee's perspective and understand why and how they got that particular perspective\cite{cassell2004essential}. To meet this goal, qualitative interviews are driven by open questions and a low degree of structure and often focus on specific situations and experiences made by the interviewee. 

We used what is referred to in literature as semi-structured interviews\cite{cassell2004essential}. We wanted interviewees to talk freely about their experiences, and thus we did not follow a strict order of predefined questions. However, to ensure we got all information required to answer our research questions we used an interview guide. The interview guide works as an incomplete script and states the main goals for our research as well as the main research questions and topics for the interview.

The interview questions were not asked in any particular order, but rather when found appropriate to ensure a smooth dialogue. This gave opportunities to ask follow-up questions. When using semi-structured interviews, interviewees can be seen as being ``participants" in the research, rather than someone who only answers pre-defined questions.

We performed the interviews face-to-face voice recorded the dialogue. By conducting interviews face-to-face we hoped to build trust with the interviewee and thus get better and more elaborative answers. It also gave us opportunity to explain and elaborate questions that were unclear. Since we recorded all of the interviews we could focus on listening and thus ask the best follow-up questions instead of being extracted by writing answers. Also, we could listen to the recordings several times as needed and clarify things that were unclear later.

Developing an interview guide, carrying out interviews and analysing them can be highly time-consuming activities. Participating in interviews is also time-consuming for interviewees, which could have made it challenging to recruit interviewees for our study. One mitigation to the risk of not recruiting interviewees was sending out letters explaining the main purpose of our research, what was expected by the interviewee, and how the interviews would play out. We also notified organizations at an early stage and set dates for the interviews to ensure their commitment.

Qualitative interviews are very flexible as they can be used to tackle various forms of research questions related to organizations. Despite the time effort spent on developing an interview guide, questions and conducting the interviews, we found qualitative interviews to add great value and thus the most suited for our research.

\subsection{Document study}
\label{sec:documentStudy}
There are both advantages and disadvantages to document-based evidence. Documentation is stable, it can be reviewed repeatedly, it is not created as a result of the case study and may have broad coverage. However it can be difficult to find, biased selectivity may occur, there may exist some unknown bias of the author and some documents may be deliberately withheld. In this study, some of these disadvantages are attempted mitigated by trying to find all relevant documentation and by informing the participants of the data handling process and that sensitive information would not be revealed to any other party than the students and their supervisors. The latter was to try to avoid documentation being withheld.  

%\subsection{Action Research}

\subsection{Participants}
In the nature of qualitative research participants were chosen to be diverse such that different perspectives on incident management (in the same context) could be studied.
large organizations that are likely to have experiences severe security incidents. + de i Gjøvik gjorde på small/medium.. her gjør vi en dybdestudie av noe annet!

Hvorfor de vi valgte? Hvorfor er de interessante?

\section{Ethical Considerations}
\label{sec:ethical}
%Litt i forhold til at man behandler sensitiv information og sånn
The main ethical concern related to our research is the potentially confidential information revealed during interviews. Organizations unlikely want details about their information security practice to become public. Hence, no names are mentioned in this report that could identify participating organizations or individuals.

Since a voice recorder was used during interviews, participants could potentially be identified later by voice recognition. Participants were given information about how collected data was handled through a statement of consent and were also given the right to withdraw from the study at any given time. This project was reported to the Norwegian Social Science Data services. The information sheet, including the statement of consent can be found in Appendix A (in Norwegian).  

signere konfidensialitetserklæring?


\subsection{Anonymization}
\section{Challenges}
This case study relies on qualitative information and a challenge related to that is to work hard to report all evidence fairly. Additionally this type of research provides little basis for scientific generalization. \cite{CaseStudyResearch}.

For quantitative data there exist clear convention for analysis, but there are fewer guidelines for the analysis of quantitative data. In relation to this Allen S. Lee states in \cite{lee1989scientific}, ``but the analyst faced with a bank of qualitative data has very few guidelines for protection against self delusion".

Basing most of our information gathering on interviews, the challenges with this approach had to carefully considered. The authors had little or no experience in preparing and conducting qualitative interviews. Michael D. Myers and Michael Newman discusses potential challenges with using qualitative interviews in\cite{myers2007qualitative}. 

They mention the artificiality of qualitative interviews where one interrogates a stranger that does not know or trust you. The lack of trust may cause the interviewee to withhold information that could be of value to the study. As an attempt to mitigate trust issues the procedure of handling data (anonymization) was presented as well as highlighting that the project had been reported to the Norwegian Social Science Data Services.   

Problems could also arise if too little time is assigned for the interviews. Questions could be rushed causing inaccurate information or that important information are left out. To avoid time limitations being a concern, we assigned more time than estimated for each interview. We also used the interview guide with predefined questions and topics as well as correcting any misunderstandings during the interview to avoid ambiguous questions.

When relying on qualitative interviews for information, one have to consider potential interviewee bias as for instance security incident management knowledge vary greatly among employees in an organization. In addition, Myers and Newman mentions the possibility for interviewees to construct knowledge to appear knowledgeable and rational. By giving interviewees enough time to answer questions and carry out interviews as a dialogue, we hope to have avoided these problem.



