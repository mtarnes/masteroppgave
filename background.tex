\chapter{Background}
\label{chp:background}

\section{Why We Need Incident Management}
\label{sec:threatLandscape}
Modern society shows an increasing use of digital solutions. Today, digital solutions are vital to most organizations' day-to-day operations and large amounts of sensitive data are stored digitally \cite{KriposTrender}. As the value and sensitivity of information increases, the number of potential threats increase accordingly. This suggests that organizations today are more exposed to attacks than before. 

Organizations are increasingly using and depending on information technology in their operations. Attacks get more advanced and attackers choose their targets more strategically. A significant challenge arises when new and severe security threats evolve faster than corresponding measures. This leads to an increasing gap between threats and security measures in organizations. To avoid severe consequences such as disclosure of sensitive information this gap must be closed.

Despite organizations' implementation of information security policies and controls, it is inevitable that new vulnerabilities and information security incidents occur occasionally. Thus, it is essential that organizations have a structured and planned approach to detect, report, assess, respond to and learn from information security incidents \cite{ISO/IEC27035}. Preventive actions are not sufficient and an incident management capability is therefore necessary.

\begin{newquote}{Roar Thon, Senior Adviser \acs{NSM}}
\textit{"Everybody should do what they can to protect themselves from being attacked, but the sad truth is that the most important thing you should plan and prepare for is how to behave when the attacker has succeeded"}
\end{newquote}

This section discusses the current threat landscape and the need for plans in situations where systems have not been sufficiently secure. 

The information security threat landscape is continuously changing and new types of security-related incidents emerge frequently \cite{nist800-61}. %Preventive actions are not sufficient in order to be able to handle this and an incident response capability is therefore necessary. Even though incident prevention is not sufficient, it is an important complement to incident response. 

\acs{NSM} \acs{NorCERT}\footnote{\acs{NSM} is the Norwegian National Security Authority and is a cross-sectional professional and supervisory authority within the protective security services in Norway \cite{AboutNSM}. \acs{NorCERT} is the \acl{NorCERT} and is part of \acs{NSM}. \acs{NorCERT} coordinates preventive work and responses against IT security breaches aimed at vital infrastructure in Norway \cite{AboutNorCERT}} has registered a 30\% increase in cases each year for the past few years. They have seen an increase in cases of all impact levels. In addition they believe that there is a large number of incidents not reported or discovered \cite{NorCERT3Kvartal2012}. These findings are supported by Kripos\footnote{Kripos is the \ac{NCIS} in Norway and it is the unit for combating organized and other serious crime \cite{policeInNorway}.}, that reports ICT related crime to be expanding \cite{KriposTrender}. There is a large increase in targeted espionage operations directed towards Norwegian industry \cite{NSMRapport2012}. Attacks are mainly driven by \ac{ROI}, thus targets are chosen based on potential profit. Other incidents not necessarily motivated by money are strategic targets and domestic political monitoring as seen in China and Syria\cite{Morketall2012}.

\acs{NSM} states that the security condition in Norway for 2012 is not satisfactory \cite{samordnaVurdering}. This seems to be a continuing trend and the security condition for 2011 was summarized in the following way \cite{NSMRapport}: 

\begin{quote}
\textit{``The \textbf{values} we want to protect increase in amount, the \textbf{threats} are increasing, new \textbf{vulnerabilities} are constantly discovered, but \textbf{measures} to reduce these vulnerabilities are not developed at the same rate in addition to being inadequate."}
\end{quote}

Additionally, there is an increasing number of vulnerabilities discovered on smart phones and tablets, which represents a relatively new part of the threat landscape. There exist persistent vulnerabilities in organizations with classified information and these exist mainly due to lack of understanding of risks \cite{NSMRapport2012}.

In 2012 \acs{NorCERT} handled a large amount of serious cases related to espionage against Norwegian high-technology organizations \cite{NorCERT3Kvartal2012}. In a recent report PST\footnote{The Norwegian Police Security Service} expressed concerns related to Norwegian research and education environments being exploited to strengthen other nations' defences \cite{PSTvurdering2013}.   

Many of the current threats cannot be stopped by antivirus software. Attacks are increasingly becoming targeted to specific organizations in addition to becoming more advanced. Delay in updates and patches of computers is a big problem for many organizations \cite{NorCERT2Kvartal2012}. \acs{NSM} has observed a change in attacks from random and opportunistic attacks to advanced and focused attacks on specific targets of high economic or social value. In addition to technical means, attackers use social engineering\footnote{Social engineering involves manipulating people into performing certain actions such as disclosing sensitive information.} to obtain sensitive information or to obtain access to systems \cite{NSMRapport}. Another trend is attacks that compromise legitimate websites and infect all users that visit them \cite{NSMRapport2012}. Such attacks are called water-holing. They are particularly difficult to protect against as these exploit websites that users are normally allowed to visit.

There is great diversification in type of attackers. Attackers can belong to foreign intelligence, traditional military, global businesses, terrorist organizations, hacker groups or they can operate individually \cite{samordnaVurdering}. Criminals are organized in new ways, and various participants contribute with services, making attacks possible \cite{KriposTrender}. It is even possible to buy attacks like \ac{DDoS} attacks or spam distribution \cite{NorCERT2Kvartal2012}. Attacks can also come from the inside, either from an insider or by social engineering, and many organizations do not focus on this threat \cite{NSMRapport2012}. This expands the group of possible attackers, in practice it includes everyone. 

Several publications and recent reports highlight the need for incident management by pointing out deficiencies in organizations' information security. PST states that information security is given low priority in Norwegian government and private institutions \cite{PSTvurdering}. This is supported by \acs{NSM} that states that organizations seem to lack the ability and/or will to prioritize ICT-security \cite{NSMmelding}. Incident handling is often not prioritized and the severity of attacks are often not understood \cite{NorCERT2Kvartal2012}. Management's knowledge of information security is often insufficient, which is unfortunate as this is key to commitment of the rest of the organization \cite{NorCERT3Kvartal2012}. Systems for reporting security incidents to the management rarely exist and \acs{NSM} almost always discover that incidents have occurred or are occurring when they perform inspections\cite{NSMRapport}. Many organizations have inadequate contingency plans related to information security. Organizations also omit to conduct rehearsals related to preventive security and omit to rehearse their contingency plans \cite{NSMRapport2012}.

Several trends described here are not unique to Norway. Verizon Enterprise's RISK team published a report in cooperation with the \ac{USSS}, the Dutch \ac{NHTCU}, the \ac{AFP}, the \ac{IRISS} and the \ac{PCeU} of the London Metropolitan Police \cite{VerizonReport}. The report discusses data breaches in 2011 in 36 different countries. 855 incidents were analysed. It shows that 96\% of the (reported) attacks were not particularly advanced. It also shows that 85\% of the breaches took weeks or more to discover and that 92\% of incidents were discovered by a third party. 86\% of the breaches were caused organized crime. Based on the cases reported to the involved organizations, 2011 seems to be the year with the second highest number of data losses since 2004. The results of this report indicate that the overall international security condition is not satisfactory.

This shows a complex threat landscape with a large variety of attackers and with organizations that are not sufficiently prepared. It is not realistic to believe that all incidents can be prevented. In addition, it is not economically feasible. Hence, it is evident that organizations need plans and procedures to handle incidents \textit{when} they occur. The existence of an incident response capability in an organization can assist them in rapidly detecting incidents, minimizing loss and destruction, mitigating the weaknesses that were exploited and restoring computing services \cite{nist800-61}. 

\section{Incident Management Overview}
%Eller noe i den duren, måtte bare ha en section til subsectionen :p
\subsection{Definitions}
\label{sec:Definitions}
%Eller et annet navn, poenget er å includere definisjoner av disse tidlig i rapporten
In information security incident management there are a few terms that need to be defined clearly. Two such terms are information or computer security incidents\footnote{In this report the terms ``information security incident", ``computer security incident" and ``incident" are used interchangeably.} %NB! På slutten, sjekk at dette faktisk stemmer!
and information or computer security events. It is important to recognize these as two terms of different meaning. The standard \acs{ISO}/\acs{IEC} 27000 \cite{ISO/IEC27000} specifies the following definitions:

\textbf{Information security:} Preservation of confidentiality, integrity and availability of information; in addition, other properties such as authenticity, accountability, non-repudiation and reliability can also be involved.

\textbf{Information security event:} Identified occurrence of a system, service or network state indicating a possible breach of information security policy or failure of safeguards, or a previously unknown situation that may be security relevant.

\textbf{Information security incident:} Single or a series of unwanted or unexpected \emph{information security events} that have a significant probability of compromising business operations and threatening information security.

\textbf{\ac{ISIRT}:} Team of appropriately skilled and trusted members of the organization that handles information security incidents during their lifecycle.

The guidelines \acs{NIST} SP 800-61 \cite{nist800-61} specifies the following definitions:

\textbf{Event:} An event is an observable occurrence in a system or network.

\textbf{Adverse event:} Adverse events are events with a negative consequence, such as system crashes, packet floods, unauthorized use of system privileges, unauthorized access to sensitive data, and execution of malware that destroys data.

\textbf{Computer security incident:} A violation or imminent threat of violation\footnote{An ``imminent threat of violation" refers to a situation in which the organization has a factual basis for believing that a specific incident is about to occur.} of computer security policies, acceptable use policies, or standard security practices.

\acs{NorCERT} specifies the following definitions \cite{NorCERT3Kvartal2012}:

\textbf{\ac{CSIRT}:} A central tool with the task of protecting important infrastructure. The team must consist of security specialists and they must handle and responds to incidents. Additionally they need to create awareness and educators.

\textbf{\ac{CERT}:} A trademark that can only be used after approval by Carnegie Mellon University. Is in practice the same as a \acs{CSIRT}.

The definition of an adverse event from \cite{nist800-61} is quite similar to the definition of information security event from \cite{ISO/IEC27000}. The definitions of incidents are also quite similar. These definitions are the ones that will be used in this report. \ac{ISIRT}, \ac{CSIRT} and \ac{CERT} define similar types of teams. In this report the term \acs{IRT} is used to denote such a team. 

\subsection{What is Incident Management}
Incident management is a collective term composing all activities associated with managing security incidents. Incident management is not restricted to incident response alone, but includes activities for the entire incident lifecycle; from planning, training and raising awareness to detecting, responding and learning from incidents. 

Various guides and standards describe best practice and suggest activities for effective and efficient incident management. It is important to note that incident response requires a substantial amount of planning and resources. Some of the most important parts of incident response are the existence of guidelines for communication and prioritization of incidents as well as the use of a lessons learned process to gain experience from incidents. \cite{nist800-61}

As part of an incident management capability, organizations should have an incident response policy, an incident response plan and incident response procedures, all of which should be tailored to the specific organization's needs. Additionally it is important to have a planned approach to reporting of vulnerabilities that have not yet been exploited \cite{ISO/IEC27035}.

%The policy should include definitions, forms and commitment from senior management, plans should outline the organization's approach towards incident response, whereas procedures should be based on the current incident management policy and plan.

%\acp{SOP} are a delineation of the specific technical processes, techniques, checklists and forms used by the incident response team. An organization should establish procedures regarding communication with various outside parties, like media, law enforcement, other incident response teams, software vendors and \acp{ISP}. It is common to have \acp{PoC} for the various outside parties. \cite{nist800-61}

Incident management is not purely an IT issue since information security incidents threaten an organization as a whole. Having a well-planned and tailored incident management capability is therefore important for organizations in order to protect information. Incident management seeks to both prevent, contain and resolve incidents, in addition to post-learning. ``Incident management is an important tool of overall governance and to have it, in whatever form or shape, is a necessity\cite{enisaGuide}." 

\subsection{\acl{IRT}}
Experience indicate that security incidents occur occasionally, despite extensive preparation and security implementations. Thus, organizations need incident response to mitigate damage caused by incidents. Having an \ac{IRT} will aid organizations in responding to incidents more effectively and efficiently, in addition to providing a structured approach for learning from previous incidents. 

As the various definitions indicate in \ref{sec:Definitions}, an incident response team ``is a team that responds to computer security incidents by providing all necessary services to solve the problem(s) or to support the resolution of them\cite{enisaCSIRTGoodPractices}." The team structure, members, tasks and responsibilities may vary depending on organizations' resources and needs. 

\ac{NIST} recommends having one person in charge of incident response, taking the role as team manager. The team manager should act as a liaison to senior management as well as ensuring that the team has the necessary resources, personnel and skills. It is recommended that team members have diverse backgrounds so they can handle different incidents that may arise. The team manager should assess the situation and assign responsibility for incidents to the most appropriate team member.

Usually teams consist of highly technically skilled persons, and teams should have at least one member with expertise in each major technological category. However, it is not necessarily required that all team members are technical experts. Nevertheless, good problem solving skills and communication skills are essential to the team since effective incidents response require collaboration and coordination within the team and throughout the organization. 

The structure of the team may vary and can be dedicated, virtual or a mix of the two. Number and frequency of incidents as well as team responsibilities should guide organizations' choice of team structure \cite{ISO/IEC27035}. However, whenever justified the ISO/IEC 27035 standard recommends having a permanent team. Some large organizations may even choose to have several incident response teams, e.g. one team per division.

%\acp{IRT} should have clearly defined responsibilities, processes, allocation of roles and appropriate training programs\cite{ISO/IEC27035}. Team members should hold appropriate access permissions and have access to supportive tools to be best prepared to perform incident response \cite{SANShandbook}.
%
It is important for \acp{IRT} to maintain a good relationship to the organization's management. Management ensures proper support and funding for the team, thus building awareness around the added value the team provides as well as including management in training are essential. It is also important for ensuring that team members have the appropriate authority to make decisions whenever necessary in a crisis situation. 

The mission of the \ac{IRT} should reflect what is really important for the organization. The \ac{IRT} must define its constituency. The constituency is the organization or group of organizations and/or people whose incidents the team handles. \cite{enisaGuide}. A satisfied constituency is essential for the existence of \acp{IRT} and it is therefore important that the constituency is included in training to enhance engagement around security and to improve cooperation during incident handling. 

Participating in a community of teams will be beneficial for teams due to collaboration on standards and procedures as well as information and resource sharing. To minimize the frequency of incidents and to mitigate negative impact caused by them, most \acp{IRT} do not \emph{only} provide reactive services, but may also have other responsibilities, such as intrusion detection, advisory distribution, education and raising awareness within the organization\cite{nist800-61}.  
  

\section{Standards and Guidelines}
\label{section:standardsandguidelines}
\subsection{The \acs{ISO}/\acs{IEC} 27001 Standard}
\label{sec:iso27001}
This standard provides a model for establishing, implementing, operating, reviewing, maintaining and improving an \ac{ISMS}. It states that management shall provide evidence of its commitment to the ISMS. This section presents clauses relevant to incident management that are directly retrieved from the standard \cite{ISO/IEC27001}. 

\textbf{4.2.2 Implement and operate the \ac{ISMS}} \\
The organization should do the following.
\begin{enumerate}[h)]
\item Implement procedures and other controls capable of enabling prompt detection of security events and response to security incidents.
\end{enumerate}

This clause specifies that organizations should be able to detect and handle security incidents.

\textbf{4.2.3 Monitor and review the ISMS}\\
The organization shall do the following.
\begin{enumerate}[a)]
\item Execute monitoring and reviewing procedures and other controls to:
\begin{enumerate}[2)]
\item promptly identify attempted and successful security breaches and incidents;
\end{enumerate}
\vspace{-0.2cm}
\begin{enumerate}[4)]
\item help detect security events and thereby prevent security incidents by the use of indicators;
\end{enumerate}
\vspace{-0.2cm}
\begin{enumerate}[5)]
\item determine whether the actions taken to resolve a breach of security were effective.
\end{enumerate}
\item Undertake regular reviews of the effectiveness of the \ac{ISMS} (including meeting \ac{ISMS} policy and objectives, and review of security controls) taking into account results of security audits, incidents, results from effectiveness measurements, suggestions and feedback from all interested parties.
\end{enumerate}

\textbf{4.3.3 Control of records}\\
Records shall be kept of the performance of the process as outlined in 4.2 and of all occurrences of significant security incidents related to the \ac{ISMS}

Common for all clauses in this standard is that they only specify that things should be done, and not \textit{how} they should be done. The \acs{ISO}/\acs{IEC} 27002 standard\footnote{ \acs{ISO}/\acs{IEC} 27002 Information technology - Security techniques
- Code of practice for information security management} provides a code of practice for information security management and the \acs{ISO}/\acs{IEC} 27035 standard provides guidelines for the establishment of information security incident management. These standards are further described in sections \ref{sec:iso27002} and \ref{sec:iso27035} and can be used as aids to fulfil the clauses presented in the \acs{ISO}/\acs{IEC} 27001 standard.
\subsection{\acs{ISO}/\acs{IEC} 27002}
\label{sec:iso27002}
This standard represents a code of practise for information security management and establishes guidelines for initiating, implementing, maintaining and improving information security management in an organization. The standard is intended to work as a starting point for developing organization specific guidelines and contains 11 security control clauses that outline various security objectives and provides implementation guidance. It is emphasized that organizations should initially identify and establish its security requirements and then choose which of the security controls to implement.

This section describes parts of the standard that are relevant to incident management and the content is, unless specified otherwise, derived from \cite{ISO/IEC27002}.

\textbf{13.1 Reporting information security events and weaknesses } \\
The objective is to ensure that all significant information security events and weaknesses are reported such that corrective actions can be made in time. Reporting procedures and employee awareness are important success factors and it should be required to report any events or weaknesses to the point of contact as quickly as possible.

\textbf{13.1.1 Reporting information security events} \\
\emph{Control:} Information security events should be reported through appropriate management channels as quickly as possible.

\emph{Implementation guidance:} A point of contact and a formal event reporting procedure should be established and employees should be made aware of these. The reporting procedure should include the following.
\begin{enumerate}[a)]
\item suitable feedback processes to ensure that those reporting information security events are notified of results after the issue has been dealt with and closed.
\item information security event reporting forms to support the reporting action, and to help the person reporting to remember all necessary actions in case of an information security event.
\item the correct behaviour to be undertaken in case of an information security event.
\item reference to an established formal disciplinary process for dealing with employees, contractors or third party users who commit security breaches.
\end{enumerate}

\textbf{13.1.2 Reporting security weaknesses} \\
\emph{Control:} All employees, contractors and third party users of information systems and services should be required to note and report any observed or suspected security weaknesses in systems or services.

\emph{Implementation guidance:} There should exist an easy, accessible and available reporting mechanism for employees, contractors and third party users. Weaknesses should be reported as quickly as possible to either management or the service provider and not attempted to be proven.

\textbf{13.2 Management of information security incidents and improvements}\\
The objective is to ensure that the management of security incidents follows a consistent and effective approach where responsibilities and procedures are in place to handle incidents once they have been reported. Procedures should be in place for continual improvement of management processes. When necessary to collect evidence, this should be done in compliance with legal requirements.

\textbf{13.2.1 Responsibilities and procedures}\\ 
\emph{Control:} Management responsibilities and procedures should be established to ensure a quick, effective and orderly response to information security incidents.

\emph{Implementation guidance:} In addition to reporting, monitoring should be used to discover incidents. When implementing incident management procedures organizations should consider the following.
\begin{enumerate}[a)]
\item procedures should be established to handle different types of information security incidents, including:
\begin{enumerate}[1)]
\item information system failures and loss of service.
\item malicious code.
\item denial of service.
\item errors resulting from incomplete or inaccurate business data.
\item breaches of confidentiality and integrity.
\item misuse of information systems.
\end{enumerate}
\item in addition to normal contingency plans, the procedures should also cover:
\begin{enumerate}[1)]
\item analysis and identification of the cause of the incident.
\item containment.
\item planning and implementation of corrective action to prevent recurrence, if necessary.
\item communication with those affected by or involved with recovery from the incident.
\item reporting the action to the appropriate authority.
\end{enumerate}
\item audit trails and similar evidence should be collected and secured, as appropriate, for:
\begin{enumerate}[1)]
\item internal problem analysis.
\item use as forensic evidence in relation to potential breach of contract or regulatory requirement or in the event of civil or criminal proceedings, e.g. under computer misuse or data protection legislation.
\item negotiating for compensation from software and service suppliers.
\end{enumerate}
\item action to recover from security breaches and correct system failures should be carefully controlled. The procedures should ensure that:
\begin{enumerate}[1)]
\item only certain identified and authorized personnel are allowed access to live systems and data.
\item all emergency actions taken are documented in detail.
\item emergency action is reported to management and reviewed in an orderly manner.
\item the integrity of business systems and controls is confirmed with minimal delay.
\end{enumerate}
\end{enumerate}

\textbf{13.2.2 Learning from information security incidents}\\
\emph{Control:} There should be mechanisms in place to enable the types, volumes, and costs of information security incidents to be quantified and monitored.

\emph{Implementation guidance:} By monitoring incidents, reoccurring and high impact incidents can be identified and need for additional controls can be evaluated.

\textbf{13.2.3 Collection of evidence}\\
\emph{Control:} Where a follow-up action against a person or organization after an information security incident involves legal action (either civil or criminal), evidence should be collected, retained, and presented to conform to the rules for evidence laid down in the relevant jurisdiction(s).

\emph{Implementation guidance:} The rules of evidence involve admissibility and weight of evidence, that is whether or not evidence can be used in court and the quality and completeness of the evidence. To achieve admissibility and weight of evidence, organizations should ensure their systems comply with standards and that controls used to protect evidence are complete and consistent.

\subsection{\acs{ISO}/\acs{IEC} 27035}
\subsection{ITIL}
\ac{ITIL} is a framework and a source of good practice in service management. The \acs{ISO}/\acs{IEC} 27000 standard is aligned with \ac{ITIL}. This section gives a brief introduction to \ac{ITIL}, focusing on the parts related to incident management and the content is, unless specified otherwise, derived from \cite{itilbok}. The definitions presented in this section are taken directly from \cite{itilbok}.

To describe service management, the \ac{ITIL} framework uses the following definitions:

\textbf{Service:} A service is a means of delivering value to customers by facilitating outcomes that customers want to achieve without the ownership of specific costs and risks.

\textbf{Service Management:} Service management is a set of specialized organizational capabilities for providing value to customers in the form of services.

The specialized organizational capabilities include the processes, activities, functions and roles that a service provider uses in delivering services. The framework is generic and is meant to be useful for any type of organization. It describes a set of functions and processes that can be implemented in order to be able to perform service management. The terms function and process are defined in the following ways:

\textbf{Function:} A team or group of people and the tools they use to carry out one or more processes or activities.

\textbf{Process:} A process is a structured set of activities designed to accomplish a specific objective. A process takes one or more defined inputs and turns them into defined outputs. A process may include any of the roles, responsibilities, tools and management controls required to reliably deliver the outputs. A process may define policies, standards, guidelines, activities and work instructions if they are needed.

%Risk is defined as a possible event that could cause harm or loss, or affect the ability to achieve objectives. Risk can also be defined as the uncertainty of outcome.

This section describes processes and functions related to incident management.

\paragraph{Availability Management}
Availability management is essential for an organization and it is primarily a proactive process. In addition to activities such as preparing and maintaining an availability plan and monitoring availability levels this process includes assisting with investigation and resolution of availability-related incidents and problems. The latter is a reactive part of availability management. This process is related to other processes including IT service continuity, information security, event, incident and problem management. The availability manager is responsible for this process.

\paragraph{IT Service Continuity Management}
This process is concerned with key systems in the event of a failure. The purpose of the process is to ensure that IT resources, systems and services can be restored within agreed timescales in the event of a major incident. The process is related to availability and information security management. The IT service continuity manager is responsible for ensuring that the objectives of this process are met.

\paragraph{Information Security Management}

\paragraph{The Service Desk}
The service desk is a function.

\paragraph{Incident Management}

\paragraph{Problem Management}

\paragraph{Event Management}
\subsection{\acs{NIST} Special Publication 800-61}
This subsection gives an introduction to the guidelines \acs{NIST} SP 800-61 and the content is, unless specified otherwise, derived from \cite{nist800-61}. This publication aims to  assist organizations in mitigating risks from computer security incidents by providing guidelines on how to respond to incidents effectively and efficiently. 

One of the first considerations for a \ac{CSIRC} should be to agree on a definition of the term incident. This guidelines' definitions of events and incidents are included in section \ref{sec:Definitions} of this report. 

\acs{NIST} SP 800-61 describes the four phases of incident response; preparation, detection and analysis, containment, eradication and recovery and post-incident activity. The phases and the relationship between them are illustrated in figure \ref{fig:NISTIncidentResponse}.

\begin{figure}[ht]
%\hspace*{-0.4cm}
\begin{center}
\includegraphics[scale=0.27]{NISTIncidentResponseCycle.png}
\caption[The Incident Response Life Cycle]{The Incident Response Life Cycle \cite{nist800-61}}
\label{fig:NISTIncidentResponse}
\end{center}
\end{figure}

\paragraph{Preparation} 
This phase includes establishing an incident response capability as well as preventing incidents. The latter is not typically a part of the \ac{IRT}'s tasks, but it is fundamental to the success of the organization's incident response. If a large number of incidents occur, it may overwhelm the \ac{IRT}. To prepare for incidents the incident handlers should have tools and resources such as contact information, incident reporting mechanisms, issue tracking system, digital forensic workstations\footnote{A digital forensic workstation is specially designed for acquiring and analysing data. It usually contains a set of removable hard drives that can be used for evidence storage.} and digital forensic software. It is common to create a portable \emph{jump kit} containing materials that may be needed during incident response.

\paragraph{Detection and Analysis}
Organizations should prepare to handle any type of incident in addition to common incident types. A classification of incidents can be used as a basis for incident handling. The guideline provides a list of example categories for incidents that contains web, email, improper usage and loss or theft of equipment. It focuses on all kinds of incidents and does not address specific incident categories. A challenge related to incident handling is to detect the incident and determine the potential impact the incident may have. The actual detection may be the hardest part of incident handling. The guideline defines two types of signs of incidents; precursors and indicators, with indicators being the most common. These are defined in the following way: "A \emph{precursor} is a sign that an incident may occur in the future. An \emph{indicator} is a sign that an incident may have occurred or may be occurring now." Common sources for precursors and indicators are \acp{IDPS}, antivirus and antispam software, third-party monitoring services, logs, information on new vulnerabilities and exploits and people. 

A challenging part of this phase is the analysis, i.e. to determine which indicators and precursors are legitimate, if they are really related to an incident and what has actually happened. When the team believes an incident to have occurred they should try to determine the scope. All steps taken should be documented and timestamped. It is important to note that any such documentation can be used in court. The incident response team should maintain a database containing information about incidents, such as status, indicators, related incidents and actions taken by the incident handlers. It is important to prioritize incidents and to handle them accordingly. Factors that can be used as a basis for prioritization include the functional impact of the incident, the information impact of the incident and recovery from the incident. When the prioritization is performed, the \ac{IRT} should notify the appropriate people. It is important to have procedures regarding who these people should be.

\paragraph{Containment, Eradication and Recovery}
Containment is obviously an important part of incident handling. The existence of strategies and procedures for containment is helpful. These strategies and procedures are different for different types of incidents. Gathering and handling of evidence are part of this phase. For some incidents eradication is necessary and it is sometimes done during recovery. Eradication can include deleting malware and disabling breached user accounts. Recovery consists of restoring systems to normal operations and in some cases eliminating vulnerabilities that could cause similar incidents. The guideline does not offer specific recommendations for eradication and recovery as these are often OS specific. 

\paragraph{Post-Incident Activity}
Learning and improving is one of the most important parts of incident response. It is recommended to hold a ``lessons learned" meeting after each major incident and periodically after minor incidents. One meeting could potentially cover several incidents. ``Lessons learned" meetings should generally focus on revealing what was done well and what could be improved. The desired result is that the organization will be better equipped for the next incident. Often, incident response policies and procedures are updated. Areas these meetings should focus on are how well the staff performed and what they could have done differently, if documented procedures were followed and if they were adequate and how information sharing with other organizations could have been improved. To prevent similar incidents in the future, potential corrective actions and potential additional tools and resources should be reviewed. Both people involved in the incident(s) in question and people needed for future cooperation should be included in these meetings. A follow-up report that provides a reference that can be used when handling similar future incidents should be created. Other post-incident activities include the use of collected data for risk assessment, measurement processes to determine the success of the incident response team and audits of incident response programs. 




\subsection{\acs{ENISA} - Good Practice Guide for Incident Management}
This guide is developed by the \ac{ENISA} and provides a description of good practices for security incident management. The content is, unless specified otherwise, derived from \cite{enisaGuide}. The focus of this guide is IT and information security incidents. It specifically addresses the incident handling part of incident management. The incident management and incident handling processes are illustrated in figure \ref{fig:ENISAIncidentManagement}. The incident handling process has four major components, as shown in the figure. 

\begin{figure}[h]
\begin{center}
\includegraphics[scale=0.68]{enisaIncidentManagement.png}
\caption[ENISA Incident Management and Incident Handling]{Incident Management and Incident Handling \cite{enisaGuide}}
\label{fig:ENISAIncidentManagement}
\end{center}
\end{figure}

\paragraph{Detection:} The \ac{CERT} can receive incident reports from various sources. This guide recommends to use e-mail as a communication channel as people prefer this. Additionally it recommends to not only work with reports sent by others, but also use monitoring systems. Detection also includes registration of incident reports in an incident handling system. This stage is a good place to implement pre-filtering mechanism for incident reports. The registration process could include the use of an incident report form.

\paragraph{Triage:} This stage consists of the three phases verification, initial classification and assignment. During these phases the following questions should be answered:

\begin{itemize}\itemsep-0.2cm
\item Is it really an IT security incident?
\item Is it related to one of your constituents?
\item Does it fit within the mandate the \ac{CERT} has?
\item What is the impact?
\item Is there collateral damage?
\item How fast could it spread to other constituents?
\item How many people do you need to handle this incident?
\item Which incident handler should be appointed to the incident?
\end{itemize}

The verification phase seeks to answer the first question. It is however recommended to respond to and archive all reports, even those not defined as information security incidents. They may include information relevant to other incidents or lead to an incident. After an incident report has been verified the incident should be initially classified according to a classification schema. The last part of the triage component is to assign the incident to an incident handler.

\paragraph{Analysis and Incident response:} These components are illustrated by figure \ref{fig:IncidentResolutionCycle}. This cycle may need to be performed several times. To perform \textit{data analysis} there should be collected as much data as possible. Prior to the collection all involved parties should be notified. Sources for data collection could be an incident reporter, monitoring systems, referring database and relevant log files. The collected data should be used to try to determine the source of the incident. Prior to the data analysis, decisions about what data to analyse and in what order must be made. During the analysis people will often exchange ideas and observations as well as draw conclusions. This belongs to the \textit{resolution research}. It is recommended to advise team members to write down any observations that can be discussed in review meetings. The \textit{action proposed} part consists of preparing a set of tasks for each party involved. The \textit{action performed} should be monitored, where possible. The main goal for all actions is the \textit{eradication and recovery}.

\begin{figure}[h]
\begin{center}
\includegraphics[scale=0.4]{IncidentResolutionCycle.png}
\caption[ENISA Incident Resolution Cycle]{Incident Resolution Cycle \cite{enisaGuide}}
\label{fig:IncidentResolutionCycle}
\end{center}
\end{figure}

When you have left the incident resolution cycle, there are still tasks to perform. The incident needs to be closed properly. Each involved party needs to be informed that the incident is resolved. The classification of the incident should be revisited and a final classification should be performed. The classification could have been revisited during the resolution cycle as well. It is recommended to have a taxonomy and to classify incidents in accordance with it.

After an incident has been resolved or closed a post-analysis should be performed in order to learn from it. The guide advises to wait some time after closure before performing the analysis, so people can look at it with fresh minds. It is also recommended not to analyse all incidents, but only the most characteristic and complex ones and those that include new attack vectors. 

Incidents should be reported to the management. In addition to specific issues, the daily operations should be reported, including costs, positive results, plans and risks. This will save time and resources in situations where you need the management's operational or financial support and quick decisions.





\subsection{NorSIS' Guideline for Incident Management}
The \ac{NorSIS} has in cooperation with a group of students\footnote{The students did a survey on incident management in Norwegian \acsp{SME}\cite{sand2010hendelseshaandtering}} developed a guideline for incident management, published in 2010\cite{norsisveiledning}. The aim of this guideline is to give a thorough description of why and how organizations should plan for security incident management, do business impact analysis and explain various measures to improve information security in organizations. The guideline distinguish between minimum requirements and general recommendations. The content in this section is, unless specified otherwise, derived from\cite{norsisveiledning}.

\paragraph{Incident Management Policy} 
An incident management policy should form the basis for developing new incident management plans in organizations. A solid policy should state an organization's objectives for incident management and include a statement ensuring commitment from senior management. Any relevant laws, standards and regulations should also be included. It is essential that the policy has requirements for performing regularly risk assessment, business impact analysis, tests and training. The guideline also suggests assigning roles and responsibilities as part of the incident management policy.

\paragraph{Business Impact Analysis}
NorSIS suggests that organizations conduct a business impact analysis to identify which services are of significant value and needs to be secured. Conducting risk assessment and identifying possible consequences of security incidents are part of this process. The guideline emphasize the importance of knowing risks and potential threats.  

\paragraph{Preventive Measures}
One of the most cost effective ways to do incident management is implementing preventive measures. Listed as minimum requirements are anti-virus, logs, firewalls, backups, alarms, locks, regularly reviews of threat landscape, and report systems for employees. Other proposed measures include encryption of data and wireless networks.   

\paragraph{Recovery Strategies}
The guideline recommends having a recovery strategy to quickly re-establish business operations after an incident. Suggestions include backup and emergency solutions. Routines and plans should be in place to handle recovery efficiently.

\paragraph{Incident Management Plan}
Organizations should use previous assessments and proposed incident scenarios to develop an incident management plan. It is recommended that individual plans addressing different scenarios are developed. Each incident management plan should state type of incident, what triggered the incident, roles and responsibilities, guidelines for communication and notification, maximum response time, check-list of tasks during incident response and post-incident activities.

\paragraph{Training}
To reduce costs caused by security incidents, NorSIS suggests training employees in correct use of equipment and make sure routines for incident response are well known.

\paragraph{Plan Maintenance}
The guideline recommends that organizations conduct yearly reviews of their incident management plans. To ensure a solid and up-to-date incident management plan, changes should be made based on experience from previous incidents. 

\paragraph{Outsourcing}
When organizations decide to outsource services, they should evaluate and agree on incident management procedures. It is the organization outsourcing that is responsible for securing information properly and to make sure sufficient plans for incident management exist. An agreement should define responsibilities and state expected quality of services. 

\paragraph{Example Threat Scenarios}
There should be no hesitations among employees when responding to typical threat scenarios. Examples of some security incidents are given and they aim to help organizations decide whether they have a sufficient incident management. Some examples are:
\begin{itemize}
\item Backup is lost due to a fire
\item The organization's network is attacked by a virus
\item There has been a burglary and two servers are stolen 
\end{itemize}

\subsection{SANS: Incident Handler's Handbook}
This section gives an introduction to SANS' Incident Handler's Handbook and the content is, unless specified otherwise, derived from \cite{SANShandbook}. The purpose of this document is to provide sufficient information for IT professionals and managers to create incident response policies, standards and teams for their organization. Six phases of incident management are described and recommended to be followed in sequence as each phase builds on the previous one. 

\paragraph{Preparation} 
This is the most crucial phase as it determines how well the incident response team will be able to respond to security incidents. During this phase, several key elements should be implemented to avoid potential problems while responding to security incidents.

Organizations should develop a policy stating the organization's principles, rules and practices. After the establishment of a security policy, organizations should develop a response plan with a prioritization of incidents based on organizational impact. Having this prioritization scheme could aid in obtaining necessary resources for incident management by ensuring commitment from senior management as they will better understand risk and business impact. It is also recommended to have a communication plan so the response process is not delayed by uncertainty of whom to contact in unexpected situations. These plans should also state when it is appropriate to contact law enforcement.

Documenting incidents is beneficial for organizations. A thorough documentation is useful for lessons learned and might also serve as evidence if an incident is considered a criminal act. The establishment of a \ac{CIRT} is part of the preparation phase. It is vital that also their activities are documented properly. 

\paragraph{Identification} 
The first step of this phase is identification of security events by detecting deviations from ``normal" operations within the organization. This is followed by a decision of whether the event is categorized as an incident. Organizations should implement various tools to gather documentation about events, such that incidents and patterns can be identified. Examples of such tools include \acp{IDS}, firewalls and log files. Typically, incidents are reported to the \ac{CIRT} that decides the scope of the incidents and how to move forward.

\paragraph{Containment} 
In this phase organizations try to limit the damage and prevent further damage caused by security incidents. It is recommended to isolate compromised systems to avoid escalation. An easy measure could be to disconnect affected parts of the system. 

Several steps are necessary for a successful incident response. The first step is called short-term containment and is concerned with limiting the damage by implementing short-term but effective measures. The second step is concerned with ensuring proper back-up of information before system resources can be restored. The final step is called long-term containment and involves removing alternations made by an attacker, installing security patches and limiting further escalation of the incident.

\paragraph{Eradication} 
Affected assets and systems are restored during this phase. To avoid similar incidents in the future, defences should be improved. Continuous documentation is important in this phase to ensure that proper steps were taken in previous phases in addition to determine the overall impact on the organization. It is recommended that all affected systems are scanned with anti-malware software to ensure that all potential latent malware is removed. 

\paragraph{Recovery} 
Activities in this phase include bringing affected systems back into operation and preventing future incidents caused by the same problem as previous incidents. Other activities are testing, monitoring and validating systems to ensure they are not reinfected. 

\paragraph{Lessons Learned} 
The final phase's main objectives are to learn from incidents to improve the CIRT's performance and to provide material to aid in future incident responses. An important activity is to hold a post-incident meeting that summarizes the incident management process. This phase evaluates an organization's incident management procedures and identifies areas of improvement.

\section{Related Work}
In recent years the amount of available academic literature addressing incident management has increased along with overall interest for the topic. However, despite the amount of available literature there is limited knowledge about how organizations apply incident management in practice and thus an interesting topic for research. As a starting point we looked at related research papers and surveys and we discuss some of them briefly in this section.

Eugene H. Spafford \cite{spafford2003failure} wrote a paper in 2003 where he presented the first large internet worm and discussed what happened during the years after this large incident, which occurred in 1988. The worm led to the CERT at Carnegie-Mellon University being established. The three flaws that this worm exploited were trust relationships, buffer overflows and poor default configuration. The author claims that these flaws have not been removed but rather worsened. The author also questions the CERT model. He claims that incident response is uncoordinated and of minimal effectiveness. Lastly he expects that he can in 2013 or 2018 write a paper about 2003 as the time were we did not know how bad it was going to get. This work is quite different from our thesis, but it is interesting that he points to lack of lessons learned and predicts that the situation in the time of this writing is going to be bad.

In 2005 Jamie Riden \cite{riden2005responding} wrote a paper based on a case study conducted over two years where he studied security incident response in a large academic network. The study is mainly a description of responses to specific incidents. It does not focus on plans and procedures, which is an important part of our study. Riden focuses on why the incidents happened and what could have been done to prevent them while recognizing that not all incidents can be prevented.

In a study from 2005\cite{brage}, a survey of Norwegian companies and public institutions was conducted looking at routines for information security incidents, how theory and practice differed as well as potential differences between organizations in public and private sectors. The survey showed that statistical material about incidents were inaccurate due to lack of implemented routines, lack of training and weak definitions of security incidents in general. Public institutions were found to have greater shortcomings in reporting, training and statistics than private ones. A lack of documentation and use of metrics when outsourcing IT-systems were also revealed. Of all the participating organizations only half followed international standards for information security. Further, the study disclosed a gap between incident management theory and practice in terms of how organizations handle information security incidents. Even though private organizations were found to have overall better incident management, there were still room for improvements, especially regarding reporting, training and statistics. Keeping track of how previous incidents were handled can be an important part of a learning and improvement process and we therefore wish to look at to what extent previous incidents have been handled in accordance with plans and how they were used to improve current procedures.

In 2007 Werlinger et al. \cite{werlinger2007detecting} conducted an exploratory study using interviews and questionnaires to reveal what tasks security practitioners performs during security incidents, what skills and tools are necessary and what strategies are required in order to deal with security incidents. They grouped tasks into the main stages detection, analysis and response. They identified pattern recognition, hypothesis generation and cooperation as needed skills. Two identified strategies in incident response were isolation and simulation. The study is somewhat similar to our study, but it lacks detailed information and analysis.

Werlinger et al. \cite{werlinger2010preparation} conducted in 2009 16 semi-structured interviews with IT security practitioners from seven organizational types. Their research focused on diagnostic work performed in response to security incidents as well as the tools used in this process. Their findings show that a great deal of tacit knowledge is used in the diagnostic work. In addition to relying on tools the employees used their own technical knowledge as well as their knowledge of the organization and its systems. The findings also showed that intensive diagnostic work was needed to be able to respond to security incidents. This research differentiate from our research in the sense that they focus mainly on diagnosis and the tools used for that and not the entire process of incident management. Additionally there is no comparison with existing standards and guidelines in the analysis of the data.

In 2010, a group of students at HiG\footnote{H\o gskolen i Gj\o vik (Norwegian)} did a survey of incident management policies, implementations, training and routines in Norwegian \acp{SME}\cite{sand2010hendelseshaandtering}. They performed interviews and questionnaires and concluded that there was still room for improvement regarding incident management in Norwegian \acp{SME}. Also, having a chief of information security was shown to be beneficial as those organizations tended to have better plans for incident management in addition to using their plans more often. Additionally, how well organizations perform incident management was shown to differ between public and private sector. Their research indicate overall insufficient plans for incident management among Norwegian \acp{SME}, and poor quality in existing plans. One of our research objectives is thus identifying what plans and procedures are established in organizations. Finally, the students proposed in cooperation with the Norwegian Centre for Information Security (NorSIS) a guide for incident management customized for \acp{SME}. Since then, both new standards and guidelines have been published addressing incident management. It is thus interesting to look at how organizations perform incident management and how these standards and guidelines are adopted in current plans and procedures for incident management.

Incident response teams are of utmost importance to incident management. We therefore found research related to \acp{IRT}' tasks, structure and responsibilities interesting. As described in section \ref{section:standardsandguidelines}, several guides address establishment and running of \acp{IRT} and a few studies also look at how \acp{IRT} operate in practice. In 2003, Killcrece et al. \cite{killcrece2003state} looked at current state of practice for \acp{IRT} and found several shortcomings for teams in general such as lack of tools, training and experienced personnel. However, during the past decade new standards and guidelines have emerged and the field of incident management has matured significantly.  Based on this and several other studies, Ahmad et al. \cite{ahmad2012incident} presents a case study exploring issues faced by incident response teams that affect the greater organizational security function. They found organizations lacked the ability to exploit their organizational learning capability. A lack of proper information dissemination and organizations tending to focus on technical learning over policy and risk were also discussed. Further, Wiik et al. \cite{gonzalezlimits} presents a simulation model to better understand the main factors influencing an \ac{IRT}'s effectiveness. They identified that ``short-term pressure from a growing incident work load prevents attempts for developing more response capability long-term". 

While studying related work we came to understand that threat landscape, standards and best practise guidelines change rapidly. Surveys conducted only few years apart reveal that information security and incident management are maturing. Thus, we find looking at how organizations perform incident management in practice highly relevant. 

