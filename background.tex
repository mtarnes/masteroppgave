\chapter{Background}
\label{chp:background}

\section{Why We Need Incident Management}
\label{sec:threatLandscape}
Modern society shows an increasing use of digital solutions. Today, digital solutions are vital to most organizations' day-to-day operations and large amounts of sensitive data are stored digitally \cite{KriposTrender}. As the value and sensitivity of information increases, the number of potential threats increase accordingly. This suggests that organizations today are more exposed to attacks than before. 

Organizations are increasingly using and depending on information technology in their operations. Attacks get more advanced and attackers choose their targets more strategically. A significant challenge arises when new and severe security threats evolve faster than corresponding measures. This leads to an increasing gap between threats and security measures in organizations. To avoid severe consequences such as disclosure of sensitive information this gap must be closed.

Despite organizations' implementation of information security policies and controls, it is inevitable that new vulnerabilities and information security incidents occur occasionally. Thus, it is essential that organizations have a structured and planned approach to detect, report, assess, respond to and learn from information security incidents \cite{ISO/IEC27035}. Preventive actions are not sufficient and an incident management capability is therefore necessary.

\begin{newquote}{Roar Thon, Senior Adviser \acs{NSM}}
\textit{"Everybody should do what they can to protect themselves from being attacked, but the sad truth is that the most important thing you should plan and prepare for is how to behave when the attacker has succeeded"}
\end{newquote}

This section discusses the current threat landscape and the need for plans in situations where systems have not been sufficiently secure. 

The information security threat landscape is continuously changing and new types of security-related incidents emerge frequently \cite{nist800-61}. %Preventive actions are not sufficient in order to be able to handle this and an incident response capability is therefore necessary. Even though incident prevention is not sufficient, it is an important complement to incident response. 

\acs{NSM} \acs{NorCERT}\footnote{\acs{NSM} is the Norwegian National Security Authority and is a cross-sectional professional and supervisory authority within the protective security services in Norway \cite{AboutNSM}. \acs{NorCERT} is the \acl{NorCERT} and is part of \acs{NSM}. \acs{NorCERT} coordinates preventive work and responses against IT security breaches aimed at vital infrastructure in Norway \cite{AboutNorCERT}} has registered a 30\% increase in cases each year for the past few years. They have seen an increase in cases of all impact levels. In addition they believe that there is a large number of incidents not reported or discovered \cite{NorCERT3Kvartal2012}. These findings are supported by Kripos\footnote{Kripos is the \ac{NCIS} in Norway and it is the unit for combating organized and other serious crime \cite{policeInNorway}.}, that reports ICT related crime to be expanding \cite{KriposTrender}. There is a large increase in targeted espionage operations directed towards Norwegian industry \cite{NSMRapport2012}. Attacks are mainly driven by \ac{ROI}, thus targets are chosen based on potential profit. Other incidents not necessarily motivated by money are strategic targets and domestic political monitoring as seen in China and Syria\cite{Morketall2012}.

\acs{NSM} states that the security condition in Norway for 2012 is not satisfactory \cite{samordnaVurdering}. This seems to be a continuing trend and the security condition for 2011 was summarized in the following way \cite{NSMRapport}: 

\begin{quote}
\textit{``The \textbf{values} we want to protect increase in amount, the \textbf{threats} are increasing, new \textbf{vulnerabilities} are constantly discovered, but \textbf{measures} to reduce these vulnerabilities are not developed at the same rate in addition to being inadequate."}
\end{quote}

Additionally, there is an increasing number of vulnerabilities discovered on smart phones and tablets, which represents a relatively new part of the threat landscape. There exist persistent vulnerabilities in organizations with classified information and these exist mainly due to lack of understanding of risks \cite{NSMRapport2012}.

In 2012 \acs{NorCERT} handled a large amount of serious cases related to espionage against Norwegian high-technology organizations \cite{NorCERT3Kvartal2012}. In a recent report PST\footnote{The Norwegian Police Security Service} expressed concerns related to Norwegian research and education environments being exploited to strengthen other nations' defences \cite{PSTvurdering2013}.   

Many of the current threats cannot be stopped by antivirus software. Attacks are increasingly becoming targeted to specific organizations in addition to becoming more advanced. Delay in updates and patches of computers is a big problem for many organizations \cite{NorCERT2Kvartal2012}. \acs{NSM} has observed a change in attacks from random and opportunistic attacks to advanced and focused attacks on specific targets of high economic or social value. In addition to technical means, attackers use social engineering\footnote{Social engineering involves manipulating people into performing certain actions such as disclosing sensitive information.} to obtain sensitive information or to obtain access to systems \cite{NSMRapport}. Another trend is attacks that compromise legitimate websites and infect all users that visit them \cite{NSMRapport2012}. Such attacks are called water-holing. They are particularly difficult to protect against as these exploit websites that users are normally allowed to visit.

There is great diversification in type of attackers. Attackers can belong to foreign intelligence, traditional military, global businesses, terrorist organizations, hacker groups or they can operate individually \cite{samordnaVurdering}. Criminals are organized in new ways, and various participants contribute with services, making attacks possible \cite{KriposTrender}. It is even possible to buy attacks like \ac{DDoS} attacks or spam distribution \cite{NorCERT2Kvartal2012}. Attacks can also come from the inside, either from an insider or by social engineering, and many organizations do not focus on this threat \cite{NSMRapport2012}. This expands the group of possible attackers, in practice it includes everyone. 

Several publications and recent reports highlight the need for incident management by pointing out deficiencies in organizations' information security. PST states that information security is given low priority in Norwegian government and private institutions \cite{PSTvurdering}. This is supported by \acs{NSM} that states that organizations seem to lack the ability and/or will to prioritize ICT-security \cite{NSMmelding}. Incident handling is often not prioritized and the severity of attacks are often not understood \cite{NorCERT2Kvartal2012}. Management's knowledge of information security is often insufficient, which is unfortunate as this is key to commitment of the rest of the organization \cite{NorCERT3Kvartal2012}. Systems for reporting security incidents to the management rarely exist and \acs{NSM} almost always discover that incidents have occurred or are occurring when they perform inspections\cite{NSMRapport}. Many organizations have inadequate contingency plans related to information security. Organizations also omit to conduct rehearsals related to preventive security and omit to rehearse their contingency plans \cite{NSMRapport2012}.

Several trends described here are not unique to Norway. Verizon Enterprise's RISK team published a report in cooperation with the \ac{USSS}, the Dutch \ac{NHTCU}, the \ac{AFP}, the \ac{IRISS} and the \ac{PCeU} of the London Metropolitan Police \cite{VerizonReport}. The report discusses data breaches in 2011 in 36 different countries. 855 incidents were analysed. It shows that 96\% of the (reported) attacks were not particularly advanced. It also shows that 85\% of the breaches took weeks or more to discover and that 92\% of incidents were discovered by a third party. 86\% of the breaches were caused organized crime. Based on the cases reported to the involved organizations, 2011 seems to be the year with the second highest number of data losses since 2004. The results of this report indicate that the overall international security condition is not satisfactory.

This shows a complex threat landscape with a large variety of attackers and with organizations that are not sufficiently prepared. It is not realistic to believe that all incidents can be prevented. In addition, it is not economically feasible. Hence, it is evident that organizations need plans and procedures to handle incidents \textit{when} they occur. The existence of an incident response capability in an organization can assist them in rapidly detecting incidents, minimizing loss and destruction, mitigating the weaknesses that were exploited and restoring computing services \cite{nist800-61}. 

\section{Incident Management Overview}
%Eller noe i den duren, måtte bare ha en section til subsectionen :p
\subsection{Definitions}
\label{sec:Definitions}
%Eller et annet navn, poenget er å includere definisjoner av disse tidlig i rapporten
In information security incident management there are a few terms that need to be defined clearly. Two such terms are information or computer security incidents\footnote{In this report the terms ``information security incident", ``computer security incident" and ``incident" are used interchangeably.} %NB! På slutten, sjekk at dette faktisk stemmer!
and information or computer security events. It is important to recognize these as two terms of different meaning. The standard \acs{ISO}/\acs{IEC} 27000 \cite{ISO/IEC27000} and \acs{ISO}/\acs{IEC} 27035\cite{ISO/IEC27035} specifies the following definitions:

\textbf{Information security event:} Identified occurrence of a system, service or network state indicating a possible breach of information security policy or failure of safeguards, or a previously unknown situation that may be security relevant.

\textbf{Information security incident:} Single or a series of unwanted or unexpected \emph{information security events} that have a significant probability of compromising business operations and threatening information security.

\textbf{\ac{ISIRT}:} Team of appropriately skilled and trusted members of the organization that handles information security incidents during their lifecycle.

The guidelines \acs{NIST} SP 800-61 \cite{nist800-61} specifies the following definitions:

\textbf{Event:} An event is an observable occurrence in a system or network.

\textbf{Adverse event:} Adverse events are events with a negative consequence, such as system crashes, packet floods, unauthorized use of system privileges, unauthorized access to sensitive data, and execution of malware that destroys data.

\textbf{Computer security incident:} A violation or imminent threat of violation\footnote{An ``imminent threat of violation" refers to a situation in which the organization has a factual basis for believing that a specific incident is about to occur.} of computer security policies, acceptable use policies, or standard security practices.

\acs{NorCERT} specifies the following definitions \cite{NorCERT3Kvartal2012}:

\textbf{\ac{CSIRT}:} A central tool with the task of protecting important infrastructure. The team must consist of security specialists and they must handle and responds to incidents. Additionally they need to create awareness and educators.

\textbf{\ac{CERT}:} A trademark that can only be used after approval by Carnegie Mellon University. Is in practice the same as a \acs{CSIRT}.

The definition of an adverse event from \cite{nist800-61} is quite similar to the definition of information security event from \cite{ISO/IEC27000}. The definitions of incidents are also quite similar. These definitions are the ones that will be used in this report. \ac{ISIRT}, \ac{CSIRT} and \ac{CERT} define similar types of teams. In this report the term \acs{IRT} is used to denote such a team. 

\subsection{What is Incident Management}
It is important to note that incident response requires a substantial amount of planning and resources. Some of the most important parts of incident response are the existence of guidelines related to communication and related to prioritizing incidents and the use of a lessons learned process to gain value from incidents. \cite{nist800-61}

An organization should have an incident response policy, an incident response plan and incident response procedures, all of which should be tailored to the specific organization's needs. The policy usually includes elements such as prioritization or severity ratings of incidents, a definition of computer security incidents and reporting and contact forms. The plan usually includes an organizational approach to incident response, how the incident response team will communicate with the rest of the organization and with other organizations and metrics for measuring the incident response capability. Procedures should be based on the incident response policy and plan. \acp{SOP} are a delineation of the specific technical processes, techniques, checklists and forms used by the incident response team. An organization should establish procedures regarding communication with various outside parties, like media, law enforcement, other incident response teams, software vendors and \acp{ISP}. It is common to have \acp{PoC} for the various outside parties. \cite{nist800-61}

Additionally it is important to have a planned approach to reporting of vulnerabilities that have not yet been exploited \cite{ISO/IEC27035}.

\subsection{\acl{IRT}}
Organizations should have an incident response team available to handle incidents. The number of people responding depends on the magnitude of the incident. Large organizations can choose to have several incident response teams, e.g. one team per division. The teams can consist of employees or be partially or fully outsourced. The team members can be full-time or part-time depending on funding, staffing and incident response need.  %More about teams? Perhaps not necessary?
An incident response team usually performs intrusion detection, advisory distribution and education and awareness in addition to incident response. \cite{nist800-61}

The structure of the team may vary depending on the organizations resources and needs. The team may be dedicated, virtual or a mix of the two. However, the standard recommends a permanent team. Responsibilities, processes, allocation of roles and an appropriate training program should be prepared. \cite{ISO/IEC27035}

\section{Standards and Guidelines}
\subsection{\acs{ISO}/\acs{IEC} 27035}
\subsection{\acs{NIST} Special Publication 800-61}
This subsection gives an introduction to the guidelines \acs{NIST} SP 800-61 and the content is, unless specified otherwise, derived from \cite{nist800-61}. This publication aims to  assist organizations in mitigating risks from computer security incidents by providing guidelines on how to respond to incidents effectively and efficiently. 

One of the first considerations for a \ac{CSIRC} should be to agree on a definition of the term incident. This guidelines' definitions of events and incidents are included in section \ref{sec:Definitions} of this report. 

\acs{NIST} SP 800-61 describes the four phases of incident response; preparation, detection and analysis, containment, eradication and recovery and post-incident activity. The phases and the relationship between them are illustrated in figure \ref{fig:NISTIncidentResponse}.

\begin{figure}[ht]
%\hspace*{-0.4cm}
\begin{center}
\includegraphics[scale=0.27]{NISTIncidentResponseCycle.png}
\caption[The Incident Response Life Cycle]{The Incident Response Life Cycle \cite{nist800-61}}
\label{fig:NISTIncidentResponse}
\end{center}
\end{figure}

\paragraph{Preparation} 
This phase includes establishing an incident response capability as well as preventing incidents. The latter is not typically a part of the \ac{IRT}'s tasks, but it is fundamental to the success of the organization's incident response. If a large number of incidents occur, it may overwhelm the \ac{IRT}. To prepare for incidents the incident handlers should have tools and resources such as contact information, incident reporting mechanisms, issue tracking system, digital forensic workstations\footnote{A digital forensic workstation is specially designed for acquiring and analysing data. It usually contains a set of removable hard drives that can be used for evidence storage.} and digital forensic software. It is common to create a portable \emph{jump kit} containing materials that may be needed during incident response.

\paragraph{Detection and Analysis}
Organizations should prepare to handle any type of incident in addition to common incident types. A classification of incidents can be used as a basis for incident handling. The guideline provides a list of example categories for incidents that contains web, email, improper usage and loss or theft of equipment. It focuses on all kinds of incidents and does not address specific incident categories. A challenge related to incident handling is to detect the incident and determine the potential impact the incident may have. The actual detection may be the hardest part of incident handling. The guideline defines two types of signs of incidents; precursors and indicators, with indicators being the most common. These are defined in the following way: "A \emph{precursor} is a sign that an incident may occur in the future. An \emph{indicator} is a sign that an incident may have occurred or may be occurring now." Common sources for precursors and indicators are \acp{IDPS}, antivirus and antispam software, third-party monitoring services, logs, information on new vulnerabilities and exploits and people. 

A challenging part of this phase is the analysis, i.e. to determine which indicators and precursors are legitimate, if they are really related to an incident and what has actually happened. When the team believes an incident to have occurred they should try to determine the scope. All steps taken should be documented and timestamped. It is important to note that any such documentation can be used in court. The incident response team should maintain a database containing information about incidents, such as status, indicators, related incidents and actions taken by the incident handlers. It is important to prioritize incidents and to handle them accordingly. Factors that can be used as a basis for prioritization include the functional impact of the incident, the information impact of the incident and recovery from the incident. When the prioritization is performed, the \ac{IRT} should notify the appropriate people. It is important to have procedures regarding who these people should be.

\paragraph{Containment, Eradication and Recovery}
Containment is obviously an important part of incident handling. The existence of strategies and procedures for containment is helpful. These strategies and procedures are different for different types of incidents. Gathering and handling of evidence are part of this phase. For some incidents eradication is necessary and it is sometimes done during recovery. Eradication can include deleting malware and disabling breached user accounts. Recovery consists of restoring systems to normal operations and in some cases eliminating vulnerabilities that could cause similar incidents. The guideline does not offer specific recommendations for eradication and recovery as these are often OS specific. 

\paragraph{Post-Incident Activity}
Learning and improving is one of the most important parts of incident response. It is recommended to hold a ``lessons learned" meeting after each major incident and periodically after minor incidents. One meeting could potentially cover several incidents. ``Lessons learned" meetings should generally focus on revealing what was done well and what could be improved. The desired result is that the organization will be better equipped for the next incident. Often, incident response policies and procedures are updated. Areas these meetings should focus on are how well the staff performed and what they could have done differently, if documented procedures were followed and if they were adequate and how information sharing with other organizations could have been improved. To prevent similar incidents in the future, potential corrective actions and potential additional tools and resources should be reviewed. Both people involved in the incident(s) in question and people needed for future cooperation should be included in these meetings. A follow-up report that provides a reference that can be used when handling similar future incidents should be created. Other post-incident activities include the use of collected data for risk assessment, measurement processes to determine the success of the incident response team and audits of incident response programs. 





\subsection{NorSIS' Guideline for Incident Management}
The \ac{NorSIS} has in cooperation with a group of students\footnote{The students did a survey on incident management in Norwegian \acsp{SME}\cite{sand2010hendelseshaandtering}} developed a guideline for incident management, published in 2010\cite{norsisveiledning}. The aim of this guideline is to give a thorough description of why and how organizations should plan for security incident management, do business impact analysis and explain various measures to improve information security in organizations. The guideline distinguish between minimum requirements and general recommendations. The content in this section is, unless specified otherwise, derived from\cite{norsisveiledning}.

\paragraph{Incident Management Policy} 
An incident management policy should form the basis for developing new incident management plans in organizations. A solid policy should state an organization's objectives for incident management and include a statement ensuring commitment from senior management. Any relevant laws, standards and regulations should also be included. It is essential that the policy has requirements for regularly risk assessment, business impact analysis, tests and training. The guideline also suggests assigning roles and responsibilities as part of the incident management policy.

\paragraph{Business Impact Analysis}
NorSIS suggests that organizations conduct a business impact analysis to identify which services are of significant value and needs to be secured. Conducting risk assessment and identifying possible consequences of security incidents are part of this process. The guideline emphasize the importance of knowing risks and potential threats.  

\paragraph{Preventive Measures}
One of the most cost effective ways to do incident management is implementing preventive measures. Listed as minimum requirements are anti-virus, logs, firewalls, backups, alarms, locks, regularly reviews of threat landscape, and report systems for employees. Other proposed measures include encryption of data and wireless networks.   

\paragraph{Recovery Strategies}
The guideline recommends having a recovery strategy to quickly re-establish business operations after an incident. Suggestions include backup and emergency solutions. Routines and plans should be in place to handle recovery efficiently.

\paragraph{Incident Management Plan}
Organizations should use previous assessments and proposed incident scenarios to develop an incident management plan. It is recommended that individual plans addressing different scenarios are developed. Each incident management plan should state type of incident, what triggered the incident, roles and responsibilities, guidelines for communication and notification, maximum response time, check-list of tasks during incident response and post-incident activities.

\paragraph{Training}
To reduce costs caused by security incidents, NorSIS suggests training employees in correct use of equipment and make sure routines for incident response are well known.

\paragraph{Plan Maintenance}
The guideline recommends that organizations conduct yearly reviews of their incident management plans. To ensure a solid and up-to-date incident management plan, changes should be made based on experience from previous incidents. 

\paragraph{Outsourcing}
When organizations decide to outsource services, they should evaluate and agree on incident management procedures. It is the organization outsourcing that is responsible for securing information properly and to make sure sufficient plans for incident management exist. An agreement should define responsibilities and state expected quality of services. 

\paragraph{Example Threat Scenarios}
There should be no hesitations among employees when responding to typical threat scenarios. Examples of some security incident are given and aims to help organizations decide whether they have a sufficient incident management. Some examples are:
\begin{itemize}
\item Backup is lost due to a fire
\item The organization's network is attacked by a virus
\item There has been a burglary and two servers are stolen 
\end{itemize}

\subsection{SANS: Incident Handler's Handbook}
This section gives an introduction to SANS' Incident Handler's Handbook and the content is, unless specified otherwise, derived from \cite{SANShandbook}. The purpose of this document is to provide sufficient information for IT professionals and managers to create incident response policies, standards and teams for their organization. Six phases of incident management are described and recommended to be followed in sequence as each phase builds on the previous one. 

\paragraph{Preparation} 
This is the most crucial phase as it determines how well the incident response team will be able to respond to security incidents. During this phase, several key elements should be implemented to avoid potential problems while responding to security incidents.

Organizations should develop a policy stating the organization's principles, rules and practices. After the establishment of a security policy, organizations should develop a response plan with a prioritization of incidents based on organizational impact. Having this prioritization scheme could aid in obtaining necessary resources for incident management by ensuring commitment from senior management as they will better understand risk and business impact. It is also recommended to have a communication plan so the response process is not delayed by uncertainty of whom to contact in unexpected situations. These plans should also state when it is appropriate to contact law enforcement.

Documenting incidents is beneficial for organizations. A thorough documentation is useful for lessons learned and might also serve as evidence if an incident is considered a criminal act. The establishment of a \ac{CIRT} is part of the preparation phase. It is vital that also their activities are documented properly. 

\paragraph{Identification} 
The first step of this phase is identification of security events by detecting deviations from ``normal" operations within the organization. This is followed by a decision of whether the event is categorized as an incident. Organizations should implement various tools to gather documentation about events, such that incidents and patterns can be identified. Examples of such tools include \acp{IDS}, firewalls and log files. Typically, incidents are reported to the \ac{CIRT} that decides the scope of the incidents and how to move forward.

\paragraph{Containment} 
In this phase organizations try to limit the damage and prevent further damage caused by security incidents. It is recommended to isolate compromised systems to avoid escalation. An easy measure could be to disconnect affected parts of the system. 

Several steps are necessary for a successful incident response. The first step is called short-term containment and is concerned with limiting the damage by implementing short-term but effective measures. The second step is concerned with ensuring proper back-up of information before system resources can be restored. The final step is called long-term containment and involves removing alternations made by an attacker, installing security patches and limiting further escalation of the incident.

\paragraph{Eradication} 
Affected assets and systems are restored during this phase. To avoid similar incidents in the future, defences should be improved. Continuous documentation is important in this phase to ensure that proper steps were taken in previous phases in addition to determine the overall impact on the organization. It is recommended that all affected systems are scanned with anti-malware software to ensure that all potential latent malware is removed. 

\paragraph{Recovery} 
Activities in this phase include bringing affected systems back into operation and preventing future incidents caused by the same problem as previous incidents. Other activities are testing, monitoring and validating systems to ensure they are not reinfected. 

\paragraph{Lessons Learned} 
The final phase's main objectives are to learn from incidents to improve the CIRT's performance and to provide material to aid in future incident responses. An important activity is to hold a post-incident meeting that summarizes the incident management process. This phase evaluates an organization's incident management procedures and identifies areas of improvement.

\section{Related Work}
Due to the increased focus on incident management in recent years, a few studies of organizations' practice exist. Unlike our research, most of the previous surveys are based on quantitative questionnaires.

A group of students at HiG\footnote{H\o gskolen i Gj\o vik (Norwegian)} did a survey of incident management policies, implementations, training and routines in Norwegian \acp{SME}. Their results indicate overall insufficient plans for incident management, and poor quality in existing plans. The survey focus on \acp{SME}  in a quantitative questionnaire, whereas our research focus on qualitative analysis of large organizations. Additionally, compliance with standards was not a focus in their studies, as is in ours.

In a quantitative study from 2005\cite{brage}, a survey of Norwegian companies and public institutions were performed to look at routines for security incidents, how theory and practice differ and difference between public and private institutions. Public institutions were found to have greater shortcomings in reporting, training and statistics than private companies. 