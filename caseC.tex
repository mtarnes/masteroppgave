\subsection{Preparation}
This organization is a large organization which offer services to various sectors and thus has several individual departments. Since they are such a large organization, they operate with different incident management plans for different types of security incidents where various expertise is required to respond effectively.  

After years of experience and continuous reviewing this organization has developed three frameworks for incident management that each address one category of incidents. These frameworks describe relevant roles and activities for incident management. The organization has these three categories for incidents:
\begin{itemize}
\item IT-operation related incidents (service interruption etc.)
\item Information security incidents (breach of confidentiality, integrity and in some cases availability such as DDOS attacks)
\item All other incidents (terror, accidents etc.)
\end{itemize}
 
It is essential for the organization that customers trust them and their services. The IT security manager  describes loss of sensitive customer information, service unavailability and other security breaches that could lead to a weakening of the trust between them and their customers as the most serious consequences of incidents. 

There are many things the organization has done to prepare for security incidents. Various policies are followed by the different departments in the organization as their need for security varies. Additionally, contingency plans are implemented and revised continuously. 

The organization also has its own plan for dealing with the media during major incidents.

\paragraph{\acl{IRT}}
The organizations has its own \ac{IRT} which forms the core of their incident management. Employees work day and night with ``normal business operations" and thus need security incident expertise available 24/7 in case of security incidents. The \ac{IRT} work as a point of contact for the entire organization and they primarily assist in incident management and works as a pool of resources.

It is the incident managers that \emph{handle} incidents, whereas the \ac{IRT} assist with expertise on security incidents. Additionally, the \ac{IRT} has granted certain mandates that allows them to shut down systems or acquire external expertise and assistance up to a certain cost.    

Concerning customers, most have their own \acp{IRT}. During incidents, the \acp{IRT} communicate and coordinate the incident resopnse. The organization also offer their \ac{IRT} as a service for customers that do not have the capacity or need of their own team. 

\paragraph{Awareness and Training}
The IT security manager says they do extensive work to raise awareness around security among employees in their organization. New employees participate in courses where they are introduced to the organization's security handbook and they have to sign that the content is known and understood. Further, the security handbook is revised yearly during employee appraisals where employees have to reconfirm that the content is known.  

In addition to being introduced to the security handbook, employees are invited to lunches and presentations regularly. The intention is to increase the overall competence around security and ensure that security guidelines are read and not forgotten. 

Training?


\subsection{Detection and Analysis}
\paragraph{Initial Detection}
The initial detection of an incident can either be automatically generated from a server, it could be an alarm or something discovered manually.

\paragraph{Categorization}


\subsection{Incident Response}
\paragraph{Workflow}
\paragraph{Escalation}
\paragraph{Electronic Evidence}


\subsection{Lessons Learned}