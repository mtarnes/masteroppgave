\subsection{Preparation}
This organization is a large organization that offer IT services to various sectors and has several internal departments. They operate with different incident management plans for different types of security incidents where various expertise is required to respond effectively.  

After years of experience and continuous reviewing, the organization has developed three frameworks for incident management that each address one category of incidents. These frameworks describe relevant roles and activities for incident management. The organization has these three categories for incidents:
\begin{itemize}
\item IT-operational related incidents (service interruption etc.)
\item Information security incidents (breach of confidentiality, integrity and in some cases availability (such as DDOS attacks))
\item All other incidents (terror, accidents etc.)
\end{itemize}
 
It is essential for them that customers trust them and their services. The IT security manager describes loss of sensitive customer information, service unavailability and other security breaches that could lead to a weakening of the trust relationship between them and customers as a serious consequence of incidents. 

There are many things the organization has done to prepare for security incidents. Different policies are followed by the different departments in the organization as their need for security varies. Additionally, contingency plans are implemented and revised continuously. 

The organization also has its own plan for dealing with the media during major incidents. 

\paragraph{\acl{IRT}}
The organization has its own \ac{IRT} which forms the core of their incident management. Employees work day and night with ``normal business operations" and thus need security incident expertise available 24/7 in case of incidents. The \ac{IRT} work as a point of contact for the entire organization and they primarily assist in incident management and work as a pool of resources for incident managers.

It is the incident managers that \emph{handle} incidents, whereas the \ac{IRT} assist with expertise on security incidents. Additionally, the \ac{IRT} has granted certain mandates that allows them to shut down systems or acquire external expertise and assistance up to a certain cost.    

Most customers have their own \acp{IRT}. During incidents, the \acp{IRT} communicate and coordinate throughout the incident response. The organization also offer their \ac{IRT} as a service for customers that do not have the capacity or need of their own team. 

\paragraph{Standards and Guidelines}
The organization are basing their entire service management procedures on the ITIL framework discussed in section \ref{section:ITIL} 

They follow the ISO/IEC 27001 standard, and has several certifications. Their incident management processes however are mainly built on the ITIL framework, whereas the standard are used more as a guidance for compliance. The IT security manager are not familiar with the ISO 27035 standard that address security incident management specifically. 

\paragraph{Awareness and Training}
The IT security manager says they do extensive work to raise awareness around security among employees in their organization. New employees participate in courses where they are introduced to the organization's security handbook and they have to sign that the content is known and understood. Further, the security handbook is revised yearly during employee appraisals where employees have to reconfirm that the content is known.  

In addition to being introduced to the security handbook, employees are invited to lunches and presentations regularly. The intention is to increase the overall competence around security and to ensure that security guidelines are read and not forgotten. 

The organization does not conduct any regular training for employees addressing incident management. The IT security manager emphasize that their plans and procedures are being used so often in practice that there is no need to arrange training specifically. 

In accordance with the ITIL framework they conduct rehearsals once a year on a scenario they believe is useful. There is also regularly rehearsals performed   together with customers, as they often wish to include their IT service providers. The IT security manager states:
\begin{quote}
\textit{``Rehearsals are always expedient."}
\end{quote}
Additionally, training with the government take place every other year. 

\subsection{Detection and Analysis}
\paragraph{Initial Detection}
The initial detection of an incident can either be automatically generated from a server, it could be an alarm or something discovered manually. Often network analysis have to be done manually to discover security breaches. Whenever employees experience something abnormal or unexpected they are advised to report it to the \ac{IRT}. Typically this would be receiving emails from unknown senders or email attachment that does not work as intended; -failing to open or displays a few seconds of something indicating that a script is being run.

The IT security manager suspects underreporting of incidents among employees. He believes the threshold for reporting is high and that employees often omit to report suspicious events. This could either be due to employees failing to see the importance of reporting incidents or that they do not want to acknowledge potential mistakes they made. The IT security manager says they would rather have too many events being reported than too few, and they therefore work continually to emphasize the importance of reporting through raising awareness.

Vulnerabilities are reported through a risk framework, where they are evaluated, categorized and potentially escalated if they are assumed to be serious. They do not operate with anonymization for employees that report security events, but codes of conduct says one could use an external law firm if they wish to report incidents anonymously. However, this opportunity has not been used by employees so far. 

\paragraph{Categorization}
The organization does not follow any standards for categorization of incidents. They classify incidents as level one, two or three; - low, medium or high severity respectively. Whenever incidents are handled they are linked to these levels and handled according to predefined procedures. An incident's severity level will determine what can be done to respond. For incidents categorized as level three, authorization is needed from customers in case systems need to be shut down or changes have to be made in the production environment. This represent complex incident responses and are referred to as emergency changes, but are in some cases necessary to mitigate serious consequences. The organization are basing this approach on ITIL which stress the importance of incident categorization. ``Accurate categorisation is important because it will allow useful metrics to be gathered highlighting areas of the infrastructure where incidents are occurring\cite{itilbok}." Organizations can use this information to improve exposed infrastructure and thus decrease the number of incidents.

\subsection{Incident Response}
\paragraph{Workflow}
The workflow for incident management is based on the processes described in ITIL.

Even though the IT security manager describes an incident as whenever there is loss of information, other incidents such as DDOS attacks are also seen as an incident even though this breaches the availability of information rather than the confidentiality or integrity. In most cases he thinks of security incidents whenever there is an attacker trying to steal information. Nevertheless, he highlights:
\begin{quote}
\textit{``90-95\% of the security incidents we experience are availability related incidents"}
\end{quote}

The IT security manager believes their plans work well, since they are used often:
\begin{quote}
\textit{``Our plans work so well since we use them often in practice"}
\end{quote}

The IT manager explains why he thinks their routines work so well:
\begin{quote}
\textit{``Since we are such a large organization we experience a large volume of security incidents and thus have a framework that works well. I believe it is much tougher for smaller organizations; - if it is years since the last incident it becomes more challenging to respond effectively."}
\end{quote}

It security manager explains why keeping services up and running at all times are so important to them:
\begin{quote}
\textit{``Customers evaluate us as an IT service provider mainly on the availability of the services we deliver."}
\end{quote}

\paragraph{Escalation}
The organization has no predefined routines for escalation during incident response, even though they have done it several times in practice.

\paragraph{Electronic Evidence}
The organization has no in-house expertise on performing forensic analysis of electronic evidence. Mirroring disks for use as digital evidence approved for Norwegian courts is the only thing done by the organization itself with regards to evidence. The analysis of the disks are done by an external third party or by the police itself.

\subsection{Lessons Learned}
The organization are structuring their learning process in accordance with the problem management process from ITIL. During this process improvements are identified. 

A set of improvements are identified and summarized after rehearsals. Often, interaction with external parties are identified as areas of improvement, especially in accordance with customers. Collaboration and coordination across organizations and teams are proven to be challenging since different parts of an incidents are handled by IT service providers and customers themselves. 

A centralized tool is used to keep track of previous incidents in addition to experiences made, potential improvements and internal audits. Meetings are held after major incidents where the incident response process is reviewed. Post-incident meetings are held by the ``problem manager" which is responsible for the ITIL problem management process. Participants vary with the nature of the incident and the targeted environment. The IT security manager recognize the benefits of post-incident meetings and says:
\begin{quote}
\textit{``Often concrete measures are identified after incidents. Both organizational, process-related and on the investment side."}
\end{quote}
As an example he mentions major DDOS attacks leading to customers investing in new equipment and new routines for communication being implemented. 

Despite internal learning and review processes, experience and lessons learned are not often shared with external parties. How the organization perform incident management is not something they necessarily want to be public information as they wish to some extent to stay ``under the radar". Also, most often incidents occur with their customers and it is thus up to them whether information is shared. 

The organization are currently working on a project for improving their incident management and it has proven to be very effective. It mainly involves improving quality in their value chains. The IT security manager describes why this is important to incident management:
\begin{quote}
\textit{``One of the most important things with incident management is keeping track of and understanding value chains; -which, when and how components are communicating."}
\end{quote}
Large organizations often have complex and long supply chains. Through the project they identified weak quality in their value chain descriptions as a problem for effective incident response. The diagnostic work was complex, and sometimes it was challenging to identify what happened, where it happened and with what consequences. Extensive work was started to identify single-point-of-failure and making the value chains more robust. The project started in one of the organization's departments but quickly escalated to include larger parts of the organization. The project's objective is to identify vulnerabilities and areas of improvement, whereas it is the various departments' responsibility to implement recommended changes. So far the project has lead to improvements and new routines for interacting with third parties and shows an overall positive trend for minor incidents. 