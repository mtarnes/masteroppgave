\chapter*{Sammendrag}
En \o kende bruk av digitale l\o sninger tyder p{\aa} at virksomheter i dag er mer utsatt for angrep enn f\o r. Rapporter viser at angrep blir stadig mer avanserte og at angripere velger sine m{\aa}l med omhu. Hendelser forekommer til tross for implementering av forebyggende tiltak. Dette setter krav til en effektiv hendelsesh{\aa}ndtering. Det finnes flere standarder og retningslinjer som omhandler hendelsesh{\aa}ndtering, men det har blitt gjennomf\o rt f{\aa} praktiske studier av virksomheters hendelsesh{\aa}ndtering. I denne masteroppgaven ble en empirisk studie utf\o rt for {\aa} kartlegge virksomheters hendelsesh{\aa}ndtering. Studien ble gjennomf\o rt som et case studie av tre store norske virksomheter hvor datainnsamlingsmetodene var intervjuer og dokumentstudier. V{\aa}re funn viser at virksomhetene var relativt kompatible med standarder og retningslinjer for hendelsesh{\aa}ndtering, men at det fremdeles var rom for forbedringer. Vi fant at kommunikasjon, distribusjon av informasjon, involvering av ansatte, erfaring og fordeling av ansvar var viktige faktorer for en effektiv hendelsesh{\aa}ndtering. Vi bidrar med anbefalinger for {\aa} utf\o re en vellykket hendelsesh{\aa}ndteringsprosess. Noen av v{\aa}re anbefalte tiltak er {\aa} bruke standarder og retningslinjer som et grunnlag for hendelsesh{\aa}ndteringsprosessen, utf\o re \o velser, benytte ansatte som en del av sensornettverket for {\aa} detektere hendelser og utf\o re holdningsskapende kampanjer for de ansatte.