\subsection{\acs{NIST} Special Publication 800-61}
This subsection gives an introduction to the guidelines \acs{NIST} SP 800-61 and the content is, unless specified otherwise, derived from \cite{nist800-61}. This publication aims to  assist organizations in mitigating risks from computer security incidents by providing guidelines on how to respond to incidents effectively and efficiently. 

One of the first considerations for a \ac{CSIRC} should be to agree on a definition of the term incident. This guidelines' definitions of events and incidents are included in section \ref{sec:Definitions} of this report. 

\acs{NIST} SP 800-61 describes the four phases of incident response; preparation, detection and analysis, containment, eradication and recovery and post-incident activity. The phases and the relationship between them are illustrated in figure \ref{fig:NISTIncidentResponse}.

\begin{figure}[ht]
%\hspace*{-0.4cm}
\begin{center}
\includegraphics[scale=0.27]{NISTIncidentResponseCycle.png}
\caption[The Incident Response Life Cycle]{The Incident Response Life Cycle \cite{nist800-61}}
\label{fig:NISTIncidentResponse}
\end{center}
\end{figure}

\paragraph{Preparation} 
This phase includes establishing an incident response capability as well as preventing incidents. The latter is not typically a part of the \ac{IRT}'s tasks, but it is fundamental to the success of the organization's incident response. If a large number of incidents occur, it may overwhelm the \ac{IRT}. To prepare for incidents the incident handlers should have tools and resources such as contact information, incident reporting mechanisms, issue tracking system, digital forensic workstations\footnote{A digital forensic workstation is specially designed for acquiring and analysing data. It usually contains a set of removable hard drives that can be used for evidence storage.} and digital forensic software. It is common to create a portable \emph{jump kit} containing materials that may be needed during incident response.

\paragraph{Detection and Analysis}
Organizations should prepare to handle any type of incident in addition to common incident types. A classification of incidents can be used as a basis for incident handling. The guideline provides a list of example categories for incidents that contains web, email, improper usage and loss or theft of equipment. It focuses on all kinds of incidents and does not address specific incident categories. A challenge related to incident handling is to detect the incident and determine the potential impact the incident may have. The actual detection may be the hardest part of incident handling. The guideline defines two types of signs of incidents; precursors and indicators, with indicators being the most common. These are defined in the following way: "A \emph{precursor} is a sign that an incident may occur in the future. An \emph{indicator} is a sign that an incident may have occurred or may be occurring now." Common sources for precursors and indicators are \acp{IDPS}, antivirus and antispam software, third-party monitoring services, logs, information on new vulnerabilities and exploits and people. 

A challenging part of this phase is the analysis, i.e. to determine which indicators and precursors are legitimate, if they are really related to an incident and what has actually happened. When the team believes an incident to have occurred they should try to determine the scope. All steps taken should be documented and timestamped. It is important to note that any such documentation can be used in court. The incident response team should maintain a database containing information about incidents, such as status, indicators, related incidents and actions taken by the incident handlers. It is important to prioritize incidents and to handle them accordingly. Factors that can be used as a basis for prioritization include the functional impact of the incident, the information impact of the incident and recovery from the incident. When the prioritization is performed, the \ac{IRT} should notify the appropriate people. It is important to have procedures regarding who these people should be.

\paragraph{Containment, Eradication and Recovery}
Containment is obviously an important part of incident handling. The existence of strategies and procedures for containment is helpful. These strategies and procedures are different for different types of incidents. Gathering and handling of evidence are part of this phase. For some incidents eradication is necessary and it is sometimes done during recovery. Eradication can include deleting malware and disabling breached user accounts. Recovery consists of restoring systems to normal operations and in some cases eliminating vulnerabilities that could cause similar incidents. The guideline does not offer specific recommendations for eradication and recovery as these are often OS specific. 

\paragraph{Post-Incident Activity}
Learning and improving is one of the most important parts of incident response. It is recommended to hold a ``lessons learned" meeting after each major incident and periodically after minor incidents. One meeting could potentially cover several incidents. ``Lessons learned" meetings should generally focus on revealing what was done well and what could be improved. The desired result is that the organization will be better equipped for the next incident. Often, incident response policies and procedures are updated. Areas these meetings should focus on are how well the staff performed and what they could have done differently, if documented procedures were followed and if they were adequate and how information sharing with other organizations could have been improved. To prevent similar incidents in the future, potential corrective actions and potential additional tools and resources should be reviewed. Both people involved in the incident(s) in question and people needed for future cooperation should be included in these meetings. A follow-up report that provides a reference that can be used when handling similar future incidents should be created. Other post-incident activities include the use of collected data for risk assessment, measurement processes to determine the success of the incident response team and audits of incident response programs. 



