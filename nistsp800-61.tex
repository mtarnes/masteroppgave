\subsection{\acs{NIST} Special Publication 800-61}
This subsection gives an introduction to the guidelines \acs{NIST} SP 800-61 and the content is, unless specified otherwise, derived from \cite{nist800-61}. This publication aims to  assist organizations in mitigating risks from computer security incidents by providing guidelines on how to respond to incidents effectively and efficiently. 

The security-related threat level is continuously changing and new types of security-related incidents emerge frequently. Preventive actions are not sufficient in order to be able to handle this and an incident response capability is therefore necessary. Even though incident prevention is not sufficient, it is an important complement to incident response. The existence of an incident response capability in an organization can assist them in rapidly detecting incidents, minimizing loss and destruction, mitigating the weaknesses that were exploited and restoring computing services. It is important to note that incident response requires a substantial amount of planning and resources. Some of the most important parts of incident response are the existence of guidelines related to communication and related to prioritizing incidents and the use of a lessons learned process to gain value from incidents.

One of the first considerations for a \ac{CSIRC} should be to agree on a definition of the term incident.