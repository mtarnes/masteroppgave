\chapter*{Abstract}
Today, digital solutions are vital to many organizations and large amounts of sensitive data are stored digitally. Despite organizations' implementation of security measures, information security incidents occur. Thus, preventive measures is not sufficient and the establishment of an incident management capability is essential. Although many provide literature addressing information security incident management, not many studies examining current practice exist. In this thesis an empirical study was conducted where organizations' incident management practices were studied. The study included qualitative interviews and document studies of three large Norwegian organizations. The findings show that they to a large extent comply with standards and guidelines for incident management, but that there is still room for improvements. Main challenging factors were found to be communication, information dissemination, employee involvement, allocation of responsibilities and experience. We recommend organizations to use standards and guidelines as a basis for incident management, conduct regular rehearsals, utilize employees as part of the sensor network in incident detection and to conduct awareness campaigns for employees.