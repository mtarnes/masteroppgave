\chapter{Case Introductions}
\label{chp:CaseIntroductions}
This chapter gives an introduction to the specific cases studied in this thesis. Three  different organizations are studied in separate cases. All three organizations in this study are large. In Norway, organizations are categorized as large if they have more than 100 employees \cite{SMB}. 

\section{Case A}
The organization studied in case A is a large Norwegian government-owned organization with several thousands employees and a large number of user accounts. Throughout the rest of this report, the organization in this case will be referred to as Organization A. The interviewee was the IT security manager. He has had this role for about two years and his responsibilities include both technical and administrative tasks. 

The organization handles most of their IT operations themselves, even though some services are outsourced. They have an IT manager leading a staff that includes the IT security manager. They have a customer service center that handles user support and receives notifications from users that observe unusual activity. The organization has a network section that is responsible for ensuring that their network infrastructure works as intended. Additionally, they have a section that operates their servers. They have large systems, one for e-mail in addition to a system for handling employee and user data. The organization has a section responsible for developing, maintaining and operating their applications.

The documents studied in this case were their information security policy, principles for information security document, IT regulations and contingency plan.

A total of 15 employees participated in the employee survey in case A. They were randomly selected from four departments at different geographical locations. 

\section{Case B}
The organization in case B is a large Norwegian, independent and non-commercial organization with a couple of thousands employees. They process large amounts of valuable and sensitive information and thus information security has high priority in their business operations. Throughout the rest of this report, the organization in this case will be referred to as Organization B. The two interviewees from the organization were the IT security manager and the supply chain manager, both working in the IT department. They have had these roles for the past four and six years respectively. They deliver IT support for all of the organization's departments and are involved in incident management regularly.   

In addition to the central IT department, %that is responsible for shared IT services in the organization, IT is further organized hierarchically. 
each individual department has their own IT manager responsible for local IT support in the department. The local IT managers are also responsible for information security within their departments.

Organization B has to a large extent outsourced their IT operations and has several suppliers. The main IT operations are delivered by one organization, whereas application management is delivered by a consortium. The former is referred to as Supplier 1 and the latter as Supplier 2 in this report. These two are the main suppliers of IT operations, but there are others as well. As part of case B, representatives from the two main suppliers were interviewed to get a holistic picture of Organization B's incident management. Because of the large extent of outsourcing distributed over a number of parties, the suppliers need to coordinate and cooperate. 

Supplier 1's entire team in basic IT operations is available for Organization B. They have customer service available for the organization as well. These two teams consist of about 18 and 25 people respectively. This does not mean that all of them have access to the organization's system. About 20-30 people have access. The interviewee from Supplier 1 is a technician in basic IT operations as well as being a security coordinator between Supplier 1 and the organization. He has had this role for 3-4 years.

Supplier 2 has a team of 15-20 people available for organization B. They are responsible for application management which involves improving applications and making sure that applications are without errors. Additionally, they advise the organization about how they can improve their applications and work processes. The interviewee from Supplier 2 is the service manager for one of Organization B's systems. He has had this role for about one and a half years. Supplier 2 cooperates with Supplier 1.

The document studied in this case was Organization B's IT contingency plan.

15 people from three departments participated in the employee survey in case B. They were randomly chosen and the selection includes both employees with administrative tasks and employees with tasks related to the organization's core activities.

\section{Case C}
This organization is a large Norwegian organization with several thousands employees. They deliver IT services to customers in addition to operating their own infrastructure. Throughout the rest of this report, the organization in this case will be referred to as Organization C. The interviewee was the IT security manager and the operational leader of a department that is responsible for security, quality, compliance and risk. He has had this role for about two years. In addition, we had e-mail correspondence with one employee with several years of experience from the organization's IRT.

In Organization C, departments have different requirements for security and thus various policies are implemented throughout the organization. In addition to operating their own incident management, they are responsible for incidents concerning services they deliver to customers. Consequently, the organization deals with security incidents frequently. 

The documents studied in this case were their \ac{IRT} handbook, corporate information security policy, enterprise risk management process description, incident management process description, major incident routine description and contingency policy.

11 people from the organization participated in the employee survey. They were randomly chosen and had different roles, tasks and responsibilities within the organization.