\chapter{Case Introductions}
\label{chp:CaseIntroductions}
This chapter gives an introduction to the specific cases studied in this thesis. The study contains three separate cases where a different organization is studied in each case. All of the organizations are classified as large organizations. In Norway an organization is large if it has more than 100 employees \cite{SMB}.

\section{Case A}
This organization is a large Norwegian government-owned organization. They have a larger number of users of their systems. The interviewee is the IT security manager. His responsibilities include both technical and more administrative tasks. He has had this role for about two years.

The organization handles most of its own IT operations. Some services are outsourced, but they mainly handle the IT operations themselves. They have an IT-manager with a staff that includes the IT security manager. They have a customer service center %eller service deks eller noe? 
that handles user support and receives cases from users when something happens. The organization also has a network section that is responsible for ensuring that their network infrastructure works as intended. Additionally they have a section that operates their servers. They have large systems, like a system for e-mail and a system for handling employee and user data. They also have a section that handles their applications. This includes both developing, maintaining and operating their applications.

\section{Case B}
The Norwegian organization in this case is a large, independent and non-commercial organization with a couple of thousand employees. They hold large amounts of valuable and sensitive information and thus information security has high priority in their business operations. The two interviewees in this case are the IT- and IT security manager and the supply chain manager, both working in the IT department. They have had these roles for the past four and six years respectively. They deliver IT support for all the organization's departments and are involved in incident management regularly.   

In addition to the IT department, that is responsible for all shared IT services in the organization, IT is organized further hierarchically such that each individual department has their own IT manager responsible for local IT-support concerning their department. The IT managers are also responsible for information security within their departments.

The organization has to a large extent outsourced their IT operations and has many suppliers. The main IT operations are delivered by one external organization, whereas application management is delivered by an external consortium. The former is referred to as supplier 1 and the latter as supplier 2 in this report. The main organization in question is referred to as organization B or the organization. These two are the main suppliers of IT operations, however there are others as well. Because of the large extent of outsourcing distributed over a number of parties, the suppliers need to coordinate and cooperate with each other.

Supplier 1's entire team in basis IT operations is available for this organization. They also have customer service available for the organization. These two teams consist of about 18 and 25 people respectively. This does not mean that all of them have access to the organization's system. 20-30 people have this access. The interviewee is a technician in basis IT operations as well as being a security coordinator between supplier 1 and the organization. He has had this role in 3-4 years.

Supplier 2 has a team of 15-20 people available for organization B. They are responsible for application management which involves making sure that applications are without errors and improving applications. Additionally, they advise the organization about how they can improve applications and work processes. The interviewee is the service manager for one of organization B's systems. He has had this role for about one and a half years. Supplier 2 cooperates with supplier 1.

\section{Case C}
This organization is a large Norwegian organization with several thousand employees. They deliver IT-services to customers in addition to operating their own infrastructure. The interviewee is the IT security manager  and the operational leader of a department that is responsible for security, quality, compliance and risk. He has had this role for about two years.

In such a large organization, departments have different requirements for security and thus various policies are implemented throughout the organization. In addition to operating their own incident management, they are also responsible incidents concerning services they deliver to customers. Consequently, the organization deals with security incidents on a daily basis. In addition to dedicated incident managers, the organization has its own \ac{IRT} to assist in major incidents. The \ac{IRT} also handles incoming notifications from internal users regarding security issues and may also deal with incidents concerning customers. 