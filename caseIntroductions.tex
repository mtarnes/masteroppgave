\chapter{Case Introductions}
\label{chp:CaseIntroductions}
This chapter gives an introduction to the specific cases studied in this thesis. The study contains three separate cases where a different organization is studied in each case. All of the organizations are classified as large organizations. In Norway an organization is large if it has more than 100 employees \cite{SMB}.

\section{Case A}
This organization is a large Norwegian government-owned organization. They have 200 employees that work with IT, and a larger number of users of their systems. The interviewee is the IT security manager. His responsibilities include both technical and more administrative tasks. He has had this role for about two years.

The organization handles most of its own IT operations. Some services are outsourced, but they mainly handle the IT operations themselves. They have an IT-manager with a staff that includes the IT security manager. They have a customer service center %eller service deks eller noe? 
that handles user support and receives cases from users when something happens. The organization also has a network section that is responsible for ensuring that their network infrastructure works as intended. Additionally they have  a section that operates their servers. They have large systems, like a system for email and a system for handling employee and user data. They also have a section that handles their applications. This includes both developing, maintaining and operating their applications.

\section{Case B}
The Norwegian organization in this case is a large, independent and non-commercial organization with a couple of thousand employees. They possess large amounts of valuable and sensitive information and information security is thus a high priority in their business operations. The two interviewees in this case work as IT manager and supply manager in the IT department, roles they have had four and six years respectively. They deliver IT-support for all departments that constitute the organization and are involved in incident management regularly.   

In addition to the IT department, that is responsible for all common IT services for all employees in the organization, IT is organized further such that each individual department has their own IT manager responsible for local IT support within their department. The IT managers are also responsible for the information security in their departments.

The organization has to a large extent outsourced their IT-operations and has    many suppliers of IT-services. Basic IT operations are delivered by one external organization, whereas application operations are delivered by an external consortium. These two main suppliers of IT-operations are referred to as supplier 1 and 2 in this report. These two are the main suppliers of IT-operations, however there are others as well. 

%skrive noe om contractor 1 og 2 her kanskje?

\section{Case C}
This organization is a large Norwegian organization with several thousand employees. They deliver IT-services to customers in addition to operating their own infrastructure. The interviewee is the IT security manager  and the operational leader of a department that is responsible for security, quality, compliance and risk. He has had this role for about two years.

In such a large organization, departments have different requirements for security and thus various policies are implemented throughout the organization. In addition to operating their own incident management, they are also responsible incidents concerning services they deliver to customers. Consequently, the organization deals with security incidents on a daily basis. In addition to dedicated incident managers, the organization has its own \ac{IRT} to assist in major incidents. The \ac{IRT} also handles incoming notifications from internal users regarding security issues and may also deal with incidents concerning customers. 