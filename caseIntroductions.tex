\chapter{Case Introductions}
\label{chp:CaseIntroductions}
This chapter gives an introduction to the specific cases studied in this thesis. The study contains three separate cases where a different organization is studied in each case. All of the organizations are classified as large organizations. In Norway an organization is large if it has more than 100 employees \cite{SMB}.

\section{Case A}
This organization is a large Norwegian government-owned organization. They have 200 employees that work with IT, and a larger number of users of their systems. The interviewee is the IT security manager. His responsibilities include both technical and more administrative tasks. He has had this role for about two years.

The organization handles most of its own IT operations. Some services are outsourced, but they mainly handle the IT operations themselves. They have an IT-manager with a staff that includes the IT security manager. They have a customer service center %eller service deks eller noe? 
that handles user support and receives cases from users when something happens. The organization also has a network section that is responsible for ensuring that their network infrastructure works as intended. Additionally they have  a section that operates their servers. They have large systems, like a system for email and a system for handling employee and user data. They also have a section that handles their applications. This includes both developing, maintaining and operating their applications.

\section{Case B}

\section{Case C}
This organization is a large Norwegian organization with several thousand employees. They deliver IT-services to customers in addition to operating their own infrastructure. The interviewee is the IT security manager  and the operational leader of a department that is responsible for security, quality, compliance and risk. He has had this role for about two years.

In such a large organization various departments have different requirements for security, and thus various policies are implemented throughout the organization. In addition to having their own incident management, they are responsible for dealing with incidents concerning the IT-services they deliver to customers. Consequently, the organization deals with security incidents on a daily basis. In addition to dedicated incident managers the organization has its own \ac{IRT} to handle major incidents. The \ac{IRT} also handles incoming notifications from internal users regarding security issues and may also deal with incidents concerning customer assets. 