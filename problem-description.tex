\chapter*{Problem Description}
Despite organizations’ implementation of incident response plans and security controls, it is inevitable that severe security incidents occur from time to time. In recent years, an increasing number of ICT security incidents have been reported. Both general and single-purpose attacks caused by malware, in addition to minor errors causing severe consequences, are typical incidents. Several well-established standards and guidelines addressing incident management exist. However, how security incidents are handled differ greatly among organizations. Many factors such as routines, reporting procedures, preparation plans, distribution of responsibilities and capabilities to respond in accordance to incident severity are involved in deciding how successfully organizations respond to ICT security incidents. 

The main research question of this thesis is as follows:
How well do large organizations perform information security incident management in relation to severe information security incidents and to what extent do factors such as routines, standards and information sharing contribute to a successful management scheme?

The main research question is further divided into sub-questions. A solid basis for discussing the main research question will be established by answering the following sub-questions:

\begin{itemize}\itemsep-0.15cm
\item What plans and procedures for information security incident management are established in large (Norwegian) organizations?
\item To what extent are existing standards/guidelines adopted in plans for information security incident management?
\item To what extent have previous incidents been handled in accordance with the current plan? Was it successful?
\item How are “lessons learned” systematized and used for improvements of the information security management scheme?
\item How are experiences distributed among and utilized by organizations?
\end{itemize}

In order to answer the research questions there will be gathered information about plans and procedures for incident management in various organizations as well as about actual incidents. Experiences from a variety of incidents will be systematized and a study of incident handling processes will be performed.

Students: Cathrine Hove and Marte T\aa rnes \\
Assignment given: 21. January, 2013 \\
Supervisor: Maria B. Line