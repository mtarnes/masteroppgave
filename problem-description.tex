\chapter*{Problem Description}
Despite organizations’ implementation of incident response plans and security controls, it is inevitable that severe security incidents occur from time to time. In recent years, an increasing number of ICT security incidents have been reported. Typical incidents include both general and single-purpose attacks caused by malware, in addition to minor errors with severe consequences. Several well-established standards and guidelines addressing incident management exist. However, the way security incidents are handled differ greatly among organizations. Several factors are involved in determining how successfully organizations respond to ICT security incidents. \\

The main research question of this thesis is as follows: 
\begin{itemize}
\item How do large organizations perform information security incident management as a measure against severe information security incidents?\\
\end{itemize}

The main research question is further divided into sub-questions. A solid basis for discussing the main research question will be established by answering the following sub-questions:

\begin{itemize}
\item What plans and procedures for information security incident management are established in large organizations?
\item To what extent are existing standards/guidelines adopted in plans for information security incident management?
\item How has previous information security incidents been handled in accordance with predetermined plans? 
\end{itemize}

In order to answer the research questions there will be gathered information about plans and procedures for incident management in various organizations as well as about actual incidents. Experiences from a variety of incidents will be systematized and a study of incident handling processes will be performed.

\begin{tabular}{@{}p{4cm}l}
\vspace{0.4cm} & \vspace{0.4cm} \\
Students:		& Cathrine Hove and Marte T\aa rnes \\
Assignment given: & 21. January, 2013 \\
Supervisor:		& Maria B. Line \\
Responsible professor: 	& Karin Bernsmed \\
\end{tabular}