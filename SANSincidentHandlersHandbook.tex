\subsection{SANS: Incident Handler's Handbook}
This section gives an introduction to the Incident Handler's Handbook and the content is, unless specified otherwise, derived from \cite{SANShandbook}. The purpose of this document is to provide sufficient information for IT professionals and managers to create incident response policies, standards and teams for their organization. Six phases of incident management are described and recommended to be followed in sequence as each phase builds on the previous one. A check-list of relevant tasks for each phase, useful commands and areas to look for anomalous behaviour within both the Windows and UNIX environments are also included in the handbook.

\paragraph{Preparation} 
This is the most crucial phase as it determines how well the incident response team will be able to respond to security incidents. During this phase, several key elements should be implemented to avoid potential problems while responding to security incidents.

Organizations should develop a policy stating the organization's principles, rules and practices. After establishing a security policy, organizations should organize a response plan with a prioritization of incidents based on organizational impact. Having this prioritization scheme could aid in getting necessary resources for incident management by ensuring commitment from senior management as they will better understand risk and business impact. It is also recommended having a communication plan so the response process is not delayed by uncertainty of whom to contact in unexpected situations. These plans should also state when it is appropriate to contact law enforcement.

Documenting incidents and steps taken during incident response is extremely beneficial for organizations. A thorough documentation is useful for lessons learned and might also serve as evidence if an incident is considered a criminal act. As part of the preparation phase is the establishment of a \ac{CIRT}, and it is vital that also their activities are documented properly. 

\paragraph{Identification} 
The first steps of this phase are identification of security events by detecting deviations from ``normal" operations within the organization, followed by a decision of whether the event is categorized as an incident. Organizations should implement various tools to gather documentation about events, such that incidents and patterns can be identified. Examples of such tools include \acp{IDS}, firewalls and log files. Typically, incidents are reported to the \ac{CIRT} that decides the scope of the incidents and how to move forward with the next phase.

\paragraph{Containment} 
In this phase organizations try to limit the damage and prevent further damage caused by security incidents. It is recommended that compromised systems are isolated to avoid escalation. An easy measure could be disconnecting affected parts of the system. 

This phase comprises several steps, all necessary for a successful incident response. The first step is called short-term containment and is concerned with limiting the damage by implementing short-term but effective solutions. The second step is ensuring proper back-up of information before system resources can be restored. The final step is long-term containment and involves removing alternations made by an attacker, installing security patches and limiting further escalation of the incident.

\paragraph{Eradication} 
Affected assets and systems are restored during this phase. To avoid similar incidents from happening again, defences should be improved during this phase. Continuous documentation is important in this phase to ensure that proper steps were taken in previous phases in addition to determine the overall impact on the organization. It is recommended that all affected systems are scanned with anti-malware software to ensure that all potential latent malware is removed. 

\paragraph{Recovery} 
Activities in this phase include bringing affected systems back into operation and preventing future incidents caused by the same problem as previous incidents. Other activities are testing, monitoring and validating systems to ensure they are not reinfected. 

\paragraph{Lessons Learned} 
The final phase's main objectives are to learn from incidents to improve the CIRT's performance and to provide materials to aid in response to future incidents. An important activity is holding a post-incident meeting summarizing the incident management process. This phase evaluates an organization's incident management procedures and identifies areas of improvement.