\subsection{SANS: Incident Handlers Handbook}
This section gives an introduction to the Incident Handlers Handbook and the content is, unless specified otherwise, derived from \cite{SANShandbook}. The purpose of this document is to provide sufficient information for IT-professionals and managers to create incident response policies, standards and teams for their organization. Six phases of incident response are described and recommended to be followed in sequence as each phase builds on the previous one. A check-list of relevant tasks for each phase is also included.

\textbf{Preparation} is the most crucial phase as it determines how well the incident response team will be able to respond to security incidents. The preparation phase involves implementing several key elements to avoid potential problems when responding to security incidents.

A policy should state the organizations principles, rules and practices. After   

Organizations should establish a response plan with a prioritization of incidents based on organizational impact. Having this prioritization could help get necessary resources for incident management from senior management. It is also recommended having a communication plan so the response process is not delayed by uncertainty of whom to contact in unexpected situations. These plans should also state when it is appropriate to contact law enforcement.

\textbf{Identification}

\textbf{Containment}

\textbf{Eradication}

\textbf{Recovery}

\textbf{Lessons Learned}