\subsection{\acs{ISO}/\acs{IEC} 27002}
\label{sec:iso27002}
This standard represents a code of practice for information security management and establishes guidelines for initiating, implementing, maintaining and improving information security management in an organization. The standard is intended to be a starting point for developing organization specific guidelines and contains 11 security control clauses that outline various security objectives and provide implementation guidance. It is emphasized that organizations should initially identify and establish their security requirements and then choose which of the security controls to implement.

This section presents clauses from the standard that are relevant to incident management. They are retrieved from \cite{ISO/IEC27002}.

\textbf{13.1 Reporting information security events and weaknesses } \\
The objective is to ensure that all significant information security events and weaknesses are reported such that corrective actions can be made in time. Reporting procedures and employee awareness are important success factors and it should be required to report any events or weaknesses to the point of contact as quickly as possible.

\textbf{13.1.1 Reporting information security events} \\
\emph{Control:} Information security events should be reported through appropriate management channels as quickly as possible.

\emph{Implementation guidance:} A point of contact and a formal event reporting procedure should be established and employees should be made aware of these. The reporting procedure should include the following.
\begin{enumerate}[a)]
\item suitable feedback processes to ensure that those reporting information security events are notified of results after the issue has been dealt with and closed.
\item information security event reporting forms to support the reporting action, and to help the person reporting to remember all necessary actions in case of an information security event.
\item the correct behaviour to be undertaken in case of an information security event.
\item reference to an established formal disciplinary process for dealing with employees, contractors or third party users who commit security breaches.
\end{enumerate}

\textbf{13.1.2 Reporting security weaknesses} \\
\emph{Control:} All employees, contractors and third party users of information systems and services should be required to note and report any observed or suspected security weaknesses in systems or services.

\emph{Implementation guidance:} There should exist an easy, accessible and available reporting mechanism for employees, contractors and third party users. Weaknesses should be reported as quickly as possible to either management or the service provider and not attempted to be proven.

\textbf{13.2 Management of information security incidents and improvements}\\
The objective is to ensure that the management of security incidents follows a consistent and effective approach where responsibilities and procedures are in place to handle incidents once they have been reported. Procedures should be in place for continual improvement of management processes. When necessary to collect evidence, this should be done in compliance with legal requirements.

\textbf{13.2.1 Responsibilities and procedures}\\ 
\emph{Control:} Management responsibilities and procedures should be established to ensure a quick, effective and orderly response to information security incidents.

\emph{Implementation guidance:} In addition to reporting, monitoring should be used to discover incidents. When implementing incident management procedures organizations should consider the following.
\begin{enumerate}[a)]
\item procedures should be established to handle different types of information security incidents, including:
\begin{enumerate}[1)]
\item information system failures and loss of service.
\item malicious code.
\item denial of service.
\item errors resulting from incomplete or inaccurate business data.
\item breaches of confidentiality and integrity.
\item misuse of information systems.
\end{enumerate}
\item in addition to normal contingency plans, the procedures should also cover:
\begin{enumerate}[1)]
\item analysis and identification of the cause of the incident.
\item containment.
\item planning and implementation of corrective action to prevent recurrence, if necessary.
\item communication with those affected by or involved with recovery from the incident.
\item reporting the action to the appropriate authority.
\end{enumerate}
\item audit trails and similar evidence should be collected and secured, as appropriate, for:
\begin{enumerate}[1)]
\item internal problem analysis.
\item use as forensic evidence in relation to potential breach of contract or regulatory requirement or in the event of civil or criminal proceedings, e.g. under computer misuse or data protection legislation.
\item negotiating for compensation from software and service suppliers.
\end{enumerate}
\item action to recover from security breaches and correct system failures should be carefully controlled. The procedures should ensure that:
\begin{enumerate}[1)]
\item only certain identified and authorized personnel are allowed access to live systems and data.
\item all emergency actions taken are documented in detail.
\item emergency action is reported to management and reviewed in an orderly manner.
\item the integrity of business systems and controls is confirmed with minimal delay.
\end{enumerate}
\end{enumerate}

\textbf{13.2.2 Learning from information security incidents}\\
\emph{Control:} There should be mechanisms in place to enable the types, volumes, and costs of information security incidents to be quantified and monitored.

\emph{Implementation guidance:} By monitoring incidents, reoccurring and high impact incidents can be identified and need for additional controls can be evaluated.

\textbf{13.2.3 Collection of evidence}\\
\emph{Control:} Where a follow-up action against a person or organization after an information security incident involves legal action (either civil or criminal), evidence should be collected, retained, and presented to conform to the rules for evidence laid down in the relevant jurisdiction(s).

\emph{Implementation guidance:} The rules of evidence involve admissibility and weight of evidence, that is whether or not evidence can be used in court and the quality and completeness of the evidence. To achieve admissibility and weight of evidence, organizations should ensure their systems comply with standards and that controls used to protect evidence are complete and consistent.
