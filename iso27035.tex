\subsection{\acs{ISO}/\acs{IEC} 27035}
\label{sec:iso27035}
This section gives an introduction to the ISO 27035 standard and the content is, unless specified otherwise, derived from \cite{ISO/IEC27035}. 

Implementing this standard will aid organizations in dealing with information security incidents properly and mitigate both direct and indirect adverse business impact. This standard provides an extensive and structured approach to incident management by presenting five phases with recommended activities for large and medium-sized organizations. 

One of the standard's objectives is to provide guidelines to aid organizations in meeting the requirements specified in ISO/IEC 27001. ISO/IEC 27035 is a supplement to the implementation guidelines relevant to incident management that are presented in ISO/IEC 27002.

\paragraph{Plan and Prepare} This phase is by far the most extensive phase and involves many activities. Individual organizations have to ensure that their use of resources are proportional to their needs. Each organization should formulate an incident management policy that reviews current vulnerabilities, states the need for an incident management scheme and identifies benefits for the organization. Security and risk management policies should be reviewed and updated regularly. The standard highlights the importance of ensuring commitment from senior management in the security incident management policy to ensure the organization's commitment to resources and maintenance of an incident management capability.  

One of the main activities in the plan and prepare phase is to make a detailed incident management scheme. The scheme should include reporting forms (preferably electronic) and a classification scale for grading incidents.   

Another important activity in this phase is the establishment of the \acf{ISIRT}. Organizations should establish and implement required mechanisms of support for their incident management scheme to operate efficiently during this phase. This includes technical tools such as \acp{IDS} and log monitoring systems as well as relationships and connections to other organizations. 

All personnel should be familiar with the incident management scheme, when it becomes operational and be able to recognize its benefits. Users' awareness and participation is essential for the success of a structured incident management approach. It is recommended that an appropriate awareness and training program is developed and repeated regularly as personnel change over time.

The entire incident management scheme should be tested to verify that the scheme and the \ac{ISIRT} work in complex and real situations. After going through this phase, organizations should be fully prepared to manage security events, incidents and vulnerabilities.

\paragraph{Detection and Reporting} The first operational phase of an incident management scheme involves detection, collection of information and reporting of occurrences of security events, incidents and vulnerabilities either discovered by humans or automated systems. It is important to preserve information about vulnerabilities and incidents in a database operated and maintained by the \ac{ISIRT}. Organizations should implement security monitoring systems, \acl{IDS}/\acl{IDP} (\acs{IDS}/\acs{IDP}) and antivirus programs to aid the detection of security events, incidents and vulnerabilities. Logs from various entities should be analysed and registrations of incidents should be made in an Incident Tracking System. 

It is the person first notified about an event that is responsible for starting the activities involved in this phase. There are several ways a security event or incident can be detected and thus all employees should be aware of and have access to the guidelines for reporting. There should be clear procedures to follow for people involved in handling an incident. All relevant information should be passed to the \ac{PoC} and the responsible \ac{ISIRT} member. It is recommended that one of the \ac{ISIRT} members is appointed the responsibility for incoming reports and for making assessments about further actions.  A reporting form should be specified to ensure that all necessary and relevant information is preserved and that there is consistency in the information gathered. 

\paragraph{Assessment and Decision} This phase includes assessment of information regarding security events and decisions about whether events should be treated as incidents. The assessment and decision phase also includes assessment of information received regarding vulnerabilities and decisions of how to handle these in accordance with previously agreed actions.

The PoC should use a predefined classification scale to make an assessment of security events, whether they are incidents or false alarms and what impact they may have on the organization's core services, information and affected assets. The initial assessment made by the PoC should be verified by an \ac{ISIRT} member. The \ac{ISIRT} makes decisions about how the incident should be handled, by whom and in what priority. To be able to respond to security incidents in an efficient and effective way, a prioritization process should be conducted based on the level of adverse business impact and the required effort to solve them.  All information pertaining to an incident should be recorded in the database by the \ac{ISIRT}. A main activity for the \ac{ISIRT} is to allocate responsibilities for incident management actions and provide thorough and structured procedures for people involved. 

\paragraph{Responses} The third operational phase presents guidelines and activities for organizations to use when responding to security incidents. The response should be in accordance with the actions agreed in the previous phase. This phase also involves responding to vulnerabilities reported either internally or by external parties. As a first step, the \ac{ISIRT} has to determine whether the incident is under control, and then initiate appropriate actions. For situations out of control, escalation to crisis handling might be necessary. Otherwise, response activities including recovery, proper documentation and communication to relevant parties can be started. 

The \ac{ISIRT} should consider which internal and possibly external resources to utilize for optimal incident response. It is important that every action conducted by the \ac{ISIRT} in this phase is logged properly and that guidelines are used to ensure thorough documentation. Logging will aid in analysing how effective and efficient the incident response process was as well as ensuring that any possible evidence is not compromised. It is the \ac{ISIRT}'s responsibility to make sure affected assets become operational again and that they are not vulnerable to the same attacks. Once an incident has been handled, the case should be closed formally by the \ac{ISIRT} and recorded in the database.

\paragraph{Lessons Learned} The final phase starts after an incident has been resolved and/or closed and focuses on analysing whether the organization's incident management scheme worked successfully. During this phase improvements are identified and implemented. One of the main activities is reviewing how effective the entire incident management process was in responding to, assessing and recovering from the incident. Shortcomings and improvements in policies, procedures, security control implementations, reporting formats and risk assessments should be identified during this phase. Improvements may be implemented immediately or incorporated into future plans. The \ac{ISIRT} should make sure improvements are made to the entire system and not only the affected parts.

The lessons learned phase has many iterative activities. An essential post-incident activity is documenting incidents properly as well as ensuring that the incident trend analysis is accurate. Sharing experiences with trusted communities and partners should be done on a regular basis, regardless of whether incidents occur internally. Reviews, trend analysis and testing should be performed frequently to improve the incident management scheme over time. 




