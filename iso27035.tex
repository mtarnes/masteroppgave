\subsection{\acs{ISO}/\acs{IEC} 27035}
This subsection gives an introduction to the ISO 27035 standard and the content is, unless specified otherwise, derived from \cite{ISO/IEC27035}. Despite organizations having implemented information security policies and controls, it is inevitable that new vulnerabilities and thus information security incidents occur occasionally. Thus it is essential that organizations have a structured and planned approach to detection, reporting, assessment, responding and learning related to information security incidents. Additionally it is important to have a planned approach to reporting of vulnerabilities that have not yet been exploited. This will aid organizations to deal with information security incidents properly and mitigate both direct and indirect adverse business impact. This standard provides an extensive and structures approach to information security incident management by presenting five phases with recommended activities for large and medium-sized organizations. One of the standards objectives is to provide guidelines to aid organizations in meeting the requirements specified in ISO/IEC 27001, and a cross-reference table showing how following the guidelines presented fulfils the requirements is presented.

The standard uses the definitions given in ISO/IEC 27000 as quoted in \ref{sec:Definitions}. Another definition worth mentioning is:

\textbf{Information security incident response team:} Team of appropriately skilled and trusted members of the organization that handles information security incidents during their lifecycle.

\paragraph{Plan and Prepare} This phase is by fare the most extensive phase, and each individual organizations ensure that their use of resources are proportional to their needs. The first phase involve planning and preparing to handle information security incidents successfully. The standard highlights the importance of ensuring commitment from senior management in developing an information security incident management policy. It is essential for the success of a structured incident management approach that awareness and participation of all personell is present. All personell should be familiar with the incident management scheme when it comes operational and be able to recognize the benefits of the approach. 

An important part of the preparatory phase is the establishment of the ISIRT. The structure of the team may vary depending on the organizations resources and needs. The team may be dedicated, virtual or a mix of the two. However, the standard recommends a permanent team. 

Procedures should clearly state what to do when an incident occurs and whom to contact for further assistance. There should be established a \acs{PoC} that handles reported incidents in the detection and reporting phase.

After the initial phase, the standard presents three subsequent phases that focuses on  security incidents management in operation. 

\paragraph{Detection and Reporting} The detection and reporting phase emphasizes the importance of preserving information about vulnerabilities and incidents in a database operated and maintained by the ISIRT. This phase includes activities to detect, collect information about and report occurrences of security events and vulnerabilities either discovered by humans or automated systems. It is the person first notified about an event that is responsible for starting the activities involved in this phase. All relevant information should be passed to the PoC and the responsible ISRIT member. There should be a structured way of dealing with information security incidents as there is for other security incidents. There are several ways a security event or incident can be discovered and thus all personell should be aware of and have access to the guidelines for reporting. There should be clear procedures to follow for the people involved in dealing with an incident. It is recommended that one of the ISIRT members are appointed the responsibility for incoming reports and makes an assessment for further actions.  A reporting form should be specified to ensure that all necessary and relevant information is preserved and that there is consistency in information gathered. 

\paragraph{Assessment and Reporting} The second operational phase is the assessment and decision phase. During this phase an evaluation of whether events should be consider incidents are made along with assessment of whether escalation is required. It is recommended that this decision is the responsibility of an ISIRT team member. An essential activity during this phase is the evaluation on how the incident should be dealt with, what its priority should be and who the person responsible for handling it is. Possible consequences of an incident should also be considered. To be able to respond to security incidents in an efficient and effective way, a prioritizing process based on adverse business impact and the needed effort to solve them should be conducted to ensure that incidents are assigned to the most suitable persons. 

\paragraph{Responses} The third operational phase presents guidance and activities on how to best respond to security incidents. The response should be in accordance with the actions agreed in the previous phase, whether it is an immediate, real-time or near real-time response or involves forensics analysis. As a first step the ISIRT has to determine whether the incident is under control.  




 should be made to possibly initiate crisis activities. If the incident is determined to be under control, later responses, such as forensics analysis, can be conducted. It is important to keep the database updated and formally record and close incident. After the incident has been handled appropriately it is the ISIRT's responsibility to make sure the affected assets becomes operational again and that they are not vulnerable to the same attacks. 


\paragraph{Lessons Learned} The final phase starts after an incident has been resolved and/or closed and focus on analysing whether the organization's incident management scheme worked successfully. One of the main activities is reviewing how effective the entire incident management process were in responding to, assessing and recovering from the incident. Shortcomings and improvements in policies, procedures, security control implementations, reporting formats and risk assessments should be identified during this phase. Improvements may be implemented immediately or incorporated into future plans. The ISIRT should make sure improvements are made to the entire system and not only the affected parts.

The lessons learned phase has many iterative activities. An essential post-incident activity is documenting incidents properly, ensuring incident trend analysis is accurate. Sharing experiences with trusted communities and partners should be done on a regular basis, regardless of whether incidents occur internally. Reviews, trend analysis and testing should be performed frequently to ensure regularly improvements to the incident response scheme over time. 




