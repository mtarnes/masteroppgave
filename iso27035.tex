\subsection{\acs{ISO}/\acs{IEC} 27035}
This section gives an introduction to the ISO 27035 standard and the content is, unless specified otherwise, derived from \cite{ISO/IEC27035}. Despite organizations' implementation of information security policies and controls, it is inevitable that new vulnerabilities and thus information security incidents occur occasionally. Thus it is essential that organizations have a structured and planned approach to detect, report, assess, respond and learn from information security incidents. Additionally it is important to have a planned approach to reporting of vulnerabilities that have not yet been exploited. Implementing this standard will aid organizations in dealing with information security incidents properly and mitigate both direct and indirect adverse business impact. This standard provides an extensive and structured approach towards incident management by presenting five phases with recommended activities for large and medium-sized organizations. 

One of the standard's objectives is to provide guidelines to aid organizations in meeting the requirements specified in ISO/IEC 27001. For this purpose a cross-reference table showing how following the guidelines in ISO/IEC 27035 fulfils the requirements listed in ISO/IEC 27000.

\paragraph{Plan and Prepare} This phase is by fare the most extensive phase, and each individual organizations have to ensure that their use of resources are proportional and in accordance with their needs. Each organization needs to formulate an incident management policy reviewing current vulnerabilities, stating the need for an incident management scheme and identifying overall benefits for the organization. Security and risk management policies should be reviewed and updated regularly. The standard highlights the importance of ensuring commitment from senior management in the security incident management policy to ensure the organization's commitment to resources and maintenance of an incident management capability.  

One main activity in the plan and prepare phase is making a detailed incident management scheme that contains forms, procedures and support tools for the detection and reporting of, assessment and decision making related to, making responses to and learning lessons from incidents. The scheme should include a classification scale for grading incidents and reporting forms (preferably electronic).   

An important activity in this phase is the establishment of the ISIRT. The structure of the team may vary depending on the organizations resources and needs. The team may be dedicated, virtual or a mix of the two. However, the standard recommends a permanent team. Responsibilities, processes, allocation of roles and an appropriate training program should be prepared.

Organizations should also establish and implement the required mechanisms of support for their incident management scheme to operate efficiently during this phase. This includes technical tools such as IDS and log monitoring systems as well as relationships and connections to other organizations. 

It is essential for the success of a structured incident management approach that awareness and participation of all personell is present. All personell should be familiar with the incident management scheme when it comes operational and be able to recognize its benefits. It is therefore recommended that an appropriate awareness and training program is developed and repeated from time to time as personell change over time.

The entire incident management scheme should be tested to verify that the scheme and ISIRT work in complex and real situations. When this phase is completed, organizations should be fully prepared to manage security events, incidents and vulnerabilities that occur.

\paragraph{Detection and Reporting} The first operational phase of an incident management scheme involves detection, collection of information and reporting of occurrences of security events, incidents and vulnerabilities either discovered by humans or automated systems. It is important to preserve information about vulnerabilities and incidents in a database operated and maintained by the ISIRT. Organizations should implement security monitoring systems, IDS/IDP, antivirus programs and so forth to aid the detection of security events, incidents and vulnerabilities. Logs from various entities should be analysed and registrations of incidents should be made in an Incident Tracking System. 

It is the person first notified about an event that is responsible for starting the activities involved in this phase. There are several ways a security event or incident can be discovered and thus all personell should be aware of and have access to the guidelines for reporting. There should be clear procedures to follow for the people involved in dealing with an incident. All relevant information should be passed to the PoC and the responsible ISRIT member. It is recommended that one of the ISIRT members are appointed the responsibility for incoming reports and makes an assessment for further actions.  A reporting form should be specified to ensure that all necessary and relevant information is preserved and that there is consistency in information gathered. 

\paragraph{Assessment and Decision} This phase include assessment of information regarding security events and decisions on whether events should be treated as incidents. The assessment and decision phase also include assessment of information received regarding vulnerabilities and decisions of how to handle these in accordance with previously agreed actions.

The PoC should use a predefined classification scale to make an assessment of security events, whether they are incidents or false alarm and what impact they may have on the organization's core services, information and affected assets. The initial assessment made by the PoC should be verified by an ISIRT member. ISIRT makes decisions about how the incident should be handled, by whom and in what priority. To be able to respond to security incidents in an efficient and effective way, a prioritizing process should be conducted based on the level of adverse business impact and the required effort to solve them.  All information pertaining to an incident should be recorded in the database by ISIRT. A main activity for the ISIRT is to allocate responsibilities for incident management actions and provide thorough and structured procedures for persons involved to follow. 

\paragraph{Responses} The third operational phase presents guidelines and activities for organizations in responding to security incidents. The response should be in accordance with the actions agreed in the previous phase, whether it is an immediate, real-time or near real-time response or involves forensics analysis. This phase also involve making responses to vulnerabilities reported either internally or by other parties. As a first step the ISIRT has to determine whether the incident is under control, and then initiate the appropriate actions. For situations out of control, escalation to crisis handling might be necessary for further assistance. Otherwise, the response work such as recovery, proper documentation and communication to relevant parties can be started. 

The ISIRT should consider which internal and possibly external resources to utilize for best responding to incidents. It is very important that every action conducted by the ISIRT in this phase is logged properly and that guidelines are used to ensure thorough documentation. Logging actions will aid in analysing how effective and efficient the incident response process were as well as ensuring that any possible evidence is not compromised. It is the ISIRT's responsibility to make sure the affected assets becomes operational again and that they are not vulnerable to the same attack. Once an incident has been handled successfully, the case should be closed formally by the ISIRT and recorded in the database.

\paragraph{Lessons Learned} The final phase starts after an incident has been resolved and/or closed and focus on analysing whether the organization's incident management scheme worked successfully. During this phase improvements are identified and made. One of the main activities is reviewing how effective the entire incident management process were in responding to, assessing and recovering from the incident. Shortcomings and improvements in policies, procedures, security control implementations, reporting formats and risk assessments should be identified during this phase. Improvements may be implemented immediately or incorporated into future plans. The ISIRT should make sure improvements are made to the entire system and not only the affected parts.

The lessons learned phase has many iterative activities. An essential post-incident activity is documenting incidents properly, ensuring incident trend analysis is accurate. Sharing experiences with trusted communities and partners should be done on a regular basis, regardless of whether incidents occur internally. Reviews, trend analysis and testing should be performed frequently to ensure regularly improvements to the incident response scheme over time. 




