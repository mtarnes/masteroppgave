This section describes the results from interviews and document study for Case A. 

\subsection{Preparation}
The IT security manager defines an incident %Skal endre dette til information security? Han nevnte at det ikke trengte å være IT-relatert engang, så kanskje ok å bare skrive incident?
as the occurrence of something unwanted. This can something that belongs to a predefined category of unwanted incident or an incident of a nature that you see is unwanted when it occurs. More specifically an \ac{ICT} incident will usually involve loss of information or loss of control over information systems. This is a definition that is known in the organization. The categories of unwanted incidents are determined through a risk assessment. This is included in their information security policy which is approved by the management and distributed to all parts of the organization. The policy refers to a document with principles for information security, which contains principles for risk management. %Se prinsipper for informasjonssikkerhet del 1, referer videre til rutiner for risiko og sårbarhetsanalyse

The IT security manager states loss of information related to the organization's core activities as being the worst information security incident they can experience. This could damage their relationship to partners, their general reputation and credibility. Additionally it may damage the work and career of individuals with ownership of the information. %Gir dette bort for mye info? skjnner man kanskje atd et er snakk om forskning?
Another aspect is that they own information that should not fall into the wrong hands. This can be information that goes under the \ac{UN} treaty on the non-proliferation of nuclear weapons (NPT) \cite{NPT}, which includes information that may be used for spreading of nuclear weapons and weapons technology. %Er dette ok å skrive??
The loss of such information could obviously have severe consequences.

The organization generates a report once a years, that contains the incidents they have experienced. This way they have an overview over previous incidents.

\paragraph{\acl{IRT}}

\paragraph{Awareness and Training}

\subsection{Detection and Analysis}

\paragraph{Initial Detection}
%Ta med rapporteringsrutiner og ansattes kjennskap til disse her??

\paragraph{Categorization}

\subsection{Incident Response}

\paragraph{Workflow}

\paragraph{Escalation}

\paragraph{Electronic Evidence}

\subsection{Lessons Learned}