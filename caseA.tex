This section describes the results from interviews and document study for Case A. 

\subsection{Preparation}
The IT security manager defines an incident %Skal endre dette til information security? Han nevnte at det ikke trengte å være IT-relatert engang, så kanskje ok å bare skrive incident?
as the occurrence of something unwanted. This includes both occurrences belonging to a predefined category of unwanted incidents and occurrences of a nature that you see is unwanted. More specifically an \ac{ICT} incident will usually involve loss of information or loss of control over information systems. This is a definition that is known in the organization. The categories of unwanted incidents are determined through risk assessments. It is stated in their principles for information security that they have to perform risk assessments and this document is approved by the management and distributed to all parts of the organization. Their information security policy refers to this document. Several of the principles address risk management. They have another document that further specifies their risk assessment process. %Se prinsipper for informasjonssikkerhet del 1, referer videre til rutiner for risiko og sårbarhetsanalyse

The IT security manager points out loss of information related to the organization's core activities to be the worst information security incident they can experience. This could damage their relationship to partners, their general reputation and their credibility. Additionally it may damage the work and career of individuals with ownership of the information. %Gir dette bort for mye info?
Another aspect is that they own information that should not fall into the wrong hands. This can be information that goes under the \ac{UN} treaty on the non-proliferation of nuclear weapons (NPT) \cite{NPT}, which includes information that may be used for spreading of nuclear weapons and weapons technology. %Er dette ok å skrive??
The loss of such information could obviously have severe consequences.

The organization has, as mentioned, an information security policy and a principles for information security document. The latter is to be updated if there are changes in the threat landscape or at least every other year. It is currently a subject for audit. It is up to each department to make sure that employees and other users are aware of these documents. The IT security manager does not expect that all of the users have detailed knowledge of these documents, even though it is stated that all users have to comply with them. The IT department uses these documents actively. %NB spør om dette i ansatteinetrvju!!!
During the last years the level of threat has increased for this organization and as a consequence they are going to perform a risk and vulnerability assessment for all the departments in the organization. In addition to the mentioned documents, they have an internal policy for incident handling.

They have several systems that are critical for their operations. This means that they need backup solutions for these systems in case of incidents. The IT security manager thinks that they have very good continuity plans. They have among other things tried to remove any single point of failure.

They have an information department that is responsible for all external communication and they cooperate closely with the IT department. The organization has a procedure for contact with the police in case of any violation of law. If the organization wishes to report a crime it is the manager that has to do this, although the IT security manager will provide the police with any relevant information he has access to. The manager is the only one that can report crimes on behalf of the organization, as this is a government owed organization.

They do not have a holistic plan for incident handling because no incidents are the same. Related to that he stated:

\begin{quote}
\textit{``To have a detailed guideline that takes all possibilities into account and that you have to change each time you get a new system %. I think that would be 
[...] we are not that good [...] and I don't think we have ambitions to be that good either"}
\end{quote}

They do however have general guidelines for handling of security incidents. Certain types of incidents are repeating, like phishing, and they have more detailed guidelines for these types of incidents.

If a system owner discovers a vulnerability in his system he will notify the IT security manager about it and about the scheduled update for the patch. The process is the same if the system owner is generally not satisfied with the security of the system.

\paragraph{\acl{IRT}}
Because the organization has an internet domain they are required to have an abuse email-address and they have a team that is responsible for receiving and handling cases reported to this address. Some of the members of the abuse team are part of the IT-support function of the organization and are part-time employees. Cases that involve employees, botnets or spam are handled by full-time employees at the service desk. One of the full-time employees is the leader of the team. The IT security manager is also part of this team in a supervising role. The team consists of seven people. The cases sent to abuse are not just received on an email address, but go straight into a case management system. Incident handlers are required to give the reporter an answer so that he knows that the potential incident is being looked into and that his report is appreciated.

The usual notifications will be received during work-hours, but they also have someone available 24/7 if something extra serious occurs. In that case they need to be alerted specifically, and it will typically be a system owner who will notify the person on duty.

They have a tool for mapping IP-addresses to users at specific times, that they need to use related to incidents. There is one person in the team that is responsible for this tool. They also have someone responsible for administrative tasks such as shifts. There is additionally someone who is responsible for documentation, especially in cases they have not seen before.

The team performs a limited amount of preventive work. They see repeating incidents and have reviews of larger incidents and use this to identify changes that are needed. They use the \ac{ITIL} framework, discussed in section \ref{section:ITIL} of this report, and treat repeating incidents as problems.

They often need to communicate with the various sections of the organization when incidents occur. The people working there are not specialized in handling security incidents, but they know how the systems work, and are therefore important resources. There is a daily designated contact person for some of the sections. The IT security manager mentioned that this communication can be challenging, especially for the sections that do not have a permanent designated contact person. Last year they experienced a targeted phishing attack and this led them to gather a team consisting of resources from various sections.

Most of their employees involved in incident handling have an IT background and thereby a solid technological knowledge. Additionally they go through a training process when they are hired. There is also a thorough interview process. During their time in the job they often get new tasks and need training for this. 

\paragraph{Standards and Guidelines}
The organization uses \acs{ISO}/\acs{IEC} 27001 and 27002, but they do not have as a goal to be certified. Their information security policy states that they are to be in compliance with these standards. The IT security manager thinks that a certification would be useful, because the most important part of the certification is the commitment of the management related to the standard. Because they are a government owned organization they are required to use \acs{ISO} standards. This is a relatively new requirement and he sees it as being one step closer to a certification.

The organization has reviews of incidents and uses this to make changes in their \ac{ISMS}. They also use \ac{ITIL} in their security work and incident handling. They do not use all recommendations from \acs{ISO}/\acs{IEC} 27001 and 27002 and \ac{ITIL}, but choose the parts they see relevant for their organization. The IT security manager has heard about \acs{ISO}/\acs{IEC} 27035, but they do not use it.

\paragraph{Awareness and Training}
The organization conducted an information campaign with an external supplier about a year and a half ago. The supplier delivered slides containing small lectures that were sent via email. These were lectures about themes such as how to protect your password. In relation to their attitude towards awareness, he stated: 

\begin{quote}
\textit{``There is a principle that says, never waste a good crisis"}
\end{quote}

This means that they use incidents, typically that others have experienced and communicate these to the organization. He mentioned that personally he takes any opportunity to talk about the importance of protecting information.

The IT security manager thinks that their classification of information is not satisfying. He thinks that users are not aware that the information security policy states that they have to classify the information they process. This highlights that the policy needs to be better established in the organization. This is the managers responsibility and he emphasizes again that a more extensive use of \acs{ISO}/\acs{IEC} 27001 and 27002 could help as this would lead to increased management commitment.

About a year ago they conducted a contingency rehearsal with information security as the chosen theme. The management was involved in this training. It included seven levels of escalation. The training contributed to increased information security awareness in the management as well as being a test of their central contingency plan.

He thinks that you will always benefit from rehearsal. Everything he can use to increase management awareness related to information security is beneficial. It is however challenging to conduct rehearsals and he stated:

\begin{quote}
\textit{``The most important part of preparing a rehearsal is to make sure that the responsible people train on the right things and to create a good and relevant game that challenges the involved people and that they feel is realistic [...] That is the most important thing, to give them what they need in order to be able to handle a real situation."}
\end{quote}

They use their risk and vulnerability assessments to determine what to focus on in rehearsals. The use rehearsals both for situations where they already have routines and for situations where they do not yet have any. A rehearsal may identify what routines they should have.

He says that they have many examples of issues that have been revealed through rehearsals. One such example is that even though various employees may have a certain risk awareness they may not agree on what the actual risk is. They have planned rehearsals this year as well, where one is to be conducted in cooperation with a partner organization. %Heter det oartner organization?

\subsection{Detection and Analysis}
The organization uses several means for detection of information security incidents.

\paragraph{Initial Detection}
%Ta med rapporteringsrutiner og ansattes kjennskap til disse her??
Incidents can be reported via the abuse system, as described in the \ac{IRT} paragraph. Additionally their internet supplier notifies them if they see that there are any compromised hosts in their network. They can also assist in the incident handling upon request.

The organization has a relatively new deviation management system. This system can be used to report various types of deviations, from information security to \ac{HSE} related deviations. It does not necessarily have to be an incident, but it can be any type of deviation, such as a vulnerability. It is mainly used for internal cases, and thus differs from the abuse system. The IT security manager hopes that this system will be used more for information security related incidents than it is today. This system also works as a database of incidents, as it includes information about all reported incidents (in the system).  

They have a tool that can be used to monitor connections to their network. This can be a source for incident detection.

The IT security manager thinks that they have underreporting but that people in the organization are good at reporting things like suspicious emails. He does however think that they are not good at reporting the extent to which they have actually disclosed any sensitive informations. He says that this originates in establishment of attitude and training. He states that:

\begin{quote}
\textit{``Users of a system need to understand what possibilities there are in the system, but also what limitations there are."} 
\end{quote}

\paragraph{Categorization}
The organizations categorizes incidents based on type (spam, phishing, botnet etc.) and based on what service or system they belong to. Incidents reported in their deviation management system are categorized based on whether they are \ac{HSE}-related technical or of other types. The category information security does not currently exist in the system, but it is planned to be included.

They also categorize incidents based on impact. They use the categories low, medium and high. Medium is when service is unavailable for several persons and high is when service is unavailable for the whole organization. Other incidents are categorized as low. The assigned impact category sets a time limit for how fast the incident must be resolved.

\subsection{Incident Response}
Cases reported to abuse are categorized and dispatched to the second line of incident response. %hmm, litt usikker på hva akkurat det betyr
From that point it is either resolved or transferred to another section, such as the network section, if they are better equipped to handle it. It can also be solved in cooperation between incident handlers and employees from other sections.

The IT security manager is not very concerned about getting systems up and running as fast as possible after an incident. This is because he wants to make sure that that the incident is properly resolved. There have been cases where he wanted to perform risk and vulnerability assessment of systems, but has not been allowed to do so.

After an incident the system owner is asked to identify what has been lost or compromised. It is also their job to try to assess if the organization has suffered an economic loss. They do know that they have used employee time on the incident, but other factors, such as the value of information, are hard to estimate. The IT security manager wishes to work on value assessment of information. He wants to do this by asking the owners of information.

If an incident is serious it will be reported to the management. The IT manager is notified about larger incident, but not small routine cases.

An example where the IT security manager used their network monitoring tool, he discovered that several users logged in on the same foreign IP-address. He then contacted them and found that they were in fact abroad, and that their user accounts had not been compromised. By contacting them first the users avoided a cumbersome process of being blocked and having to regain access. The organization also has tools that they use to analyse emails in spam cases. They have the possibility to log in to the email server as admin and fetch emails sent to people. This is only done in serious cases, and if it is done, they only look at emails relevant to the case in order to maintain privacy. The IT security manager has only done this once, in relation to a targeted phishing attack. This attack happened in two stages where the first stage was targeted to the organization and was used to retrieve usernames and passwords from users. The compromised user accounts were subsequently used to send new bank-phishing emails.

\begin{figure}[H]
\hspace{-1.1cm}\includegraphics[scale=0.53]{WorkflowCaseABotnet.png}
\caption[Workflow for a Botnet Incident, Case B]{Workflow for a Botnet Incident}
\label{fig:WorkflowCaseABotnet}
\end{figure}

\paragraph{Workflow}
The organization has specified workflows for specific types of incidents. They do not have a general workflow that applies for all incidents. Figure \ref{fig:WorkflowCaseABotnet} illustrates the workflow for a botnet incident. The figure is derived from a description given at the interview. 

\begin{itemize}\itemsep-0.2cm
\item The process is initiated by a report received in their abuse system or from some other source. 
\item The incident is categorized based on type (botnet, virus etc.)
\item If there is more than one affected host, the case will be split into one case per host and the following steps will be taken for each case.
\item If the incident is particularly serious the host will be blocked and the user in question will subsequently be contacted.
\item If the incident is not so serious the user in question will be contacted and subsequently blocked.
\item If contact is not established the user will be blocked and then tried contacted again.
\item The affected host is cleaned up.
\item The user will regain access.
\end{itemize}

\paragraph{Escalation}
They have a contingency plan for the IT department, that describes a set of incidents. This plan is initiated if an incident is particularly serious. It does not directly target information security incidents, but it includes scenarios where systems are unavailable. A system is unavailable if they have lost control over either if someone has taken control over the system or if there has been unauthorized access. The loss of admin passwords is an example of such an incident. When there is an incident so serious that the contingency plan must be initiated the management is also involved. The contingency plan includes communication routines related to incidents. 

\paragraph{Electronic Evidence}
They have good cooperation with the police and in cases that may lead to criminal cases they do not try to investigate themselves, so that they will not destroy any evidence. If they suspect criminal activity they contact the police and do nothing else themselves prior to this contact. They do however block users upon request from the police, in order to preserve evidence.

When it comes to accessing user-owned files they follow the Norwegian Personal Data Act.

\subsection{Lessons Learned}
The IT security manager thinks that their routines usually work well. When they have not worked well there is a review of the incident.

An example of an incident that was handled well is an incident that happened after they implemented storage for the entire organization. The OS they used had a default configuration where ports were left open. This was not something they had paid much attention to before implementing their solution. There turned out to be a vulnerability in this OS that someone was able to exploit. This led to the organization being used to enhance a \ac{DDoS} attack. This was detected and reported by email. They saw that one host had several millions unique external connections during 24 hours. They contacted the user in question  and involved the network section to find out what was going on. The were able to reveal that there had been no unauthorized access to the storage system itself, but only to the OS used.

The IT security manager points to lack of staff  both in general and dedicated to information security as well as lack of routines as challenges related to information security incident handling in their organization: 

\begin{quote}
\textit{``I would rather say that it is perhaps too often that we have incidents where we do not have routines that are sufficiently described. %(godt nok beskreve rutiner)
As I said earlier, you cannot have routines for everything, but I also think it originates in the organization and how we are staffed to handle situations. There is no point in writing routines that do not establish ownership to a process [...] For example we have routines for handling a botnet incident, and these two cases were dispatched this morning and no-one has addressed them yet."}
\end{quote}

As a general statement he said: 

\begin{quote}
\textit{``There is always room for improvement"}
\end{quote}

The IT security manager gathers information about information security incidents and creates annual reports to the management. This way they have an overview over previous incidents. They do not evaluate absolutely all incidents as this would be too much work. It turns out that lack of awareness among users is often the root cause of incidents. The ``solution" to these incidents is therefore awareness-raising activities rather than changes in routines.

They have a process they call review. This includes processes and people and not only technical details. They have reviews for incidents of a certain scope and when they experience that their process is not efficient enough or clear enough. The IT security manager is satisfied with their review process. Necessary identified improvements have been implemented and identified mistakes are not repeated.

They have exchanged information and experiences with both \acs{NorCERT} and other \acp{CERT} in relation to specific incidents.
