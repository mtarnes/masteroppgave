This section describes findings in Case A. 

\subsection{Preparation}
The IT security manager defines an incident %Skal endre dette til information security? Han nevnte at det ikke trengte å være IT-relatert engang, så kanskje ok å bare skrive incident?
as the occurrence of something unwanted. This includes both occurrences belonging to a predefined category of unwanted incidents and occurrences of a nature that you see is unwanted. More specifically an \ac{ICT} incident will usually involve loss of information or loss of control over information systems. This is a definition that is known in the organization. 

The categories of unwanted incidents are determined through risk assessments. It is stated in their principles for information security document that they are to perform risk assessments. This document is approved by the management and distributed to all parts of the organization. %Several of the principles address risk management. They have another document that further specifies their risk assessment process. %Se prinsipper for informasjonssikkerhet del 1, referer videre til rutiner for risiko og sårbarhetsanalyse.

The IT security manager identifies loss of information related to the organization's core activities as the worst information security incident they can experience. Such an incident could damage their relationship to partners, their general reputation and their credibility. Additionally, it may damage the work and career of individuals with ownership of the information. %Gir dette bort for mye info?
Another aspect is that the organization processes information that is very sensitive and it is crucial that this information is not disclosed to any outside party. An example is information that is defined by the \ac{UN} treaty on the non-proliferation of nuclear weapons (NPT) \cite{NPT}, which includes information that may be used for spreading of nuclear weapons and weapons technology. %Er dette ok å skrive??
The loss of such information would obviously have severe consequences.

The organization has, as mentioned, an information security policy and a principles for information security document. The latter is to be updated if there are changes in the threat landscape or otherwise at least every other year. It is currently a subject for audit. Each department is responsible of making sure that employees and other users are aware of these documents. The IT security manager does not expect that all users have detailed knowledge of these documents, even though it is stated that all users have to comply with them. The IT department uses these documents actively. %NB spør om dette i ansatteinetrvju!!!
During the last years the threat level has increased for this organization and as a consequence they are planning to perform a risk and vulnerability assessment for all departments in the organization. In addition to the mentioned documents, they have an internal policy for incident handling.

The IT security manager believes their continuity plans are satisfactory. They have among other things tried to avoid single points of failure to increase redundancy. Several of their systems are critical for their operations. As a consequence they need backup solutions for these systems in case of severe incidents. 

Their information department is responsible for all external communication and they cooperate closely with the IT department. The organization has procedures for contact with the police in case of violations of the law. If the organization wishes to report a crime it is the manager that has to do this, although the IT security manager will upon request provide the police with any relevant information he has access to. 

The organization has not developed a holistic plan for incident handling as there are many different incidents that require different responses. They do however have general guidelines for handling security incidents. Certain types of incidents, like phishing, are repeating and the organization have more detailed guidelines for these specific types of incidents. Related to that the IT security manager stated:

\begin{quote}
\textit{``To have a detailed guideline that takes all possibilities into account and that you have to change each time you get a new system %. I think that would be 
[...] we are not that good [...] and I don't think we have ambitions to be that good either"}
\end{quote}

If a system owner discovers a vulnerability in a system he will notify the IT security manager and inform about the scheduled update for the patch. 

\paragraph{Standards and Guidelines}
The organization has implemented \acs{ISO}/\acs{IEC} 27001 and 27002, but does not aim to be certified. Their information security policy states that they are to be in compliance with these standards. Because they are a government-owned organization they are required to implement \acs{ISO} standards. This is a relatively new requirement and the IT security manager sees it as being one step closer to a certification. He thinks that a certification could be useful, since the most important part of a certification is the management's commitment to the standard. The organization conducts reviews of incidents and uses this to make changes in their \ac{ISMS} in accordance with \acs{ISO}/\acs{IEC} 27001. 

They have implemented the \ac{ITIL} framework in their incident management and security work. They have not implemented all recommendations from \acs{ISO}/\acs{IEC} 27001 and 27002 and the \ac{ITIL} framework, but have chosen the parts they see relevant for their organization. The IT security manager is somewhat familiar with \acs{ISO}/\acs{IEC} 27035, but the standard is not implemented in the organization.

\paragraph{Awareness and Training}
The organization conducted an information campaign in cooperation with an external supplier about a year and a half ago. The supplier delivered slides containing small lectures that were sent to employees via e-mail. The lectures addressed themes such as how to protect your password. In relation to their attitude towards awareness, the IT security manager stated: 

\begin{quote}
\textit{``There is a principle that says, never waste a good crisis."}
\end{quote}

This means that they use incidents, typically incidents that others have experienced and communicate these to the organization. He mentioned that personally he takes any opportunity to talk about the importance of protecting information.

About a year ago the organization conducted a contingency rehearsal that addressed information security. The management was involved in this rehearsal. It had seven levels of escalation. The training contributed to increased information security awareness in the management as well as being a test of their central contingency plan.

The IT security manager believes rehearsals are always beneficial and thus uses all possibilities to increase management awareness related to information security. It is however challenging to conduct rehearsals. He stated:

\begin{quote}
\textit{``The most important part of preparing a rehearsal is to make sure that the responsible people train on the right things and to create a good and relevant game that are challenging to the involved people and that they feel is realistic [...] That is the most important thing, to give them what they need in order to be able to handle a real situation."}
\end{quote}

The organization uses their risk and vulnerability assessments to determine what to focus on in rehearsals. They use rehearsals both for situations where they already have routines and where they do not yet have any. A rehearsal may thus identify what routines they should implement. They have planned rehearsals this year as well, where one is to be conducted in cooperation with a partner organization. 

There are many examples of issues that have been revealed through rehearsals. One such example is employees' understanding of risk. Even though various employees may have a certain risk awareness they may not agree on what the actual risk is. 

The IT security manager claimed that their classification of information is not satisfactory. He thinks that users are not aware that the information security policy states that they have to classify the information they process. This highlights that the policy needs to be better established in the organization. This is the manager's responsibility. The IT security manager emphasized again that a more extensive use of \acs{ISO}/\acs{IEC} 27001 and 27002 could help as this would lead to increased management commitment.

\subsection{Detection and Analysis}
The organization uses several means for detection of incidents.

\paragraph{Initial Detection}
%Ta med rapporteringsrutiner og ansattes kjennskap til disse her??
Incidents can be reported via their abuse system, which is further described in the \ac{IRT} paragraph. Additionally their internet supplier notifies them if they see that there are any compromised hosts in their network. They can also assist in incident response upon request.

The organization has a relatively new deviation management system. This system can be used to report various types of deviations, from information security to \ac{HSE} related deviations. The deviation does not necessarily have to be an incident, but can be any type of deviation, such as a vulnerability. The system is mainly used for internal cases, and thus differs from the abuse system. The IT security manager hopes that the system will be used for information security related incidents to a larger extent than it is today. The system also works as a database of incidents, as it includes information about all reported deviations.  

They have a tool that can be used to monitor connections to their network. This can be a source of incident detection.

The IT security manager is convinced there is an underreporting of incidents. However, he thinks that people in the organization are good at reporting issues, such as suspicious e-mails but that they are not good at reporting to which extent they have actually disclosed sensitive information. He said that this originates in establishment of attitude and training. He stated that:

\begin{quote}
\textit{``Users of a system need to understand what possibilities there are in the system, but also what limitations there are."} 
\end{quote}

\paragraph{Categorization}
The organization categorizes incidents based on type (spam, phishing, botnet etc.) and based on what service or system they affect. Incidents reported in their deviation management system are categorized based on whether they are \ac{HSE}-related, technical or of other categories. An explicit category for information security deviations does not currently exist in the system, but they have planned to include it in the future.

The organization categorizes incidents based on impact as well. They use the categories low, medium and high. Medium is when service is unavailable for several people and high is when service is unavailable for the whole organization. Other incidents are categorized as low. The assigned impact category sets a time limit for when the incident must be resolved.

\subsection{Incident Response}
Cases reported to the abuse system are categorized and dispatched to the second line of incident response. %hmm, litt usikker på hva akkurat det betyr
From that point it is either resolved or transferred to another section, such as the network section, if they are better equipped to handle the incident. It can also be solved in cooperation between incident handlers and employees from other sections.

The IT security manager is not very concerned about getting systems up and running as fast as possible after an incident. He wants to make sure that incidents are properly resolved before restoring normal operations. There have been cases where he wanted to perform risk and vulnerability assessments of systems, but has not been allowed to do so.

After an incident the system owner identifies what information has been lost or compromised. It is also their job to assess whether the organization has suffered economical losses. They do know that they have used employees' time on the incident, but other factors, such as the value of information, are more difficult to estimate. The IT security manager said that he wanted to work on value assessment of information by asking information owners.

If an incident is serious it will be reported to the management. The IT manager is notified about larger incident, but not small routine cases.

The IT security manager provided an example of an incident using their network monitoring tool. He discovered that several users logged in with the same foreign IP-address. He contacted them and found that they were in fact abroad, and that their user accounts had not been compromised. By contacting them first, the users avoided a cumbersome process of being blocked and having to regain access. 

The organization has tools that they use to analyse e-mail in spam cases. They have the possibility to use an admin account to fetch e-mails in serious cases. If this is done, they only examine e-mails relevant to the case in order to ensure privacy. The IT security manager has only done this once, in relation to a targeted phishing attack. This attack happened in two stages where the first stage was targeted to the organization and was used to retrieve usernames and passwords from users. The compromised user accounts were subsequently used to send bank-phishing e-mails.

\paragraph{\acl{IRT}}
Because the organization has an internet domain they are required to have an abuse e-mail address. The organization has a team that is responsible for receiving and handling cases reported to this e-mail address. Some of the members of this team are part of the IT support function of the organization and are part-time employees. Others work at the service desk and are full-time employees. One of the full-time employees is the leader of the team. The IT security manager is also part of this team in a supervising role. The team consists of seven people. The cases sent to abuse are not only received on an e-mail address, but go straight into a case management system. Incident handlers are required to give the reporter an answer so that he knows that the potential incident is being looked into and that his report is appreciated. Incidents that involve employees, botnets or spam are handled by the full-time employees at the service desk.

Usually, notifications are received during work-hours, but the organization has someone available 24/7 if something extra serious occurs. In those cases they need to be alerted specifically, and it will typically be a system owner who notifies the person on duty.

%The organization has a tool for mapping IP-addresses to user accounts used at specific times. There is one person in the team that is responsible for this tool. They also have someone responsible for administrative tasks such as shifts. There is additionally someone who is responsible for documentation, which is especially important in new cases.

The team performs a limited amount of preventive work. They see repeating incidents and conduct reviews of larger incidents and use this to identify changes that are needed. They use the \ac{ITIL} framework, discussed in section \ref{section:ITIL}, and treat repeating incidents as problems that needs to be analysed further to find the root cause.

The team often needs to communicate with various sections of the organization when incidents occur. Regular employees are not necessarily specialized in handling security incidents, but they know how the systems work, and are thus important resources in resolving incidents. There is a daily designated contact person for some of the sections. The IT security manager mentioned that this communication can be challenging, especially for sections that do not have a permanent designated contact person. Last year they experienced a targeted phishing attack and this led them to gather a team consisting of resources from various sections.

Most of the employees involved in incident handling have an IT background and thereby a solid technological knowledge. Additionally they go through a training process when they are hired in addition to training when they are appointed new tasks. There is also a thorough interview process.

\begin{figure}[H]
\hspace{-1.1cm}\includegraphics[scale=0.53]{WorkflowCaseABotnet.png}
\caption[Workflow for a Botnet Incident, Case A]{Workflow for a Botnet Incident}
\label{fig:WorkflowCaseABotnet}
\end{figure}

\paragraph{Workflow}
The organization has specified workflows for specific types of incidents. They have not developed a general workflow that applies to all incidents. Figure \ref{fig:WorkflowCaseABotnet} illustrates the workflow for a botnet incident. The figure is derived from a description given during the interview. 

\begin{itemize}\itemsep-0.2cm
\item The process is initiated by a report received in their abuse system or from some other source. 
\item The incident is categorized based on type (botnet, virus etc.)
\item If there is more than one affected host, the case will be split into one case per host and the following steps will be taken for each case.
\item If the incident is particularly serious the host will be blocked and the user in question will subsequently be contacted.
\item If the incident is not so serious the user in question will be contacted and subsequently blocked.
\item If contact is not established the user will be blocked and then tried contacted again.
\item The affected host is cleaned up.
\item The user will regain access.
\end{itemize}

\paragraph{Escalation}
The organization has a contingency plan for the IT department that describes a set of incidents. This plan is initiated if an incident is particularly serious. It does not directly target information security incidents, but it includes scenarios where systems are unavailable. A system is unavailable if they have lost control over it. They have lost control if someone has taken control over the system or if there has been unauthorized access. The loss of admin passwords is an example of such an incident. When there is an incident so serious that the contingency plan must be initiated the management is also involved. The contingency plan includes communication routines related to incidents. 

\paragraph{Electronic Evidence}
The organization has a well functioning cooperation with the police. In cases that may lead to criminal cases they do not try to investigate themselves, to avoid compromising evidence. If they suspect criminal activity they contact the police and do nothing else prior to this contact. They do however block users upon request from the police, in order to preserve evidence. Accessing user-owned files is done in compliance with Norwegian Personal Data Act.

\subsection{Lessons Learned}
The IT security manager thinks that their routines usually work well. In cases where they have not worked well there is a review of the incident.

%An example of an incident that was handled well is an incident that happened after they implemented storage for the entire organization. The OS they used had a default configuration where ports were left open. This was not something they had paid much attention to before implementing the solution. It turned out to be a vulnerability in the OS that someone was able to exploit. This led to the organization being used to enhance a \ac{DDoS} attack. This was detected and reported by e-mail. They saw that one host had several millions unique external connections during a period of 24 hours. They contacted the user in question and involved the network section to find out what was going on. They were able to reveal that there had been no unauthorized access to the storage system itself, but only to the OS used.

The IT security manager points to lack of staff and routines as challenges related to incident management in the organization: 

\begin{quote}
\textit{``I would say that it is perhaps too often that we have incidents where we do not have routines that are sufficiently described. %(godt nok beskreve rutiner)
As I said earlier, you cannot have routines for everything, but I also think it originates in the organization and how we are staffed to handle situations. There is no point in writing routines that do not establish ownership to a process [...] For example we have routines for handling a botnet incident, and two cases were dispatched this morning and no-one has addressed them yet."}
\end{quote}

The IT security manager gathers information about information security incidents and delivers annual reports to the management. This way they keep an overview over previous incidents. They do not evaluate absolutely all incidents as this would be too much work. It turns out that lack of awareness among users is often the root cause of incidents. The ``solution" to these incidents is therefore awareness-raising activities rather than changes in routines.

They have a review process. They review processes and people and not only technical details. They have reviews for incidents of a certain scope and when they experience that their process is not efficient or clear enough. The IT security manager is satisfied with their review process. Necessary identified improvements have been implemented and identified mistakes are not repeated.

The organization has exchanged information and experiences with both \acs{NorCERT} and other \acp{CERT} in relation to specific incidents.

In relation to the use of lessons learned he stated: 

\begin{quote}
\textit{``There is always room for improvement"}
\end{quote}

\subsection{Employee Survey}
When employees were asked whether they were familiar with the organization's security policy their answers varied greatly. One fourth answered yes and one fourth no, whereas one half said they had heard about a security policy but that the content was not known in detail. 

Most employees said they had received suspicious e-mails, many on several occasions. However, one employee emphasized that it is primarily spam and rarely customized or targeted e-mails. None of the employees in the survey acknowledge to have opened attachments or carried out instructions given in these e-mails. The majority of the participants said they did not report suspicious e-mails to anyone, while a few said they report some cases.

Only a few of the employees in the survey claimed they knew what an information security incident is. About half acknowledged that they did not know or that they were not familiar with the definition, whereas the rest were unsure. When asked to give examples most employees mentioned sharing passwords, login or sensitive information as possible information security incidents. Even though many employees were uncertain what an information security incident is, all of them stated that they were attentive to incidents in their everyday work. One employee said that an implementation of a new door locking system to improve security in the department had lead to employees becoming more incautious and left their computers and office doors unlocked more often.

Only one employee claimed to know in which cases and to whom security incidents should be reported. The majority of the survey's participants were unsure which incidents to report. Nevertheless, most said that common sense guided their choices and that they would report to the nearest leader if necessary.

About half of the employees had participated in online lectures or presentations addressing information security. One employee emphasized that:

\begin{quote}
\textit{There are always some things that are useful to be reminded of.}
\end{quote}

Even though several employees said most of the content in these lectures was already known material, they still found it useful.