\chapter{Introduction}
\section{Scope}

\section{Problem Definition}

\section{Motivation}
Organizations are increasingly using and depending on information technology in business operations. As value and sensitivity of information increases, so does potential threats. Attacks get more advanced and attackers choose their targets wisely. A significant challenge is composed by new and severe security threats arising faster than implementations of measures against new threats. This leads to an increasing gap between threats and security measures in organizations.

According to a recent report\footnote{``Mørketallsundersøkelsen 2012 (Norwegian)", a survey identifying cybercrime and ICT incidents in Norwegian organizations} published by NSR\footnote{``Næringslivets sikkerhetsråd (Norwegian)" is a norwegian organization with objective to prevent crime in and against business by presenting security threats and trends.}\cite{Morketall2012}, there have been increase in information espionage and cyber crime in the last couple of years. In the same report, the \ac{NSM} made an assessment where results show that there is an overall low understanding of risks and that important measures such as risk assessment, improving skills of personnel, incident reporting and security audits are not satisfactory in organizations. Through our research, we hope to draw attention to some of the challenges faced by organizations. 

%Especially targeted attacks known as Advanced Persistent Threats (APT) are increasing. These attacks involve industrial, political and military espionage performed by attackers with extensive resources.  


%In recent years, trends show increase in severe ICT security incidents, more targeted and professional attacks and malware spread to and from mobile units. (Maria snakket om at SINTEF, i.e utsatt forskningsinstitusjon, hadde planer som gikk veldig langsomt å gjennomføre, kanskje vi kan gjøre noe bra her, komme med forbedringsforslag i og med at noen av våre participants i studien er i en utsatt gruppe?).


For the above mentioned reasons, it is interesting for us to look at how organizations do incident management in practice. How they do preventive work, handle incidents and learn from mistakes made both internally and by others are in focus. We want to assess how various factors contribute to the efficiency and effectiveness of organizations’ incident response processes. By identifying how various factors affect successful incident response, we hope to find improvements to incident management practice for organizations in the ICT sector so they might better protect against severe security incidents in the future. 


%We want to study how organizations perform information security incident management in practice because we want to assess how various factors contribute to the efficiency and effectiveness of organizations’ incident response processes. By identifying how various factors affect successful incident response, we hope to find improvements to incident management practice for organizations in the ICT sector so they might better protect against severe security incidents in the future. 



\section{Limitations}

\section{Outline}