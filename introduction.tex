\chapter{Introduction}
Computer security incidents have been a known issue ever since the introduction of the PC. However, in recent years there has been an increased focus on information security incidents. Several major incidents have received attention in media and have drawn attention to the topic. %A few studies exist, new standards such as ISO/IEC 27035 have emerged and several guidelines have been developed.

\section{Scope and Limitations}
We have collected information about three large Norwegian organizations by conducting qualitative interviews, document studies and a survey. We did not include any small or medium organizations, as we wished to study presumed experienced organizations and thus large organizations appeared as a natural choice. As we only included three organizations in our studies, generalization was not possible. The reason for not including more organizations in our research was the time restrictions for this thesis.

\section{Motivation}
Organizations are increasingly using and depending on information technology in business their operations. As value and sensitivity of information increases, potential threats increase accordingly. Attacks get more advanced and attackers choose their targets more strategically. A significant challenge arise when new and severe security threats evolve faster than corresponding measures. This leads to an increasing gap between threats and security measures in organizations. To avoid severe consequences and leakage of sensitive information this gap must be closed.

A recent report\footnote{``Mørketallsundersøkelsen 2012 (Norwegian)", a survey identifying cybercrime and \ac{ICT} incidents in Norwegian organizations.} published by NSR\footnote{``Næringslivets sikkerhetsråd (NSR) (Norwegian)" is a Norwegian organization with objective to prevent crime in and against business by presenting security threats and trends.} shows an increase in information espionage and cybercrime among Norwegian organizations in the last couple of years\cite{Morketall2012}. Especially targeted attacks, known as advanced persistent threats, appears to be increasing. These attacks involve industrial, political and military espionage performed by attackers with extensive resources aimed at a single organization. Typical attacks are customized e-mails with malicious attachments or links. Hence, it is important that exposed organizations are prepared to handle such attacks. 

On the basis of this increasing problem, it is interesting to look at how organizations perform incident management in practice. How organizations prepare for and handle incidents, comply with standards and learn from mistakes made both internally and by others are of interest. We wanted to assess how various factors contribute to the efficiency and effectiveness of organizations’ incident management. By identifying how these factors affect successful incident management, we hoped to find improvements to incident management practice for relevant organizations. 

%Awareness among employees is important to avoid severe incidents breaking out. The \ac{NSM} made an assessment where results show that there is an overall low understanding of risks in organizations and that important measures such as risk assessment, improving skills of personnel, incident reporting and security audits are not satisfactory\cite{Morketall2012}. Through our research, we hope to draw attention to some of these challenges faced by organizations and increase awareness. 

%In recent years, trends show increase in severe ICT security incidents, more targeted and professional attacks and malware spread to and from mobile units. (Maria snakket om at SINTEF, i.e utsatt forskningsinstitusjon, hadde planer som gikk veldig langsomt å gjennomføre, kanskje vi kan gjøre noe bra her, komme med forbedringsforslag i og med at noen av våre participants i studien er i en utsatt gruppe?).

%We want to study how organizations perform information security incident management in practice because we want to assess how various factors contribute to the efficiency and effectiveness of organizations’ incident response processes. By identifying how various factors affect successful incident response, we hope to find improvements to incident management practice for organizations in the ICT sector so they might better protect against severe security incidents in the future. 

\section{Objectives}
\label{sec:objectives}
We aimed to draw attention to and increase awareness around incident management. By investigating how various organizations perform incident management, what plans and procedures exist and to what extent they comply with standards, we also hoped to find potential improvements.

The main research question of this thesis is:
\begin{itemize}
\item How do organizations perform information security incident management in practice?
\end{itemize}

This research question is further divided into the following sub-questions:

\begin{itemize}\itemsep-0.1cm
\item What plans and procedures for information security incident management are established in organizations?
\item To what extent are existing standards/guidelines adopted in plans for information security incident management?
\item To what extent have previous information security incidents been handled in accordance with predetermined plans? 
\end{itemize}

\section{Outline}
Chapter \ref{chp:background} presents a background on information security incident management. This includes what incident management is, why it is needed as well as relevant standards and guidelines. Chapter \ref{chp:method} discusses the research method used for this study. In chapter \ref{chp:CaseIntroductions} the three cases in the case study are presented. The findings from the case study are presented in chapter \ref{chp:findings}. Chapter \ref{chp:discussion} discusses the findings presented in chapter \ref{chp:findings} and compares these with the literature presented in chapter \ref{chp:background}. Chapter \ref{chp:conclusion} includes a conclusion of the findings as well as suggestions for future work. In Appendix A the information sheet given to the participating organizations can be found. Appendix B contains the interview guide used as a basis for the collection of empirical data in this study. In Appendix C, the employee survey is included. All of the appendices are written in Norwegian.