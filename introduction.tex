\chapter{Introduction}
In the last couple of years there has been an increased focus on incident management. Some major incidents received attention in media and has drawn attention to the topic. A few surveys exist and new standards such as ISO/IEC 27035 has emerged and several guidelines has been developed.

\section{Scope}
This theses focus mainly on large, Norwegian organizations and how they do security incident management.\\

-large organizations, not SME.\\
-look at ICT security incidents, does not apply to other security breaches\\
-mainly focues on norwegian organizations, kan ikke nødvendigvis si noe om resten av verden.

\section{Motivation}
Organizations are increasingly using and depending on information technology in business operations. As value and sensitivity of information increases, so does potential threats. Attacks get more advanced and attackers choose their targets more wisely. A significant challenge arise when new and severe security threats evolve faster than corresponding measures. This leads to an increasing gap between threats and security measures in organizations. To avoid severe consequences in business operations and leakage of sensitive information this gap must be closed.

A recent report\footnote{``Mørketallsundersøkelsen 2012 (Norwegian)", a survey identifying cybercrime and ICT incidents in Norwegian organizations.} published by NSR\footnote{``Næringslivets sikkerhetsråd (NSR) (Norwegian)" is a Norwegian organization with objective to prevent crime in and against business by presenting security threats and trends.} show an increase in information espionage and cybercrime among Norwegian organizations in the last couple of years\cite{Morketall2012}. Especially targeted attacks, known as advanced persistent threats, appears to be increasing. These attacks involve industrial, political and military espionage performed by attackers with extensive resources aimed at a single organization. Typical attacks include customized e-mails containing malicious attachments or links. Hence, it is important that exposed organizations are prepared to handle such attacks and potential incidents caused by them. 

On the basis of this increasing problem, it is interesting to look at how organizations do incident management in practice. How organizations prepare for and handle incidents, comply with standards and learn from mistakes made both internally and by others are of interest. We want to assess how various factors contribute to the efficiency and effectiveness of organizations’ incident response. By identifying how these factors affect successful incident management, we hope to find improvements to incident management practice for relevant organizations. 

%Awareness among employees is important to avoid severe incidents breaking out. The \ac{NSM} made an assessment where results show that there is an overall low understanding of risks in organizations and that important measures such as risk assessment, improving skills of personnel, incident reporting and security audits are not satisfactory\cite{Morketall2012}. Through our research, we hope to draw attention to some of these challenges faced by organizations and increase awareness. 

%In recent years, trends show increase in severe ICT security incidents, more targeted and professional attacks and malware spread to and from mobile units. (Maria snakket om at SINTEF, i.e utsatt forskningsinstitusjon, hadde planer som gikk veldig langsomt å gjennomføre, kanskje vi kan gjøre noe bra her, komme med forbedringsforslag i og med at noen av våre participants i studien er i en utsatt gruppe?).

%We want to study how organizations perform information security incident management in practice because we want to assess how various factors contribute to the efficiency and effectiveness of organizations’ incident response processes. By identifying how various factors affect successful incident response, we hope to find improvements to incident management practice for organizations in the ICT sector so they might better protect against severe security incidents in the future. 

\section{Objectives}
We wish to draw attention to and increase awareness around incident management. By looking at how various organizations do incident management, what plans and procedures exist and to what extent they comply with standards, we also hope to find potential improvements.


\section{Limitations}

- few organizations, thus no generalization\\
- ICT incidents, not other.\\
- norwegian?

\section{Outline}