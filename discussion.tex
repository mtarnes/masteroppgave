\chapter{Discussion}
\label{chp:discussion}
In this chapter the findings from chapter \ref{chp:findings} are discussed and links between research questions and findings are established in sections \ref{sec:discussionCaseA}, \ref{sec:discussionCaseB} and \ref{sec:discussionCaseC}. The research questions were presented in section \ref{sec:objectives}. Further, underlying structures of experiences  are discussed in section \ref{sec:discussionStructures}. These findings are not necessarily  directly linked to the research questions, but emerged from the data. This analysis method is based on the general inductive research approach as described in section \ref{sec:qualitativeAnalysis}. Finally, section \ref{sec:rec} presents our recommendations for performing successful incident management.  

\section{Case A}
\label{sec:discussionCaseA}
SANS and \acs{ITIL} state that having an information security policy is important. Further, the \acs{ITIL} framework emphasizes that employees should have access to and be aware of this policy. Organization A has developed an information security policy. It may, however, not be well enough established throughout the organization. This is supported by the employee survey where few claimed to be familiar with the policy. It is also consistent with the fact that the IT security manager believes that not all users have detailed knowledge of the policy. Organization A is compliant with the ISO/IEC 27035 standard's recommendation of having a specific policy for incident handling. 

Management commitment is highlighted as critical by ISO/IEC 27001 and the IT security manager stated that it is the management's responsibility to make sure the policy is well established. The policy does not seem to be well established throughout the organization which might indicate that there is lack of management commitment as this is their responsibility.

It is stated in Organization A's information security policy that information shall be classified by the information owners. This is compliant with recommendations from the ISO/IEC 27002 standard that additionally emphasizes the importance of classification to ensure proper protection of information. The fact that so few of the employees were familiar with the information security policy could indicate that they are not aware of their responsibility for classifying their information. This is supported by the IT security manager who stated that the organization's information classification is not satisfactory. He believes that employees are not aware of this specific policy requirement. This could indicate that the organization's information is not sufficiently protected with regard to its sensitivity and value. Organization A processes large amounts of sensitive data, and this finding is therefore alarming.

Organization A has reporting and documentation systems and procedures, which is recommended by all guidelines presented in section \ref{section:standardsandguidelines}. They also follow the implementation guidance in the ISO/IEC 27002 standard which recommends that those reporting information security events should be notified of results. 

Organization  A has implemented monitoring systems, such as IDSs, which is recommended by most relevant standards and guidelines. In addition to technical detection mechanisms, users can be valuable resources for detecting incidents, and therefore organizations should have available reporting systems. This was highlighted in a presentation at the ``Sikkerhet \& S\aa rbarhet 2013" conference:

\begin{newquote}{Vidar Sandland and Hans Marius Tessem, NorSIS}
``Consider employees as part of the organization's sensor network."
\end{newquote}

This is especially important with regard to social engineering and targeted attacks, which are increasing. One interesting observation is that employees in Organization A seem to lack knowledge and qualifications to be able to recognize incidents, which might indicate that they are not fully utilized as resources for incident detection. We found that the employees that participated in the survey were unaware that they are required to report incidents, unaware of how to report and under which circumstances reporting is necessary. Even though the majority of the participants thought that they would be able to figure out whether incidents should be reported, this finding is alarming, especially if it is representative for the entire group of users. 

According to SANS, NorSIS, ITIL and ISO/IEC, incident prioritization rules should be based on an organizational impact analysis. One way to evaluate potential organizational impact caused by incidents is to conduct risk assessments. Organization A conducts risk assessments regularly, which has led to a categorization scheme that provides the basis for their prioritization. Hence, Organization A complies with these recommendations. Another important part highlighted in the standards and guidelines is the establishment of an \ac{IRT}. Organization A does not comply with this recommendation as they do not have their own \ac{IRT}, but only dedicated personnel for incident handling and a crisis team to handle the most severe incidents.

Several employees mentioned that information security did not concern them. They believed it was not relevant to their work and that they were not exposed to attacks or incidents, despite having access to sensitive information and performing their work on computers. Even though most of the employees in the survey did not know what an information security incident is, they still claimed to be attentive to incidents in their everyday work. These contradictory statements might indicate that information security is not well understood and that employees have an erroneous picture of their own security knowledge and awareness. These findings might indicate a lack of risk awareness among employees and are supported by the IT security manager who said that an issue revealed through rehearsals was employees' lack of understanding of risk. As mentioned in section \ref{sec:threatLandscape}, vulnerabilities in organizations exist mainly due to lack of employees' understanding of risk. This finding is therefore worth noting. We believe that by raising employees' awareness, certain vulnerabilities and incidents can be mitigated.

The ISO/IEC 27035 standard and the NorSIS guideline recommend to conduct rehearsals. Organization A complies with this recommendation. One observation is that Organization A's rehearsals include situations where they are aware of their lack of routines. This gives the team an opportunity to train on improvising in situations where there are no predefined plans. We believe that routines developed through such a bottom-up approach might be better established within the team than routines imposed by others, due to the team's participation in the development and implementation.

According to the ISO/IEC 27035 standard, employees' awareness and participation in incident management procedures are important. The employee survey indicates that employees are positive to awareness campaigns and have found previous campaigns useful. The positive attitude towards learning more about information security, might indicate that it is a lack of management commitment that is the reason for insufficient understanding and awareness of information security and not employees' attitudes. If that is the case, it would be unfortunate as senior management commitment to incident management is highlighted as important in the ISO/IEC 27035 standard. Another indication of a lack of management commitment to information security in Organization A is that the IT security manager is often not allocated the resources needed to ensure that the root causes of incidents are identified and eradicated. 

Organization A is compliant with the ISO/IEC 27035 standard's recommendation of having escalation procedures. The standard also specifies that it should be a main activity for the \ac{IRT} to allocate responsibilities. Organization A allocates responsibilities by delegating parts of the incident handling to employees with expertise relevant for solving the incident.

Organization A's lessons learned phase is relatively compliant with the recommendations in the majority of the standards and guidelines. They perform reviews of severe incidents to identify root causes and improvements. Further, these improvements are implemented and in specific cases shared with trusted communities and partners. The latter is specifically recommended in the ISO/IEC 27035 standard. We believe mutual sharing of experiences is beneficial for organizations as it will make them better prepared for handling incidents. Other organizations may have experienced incidents that can be avoided if the appropriate security measures are implemented. 

\paragraph{The research questions revisited}
Organization A has not implemented any specific standard or guideline for incident management, but has based their approach on components from the ISO/IEC 27001 and 27002 standards as well as the \ac{ITIL} framework. Still, they seem to comply reasonably well with recommendations in most of the standards and guidelines presented in this report. They have developed several plans and procedures addressing information security and incident management specifically, but not all of these seem to be well \textit{established} throughout the organization. Additionally, they do not always have the staff required to respond efficiently to incidents. Nevertheless, the overall impression is that incidents generally seem to be handled in accordance with their predetermined plans.

\section{Case B}
\label{sec:discussionCaseB}
Organization B has developed an information security policy, with intention to define senior management's IT security position. This might indicate some level of management commitment. Having a security policy is stated to be important by SANS and ITIL. Further, the ITIL framework recommends that employees should have access to and be aware of the information security policy. Organization B seems to be compliant with this recommendation as most employees answered that they were to some extent familiar with the organization's information security policy. 

The interviewees that participated in case B provided slightly different definitions of an information security incident. One variation in their definitions was that the interviewees from Organization B specified a distinction between \textit{security breaches}, i.e. incidents caused intentionally by employees, and other incidents. The interviewees from the suppliers did not specify this distinction. This makes sense as the two external suppliers are mainly concerned with affected systems, whereas it is the organization itself that handles incidents caused intentionally by employees. They all agreed that incidents causing loss of sensitive information are the worst possible incidents Organization B can experience. This common understanding might indicate that they have the same priorities during incident handling.

Even though not all employees knew what an information security incident is or could provide a definition, most of them gave relevant examples. This might indicate a reasonably sound understanding of information security. However, most employees stated to be attentive to incidents, even though they also said they did not know what an incident is. This shows that there is still room for improvement of employees' information security understanding and awareness.

Organization B and its suppliers have implemented various measures to prevent the occurrence of incidents. Prevention of incidents is stated to be fundamental to the success of an organization's incident response by NIST SP 800-61. Supplier 1 keeps track of trends related to security incidents, by monitoring their internal systems. This is compliant with recommendations for the preparation phase in the ISO/IEC 27035 standard.

Organization B's development of communication and escalation procedures is compliant with recommendations from SANS, NorSIS and ISO/IEC. In NorSIS's guideline for incident management it is emphasized that information security should be considered when \acp{SLA} are developed for outsourcing. Supplier 2's incident management plan is developed to ensure fulfilment of the \ac{SLA}, and this complies with the recommendation in NorSIS's guideline. 

Allocating resources for the development of detailed plans is not Organization B's main focus, as they believe having experienced incident handlers are more important for a successful incident handling. This is an interesting observation as standards and guidelines tend to focus more on plans and procedures than experienced incident handlers. This might indicate that Organization B has evaluated their own needs and perform their incident management accordingly. Further, Organization B performs regular rehearsal to test their plans and gain experience. This is compliant with the ISO/IEC 27035 standard and NorSIS's guideline. 

The supply chain manager highlighted information dissemination as one of the most challenging parts of incident management. A challenge mentioned by the interviewee from Supplier 2 was handling and collecting information from various sources. Organization B is well aware of these challenges and focuses on them in rehearsals. We believe making wrong decisions about information dissemination could cause delays in the incident handling and may result in serious consequences. The finding that information dissemination is challenging, supports the findings from a case study conducted by Ahmad et al. \cite{ahmad2012incident}. That case study's participants meant that better information dissemination would improve their security procedures, which in turn would improve the overall security of their organization. 

Both the supply chain manager and the interviewee from Supplier 2 highlighted allocation of responsibilities as a challenge in incident management. Some incidents may be so complex that knowing exactly where they originated, and thus determining who is responsible for handling them is difficult. We believe the challenge of determining who is responsible in various cases could be mitigated by improving communication procedures and establishing well defined responsibilities beforehand.  

Organization B conducts awareness campaigns that address various topics. Two employees stated that these campaigns were useless as the proposed security measures were too strict. These statements emphasize the importance of having an appropriate balance between security and usability. This was also discussed in one of the presentations at the ``Sikkerhet \& S\aa rbarhet 2013" conference:

\begin{newquote}{John Arild Amdahl Johansen, Buypass AS}
``Security must \textbf{never} stop business."
\end{newquote}

In this presentation it was stated that if security measures are too complex, users will find ways to circumvent the rules and thus the initial security measures are compromised. The two employees who did not find the campaigns useful have IT backgrounds which might indicate that employees' impressions of such campaigns vary with individual background and IT knowledge. Even though these two employees were familiar with the content of the campaigns, their answers indicated a negative attitude towards awareness raising activities, and might imply an unsatisfactory security culture in Organization B. However, it should be noted that most employees in the survey found awareness campaigns useful.

Organization B uses monitoring systems and employees as sources of incident detection, which is in accordance with recommendations from most of the standards and guidelines presented in section \ref{section:standardsandguidelines}. The employee survey indicated that the knowledge of reporting procedures for employees is not satisfactory, as most of the employees were not sure where to report incidents. Their overall uncertainty related to reporting may indicate that reporting procedures are not well enough established throughout the organization. Additionally, a few stated that they were not familiar with reporting routines as they had never needed to report anything. This attitude is similar to the one found in case A where some employees believed that information security did not concern them. 

Organization B uses a predefined classification scale based on impact level for the categorization of incidents. This is compliant with the ISO/IEC 27035 standard and ENISA's Good Practice Guideline for Incident Management. The categorization is further used to prioritize incidents. Incident prioritization based on impact level is recommended by ISO/IEC, SANS, NorSIS and ITIL.

All incidents are logged and the root causes of incidents are included in the log. Most of the standards and guidelines discussed in section \ref{section:standardsandguidelines} specify logging as being important. Further, the ITIL framework focuses on root cause analysis in the problem management process.

Organization B holds regular meetings where they discuss serious incidents and they perform trend analyses by evaluating incident reports. These activities are described as essential in the ISO/IEC 27035 standard. Overall, the organization's post-incident activities seem to be in accordance with relevant standards and guidelines. The organization has routines for preservation of electronic evidence, which is compliant with the ISO/IEC 27002 standard.

As described in the standards and guidelines, recovery is an important part of incident response. Organization B has tried to ensure a high level of redundancy, which we believe makes recovery easier and more efficient. Additionally, it might limit availability related consequences of security incidents. 

Werlinger et al. recommended incident handlers to acquire knowledge about the organization's IT systems and services in order to better be able to recognize abnormal behaviour\cite{werlinger2010preparation}. Organization B may have difficulties utilizing tacit knowledge as incident handling is to a large degree outsourced. However, we believe that the incident handlers at Supplier 1 and Supplier 2 may have gained such knowledge, as they handle Organization B's daily IT operations and application management respectively. 

\paragraph{The research questions revisited}
Organization B has not strictly based plans and procedures on standards or guidelines for incident management. Still, they are relatively compliant with the standards and guidelines presented in section \ref{section:standardsandguidelines}. Some of their procedures seem to be well established such as their escalation procedures. They do, however, have some procedures that do not seem to be sufficiently established. It seems that reporting procedures are not sufficiently established in the organization as employees showed uncertainty related to these procedures. Organization B has a set of predefined plans, but the importance of having experienced incident handlers is extra evident in this case as their incident handling is distributed and their team scalable. Their plans are quite general and they thus focus on being able to improvise during incident handling, i.e. make situation-specific decisions. Our overall impression is that incident handling has been performed in accordance with predefined plans.

\section{Case C}
\label{sec:discussionCaseC}
Organization C has an information security policy which is reasonably well known among participants in the employee survey. In one of the presentations at the ``Sikkerhet \& S\aa rbarhet 2013" conference, Difi\footnote{The Norwegian Agency for Public Management and eGovernment} recommended discussing security during employee appraisals. The IT security manager said that their security handbook is always a topic in the annual employee appraisals. Further, most of the employees in the survey seemed to have an understanding of what an information security incident is. This could indicate that information security is well understood among employees in Organization C. There are several findings that could explain this. It can be assumed that the annual review of the security handbook aids employees in becoming aware of their individual security responsibilities. Further, Organization C believes it is important to have a security-positive environment. The organization's focus on employees may have increased the overall security understanding. We believe an important factor contributing to this is that Organization C's core activity is delivery of IT services. Consequently, they have a high focus on information security.

ISO/IEC, SANS and NorSIS emphasize the importance of management commitment both to incident management and information security in general. We believe Organization C has some extent of management commitment as the aim of their information security policy is to communicate the management's direction and commitment to information security. One reason for this commitment might be that they have several ISO/IEC 27001 certifications, and this standard states that management shall provide evidence of its commitment to information security. 

The fact that all of the employees had attended courses or other awareness raising activities, supports Organization C's claim of having a high focus on improving employees' security knowledge and awareness. This observation is further supported by statements from the IT security manager, and may confirm that the organization follows through on their policy objectives. Additionally, employees' attitude towards awareness raising activities shows signs of a security-positive environment.  
 
The employee survey showed some uncertainty with regard to reporting procedures. The few employees that did not know where to report claimed to know where to find relevant information. Employees' knowledge of where to find relevant information is positive, but we believe this is not efficient enough in all situations as it introduces an extra delay. The IT security manager said that employees are advised to report incidents to the \ac{IRT}. However, none of the employees mentioned this. Suspicious e-mails was given as one example of cases that should be reported. Still, none of the employees had previously reported such e-mails. The fact that none of the participants in the survey had reported incidents could indicate that employees are not fully utilized as part of the organization's sensor network for detecting incidents. This assumption is supported by one of the employees who stated that they should probably report incidents more often. Further, this is supported by the IT security manager who suspects underreporting. This is unfortunate as Organization C tries to establish a security-positive environment. Underreporting might indicate that they still have some work to do with regard to achieving this. 

Organization C has monitoring systems for incident detection. This is in compliance with recommendations from ISO/IEC, NIST and ENISA. 

Handling vulnerabilities can aid in incident prevention, which is an important part of incident management and is stated to be a fundamental factor by NIST. Additionally, NorSIS specifies preventive measures to be one of the most cost effective ways to perform incident management. Organization C has a risk framework where vulnerabilities can be reported and measures can be implemented thereafter. They are thus in compliance with recommendations.

Organization C bases their incident categorization on impact level, which is in accordance with the categorization method from the ITIL framework. The categorization determines which incident response procedures to initiate. Categorizing incidents and using the categorization to determine further actions are compliant with recommendations from the majority of the standards and guidelines discussed in section \ref{section:standardsandguidelines}.

As illustrated in figure \ref{fig:workflowcaseC} in chapter \ref{chp:findings}, the service desk function is the first line of incident response. This figure, in combination with figure \ref{fig:workflowcaseCmajor} show Organization C's escalation routines. We believe this shows mature and well established escalation routines in Organization C, that are compliant with recommendations from the ITIL framework.

It is stated in Organization C's internal documentation that they believe successful incident management is based on contingency plans and predefined tasks. The employee we had e-mail correspondence with acknowledged that it would be ideal to have plans and procedures for all possible incidents, but that this might not be practically feasible. He emphasized that incident handlers who compose a set of predefined activities to customize the incident response for specific incidents are key to successful incident handling. He stated that due to variations in incidents, an experienced incident handler is more important than rigid process adherence. We believe that thorough preparation for incident handling is of utmost importance. However, incident handlers that are capable of having situational awareness are essential to utilize these preparations to the fullest. 

Organization C has developed procedures for handling electronic evidence. The establishment of routines for handling electronic evidence is compliant with NIST SP 800-61 and the ISO/IEC 27002 standard.

Two requirements for Organization C's incident management process are that all incidents should be registered and all actions logged. It is fair to say that the organization follows best practice, as most of the standards and guidelines discussed in section \ref{section:standardsandguidelines} emphasize the importance of logging.

The fact that Organization C has initiated a project to improve their incident management scheme shows their commitment to improve their incident management process. This project is allocated extensive resources which again supports the assumption of established management commitment to information security and incident management.

Incident handling is distributed among Organization C and its customers. Hence, the challenges of collaboration and coordination are evident in this case. Communication emerges as a challenge, which has been revealed through rehearsals. We believe the establishment of more specific communication routines as well as well defined responsibilities might mitigate these challenges.

\paragraph{The research questions revisited}
We believe Organization C has a set of well established plans and procedures as well as a focus on having experienced incident handlers. Additionally, they have several ISO/IEC 27001 certifications and has implemented the ITIL framework for their IT service management. Their incident management is highly compliant with the ITIL framework as well as relatively compliant with the other standards and guidelines presented in section \ref{section:standardsandguidelines}. It seems that incidents have mostly been handled in accordance with predefined plans. The uncertainty among employees with regard to reporting routines might indicate that these routines are not sufficiently established throughout the organization. Nevertheless, our findings indicate that this organization has an overall mature incident management process. 

\section{Prominent Challenges and Observations}
\label{sec:discussionStructures}
This section discusses challenges and observations that we found prominent in our case study. There are several factors involved in determining how successfully organizations respond to information security incidents. In chapter \ref{chp:introduction} we stated that we wanted to assess how these factors contribute to the efficiency and effectiveness of organizations' incident management. The challenges discussed in this section are some of the factors that are part of determining the level of success in organizations' incident management processes. It is important to note that our findings cannot be directly generalized. Due to the organizations' size and core activities they are extra vulnerable to attacks and we therefore have assumed that they are experienced in incident handling. Hence, we find it reasonable to believe that some of these challenges and observations will be evident in other organizations as well.

\subsection{Communication}
During our case study we found that communication was regarded as challenging among all of the participants. Both internal communication, within teams and towards employees, and external communication are part of this communication challenge. The organizations had to various extents developed and implemented plans and procedures addressing communication. Successful incident response requires cooperation, thus establishing sound communication procedures for incident management is essential. Communication is further emphasized as one of the most important parts of incident management by NIST. The organizations we studied are large organizations and it is therefore not surprising that several parties are involved in their incident management. Even for Organization A, that does not have to coordinate with external parties during incident handling, incident handlers have to communicate and coordinate across several internal departments and sections. As an example from our study we highlight that the designated contact person in Organization A changes daily for some sections which imposed uncertainty for the IT security manager. 

Communication becomes even more challenging with distributed organizational structures and the establishment of sound communication procedures is vital. Our impression is that it is important to have available and updated contact lists, but being able to determine the correct person to contact during incident response is just as important. In some cases, people with special knowledge or authorizations need to be involved. To be able to determine who the correct person is, situations have to be assessed and tacit knowledge about the organization and its employees is essential. This type of knowledge is difficult to document and thus difficult to include in plans. We therefore believe that in order to mitigate communication related challenges, employees involved in incident handling must have experience. 

A speaker at the ``Sikkerhet \& S\aa rbarhet 2013" conference presented results from an audit performed by The Norwegian Data Protection Authority that highlighted a problem we also found evident in our case study. E-mail is still used for unstructured and informal communication, even though it is not a secure channel. Using insecure communication channels exposes the organization to targeted phishing attacks, e.g. as seen in a recent attack against the large Norwegian telecom corporation Telenor\cite{phisingattack}. The organizations in our case study used e-mail for communication not only as a first notification of incidents but also during major incidents. To mitigate the risk of phishing attacks and disclosure of sensitive information, we recommend using more secure communication channels where this is practically feasible. 


\subsection{Information Collection and Dissemination}
Collecting information relevant to incident management was pointed out as challenging by participants in our case study. Especially for organizations with distributed organizational structures, there are many sources of information which makes the collection of correct information difficult. This observation supports findings from a case study conducted by Ahmad et al.\cite{ahmad2012incident} where an information security manager stated that the sharing or rather the finding of information was one of the most challenging parts of her job. 

Several of the participants in our study pointed out information dissemination as a challenge in incident handling. Knowing how much information to share can be difficult. Too little information could give an erroneous picture of the incident which could in turn lead to wrong decisions being made. On the other hand, too much information can be overwhelming and can cause delays in decision making as information has to be structured in order to be useful. It is important to communicate the \textit{right} information to the \textit{right} people. Information about incidents can obviously be sensitive and communicating such information to people who are not supposed to receive it can have serious consequences. Additionally, providing unnecessary information can be an annoyance and could at worst be counterproductive.

One employee mentioned that they have often not been notified about changes made in the security policy, which is an example of poor information dissemination. However, we believe that employees' knowledge of details in the policy is not essential to a successful incident management as long as they are familiar with relevant procedures and are capable of performing necessary actions. Providing information to employees is important, although this information should be relevant and useful.

We believe the development and establishment of clear information dissemination procedures, that can for instance be based on incident categories, could improve information dissemination in organizations. If procedures for each predefined incident category exist and are established, it will be easier and more efficient to determine what information to share and with whom.

\subsection{Experience}
\label{sec:experience}
To be able to customize responses to specific incidents, experience is essential. This was highlighted by several of the participants in our study. Developing detailed plans for all possible scenarios is not feasible and probably not useful, even though well established plans and procedures for incident handling is obviously important. Several of the participants in our study highlighted that there could always occur incidents that no one thought of beforehand. Hence, we believe allocating resources to the development of detailed plans for all potential incidents is unproductive as this is not possible.

We believe having experienced incident handlers is key for making rapid and correct decisions in a complex and dynamic environment. One obvious way to gain experience is by handling real incidents. However, organizations cannot wait for incidents to occur to gain experience, and thus rehearsals is a necessity. By conducting rehearsals addressing various types of incidents, plans and procedures can be tested and incident handlers will gain experience at the same time. 

In our opinion, neither experience nor rehearsals are sufficiently highlighted in the standards and guidelines considering how important this is for incident management. The organizations in our study focus on these two factors in their incident management. However, we believe they could benefit from conducting rehearsals more often.

Our impression is that having competent and experienced incident handlers that are both familiar with existing procedures and are capable of handling unexpected scenarios is essential to a successful incident management.

\subsection{Responsibility Allocation}
It can be challenging to know exactly where an incident originated. Hence, knowing what to do and who is responsible for handling the incident is difficult. One of the interviewees said that the greatest challenge with incident handling is cases where no one understands that they ``own" the incident and thus no one takes responsibility. This ambiguity of who owns an incident can be due to uncertainty of where it originated. This challenge was also mentioned by another of the interviewees who stated that minor incidents can escalate and have serious consequences if no one takes responsibility. Further, ambiguous responsibilities in combination with costs of handling an incident might lead to a delay in the incident response if no one claims ownership and takes responsibility. 

Even though developing detailed plans for all possible scenarios is not feasible, we still believe that having an appropriate detail level in plans addressing \textit{responsibilities} could be beneficial. As it in some situations is difficult to determine who owns an incident, our best recommendation is to improve communication to better be able to determine ownership and responsibilities in situations where this cannot be determined based on a predefined plan.

We believe that rehearsals can contribute to revealing grey areas regarding responsibilities. Additionally, rehearsals can make incident handlers more suited to determine where incidents originated. As organization's incident management procedures mature, the organizations become better equipped to determine responsibilities. The supply chain manager in Organization B emphasized that after years of working closely together, their experience and tacit knowledge help them determine who is responsible for handling an incident without specific responsibilities being determined or documented beforehand. 

The challenge of determining responsibilities is extra evident in case B, as several suppliers are involved in their incident management. In this specific case, the two suppliers have separate main responsibilities. However, we assume that grey areas may emerge with regard to responsibilities even for this case if new or unexpected incidents occur. NorSIS' guideline emphasizes that responsibilities should be determined in an SLA when (parts of) incident management is outsourced. The standards and guidelines presented in section \ref{section:standardsandguidelines} do not provide specific recommendations for resolving ambiguities with regard to incident ownership and thus responsibilities for specific incidents. We recommend organizations to comply with the NorSIS Guideline for Incident Management, i.e. to determine responsibilities in the \ac{SLA}.


\subsection{Employee Involvement}
When we contacted people for the employee survey we observed an interesting attitude among employees. Several employees seemed reluctant to participate due to their perception of their own lack of knowledge about information security. This was evident in comments such as:

\begin{quote}
\textit{``I don't know if I can help, I don't know anything about information security."}
\end{quote}

We suspect that some of the reluctance was due to employees being scared that the survey would ``reveal" their insufficient knowledge about information security. They seemed somewhat embarrassed about this insufficient knowledge and several said that they should probably have been more familiar with the organization's information security policy. There were however, some employees that admitted lack of knowledge and ``excused" this by saying that information security did not concern them. We find this very alarming as information security concerns \textbf{everyone} and as attacks taking advantage of employees, such as targeted malicious e-mails, is an increasing trend. As the example provided by the IT security manager in case A shows, regular employees' accounts can be hacked and used to send phishing e-mails. This highlights that employees do not necessarily need to have direct access to sensitive information to be exposed to attacks.  

Findings from Organization A showed that their information classification is not satisfactory. Employees seem to fail in recognizing that the value and sensitivity of the information they process should determine how information should be secured and handled. Failing to classify information can lead to the information not being sufficiently secured according to its value. This could lead to a gap between the sensitivity and value of information and implemented security measures, something that was also highlighted in a recent survey \cite{Morketall2012}.

Employees in our survey seem to have an overall positive attitude towards awareness campaigns. Many of them stated that they wished such campaigns would be conducted more often. We can imagine that, as long as the campaigns are not too extensive, this attitude is consistent throughout organizations. Due to this positivity we also believe that employees can benefit from being more involved in rehearsals. Our findings did not show any employee involvement in rehearsals beyond the involvement of incident and crisis handlers. We believe that if employees are trained in reporting procedures and incident detection they can be utilized as part of the sensor network in a larger degree than they seem to be today. 

\section{Recommendations}
\label{sec:rec}
This section presents our recommendations for performing a successful incident management. These are based on both challenges and successful practices found evident in the organizations studied. We believe these recommendations can be useful for various types of organizations.

\begin{enumerate}
\item Use well established standards or guidelines as a basis for incident management, as these are based on years of experience. 
\item Perform rehearsals to gain experience, as experience has shown to be just as important as having established plans.
\begin{enumerate}[a)]
\item Perform rehearsals both for large and small incidents. Remember that a small incident that is not sufficiently handled could escalate and lead to more serious consequences than necessary. Additionally, both small and large incidents can be valuable for learning.
\item Focus on challenging areas such as information dissemination, communication and allocation of responsibilities in rehearsals.
\item Perform rehearsals for regular employees, in addition to incident handlers, as all employees have an information security responsibility. Recommended topics for such rehearsals are information classification, incident detection (such as malicious e-mails) and reporting procedures. 
\end{enumerate}
\item Share experiences with trusted parties and communities to become better prepared to handle incidents in the future. This way organizations can utilize other organizations' experiences in addition to their own.
\item Develop clear and sound plans for communication.
\item Focus on establishing a security-positive organizational culture, where employees do not hesitate to report security events.
\item Utilize employees as part of the sensor network. Make sure that developed reporting routines are actually \textit{established}.
\item Conduct awareness campaigns with a reasonable regularity, each being of a reasonable length.
\begin{enumerate}[a)]
\item Send awareness campaigns by e-mail to make them easily accessible. A tip is to send them such that they are in the employees' inboxes when they arrive at work in the morning. We believe that people are extra susceptible to campaigns at that time, as they have not started other activities yet and will thus not be disturbed.
\item Focus on making sure that employees are aware that information security \textit{does} concern them, such that they can get familiar with their responsibilities.
\item Make employees aware of security limitations in the systems they use, such that sensitive information is not unnecessary exposed. Provide examples of how information can be lost or compromised.
\item Focus on improving employees' assessment of the value and sensitivity of information they process such that appropriate security measures can be implemented to make sure that the information they process is properly secured. 
\item Focus on making sure employees are attentive to malicious targeted e-mails as well as teaching them to recognize such e-mails.
\item Use incidents caused by employees or incidents that were/could have been detected by employees, as examples in awareness-raising activities. Incidents experienced by others can also be used as examples. Further, we recommend using incidents discussed in the media as many employees will be familiar with these. 
\item If the organization does not have the resources to create awareness campaigns themselves, it is possible to buy these from external providers, as successfully done by Organization A and B.
\end{enumerate}
\end{enumerate}

Table \ref{tab:challengesAndMeasures} shows the relationship between the challenges we observed and our recommended measures for mitigating these challenges. Our first recommendation involves using well established standards for incident management. As can be seen in the table, we have only listed recommendation 1 as a secondary measure to one of the observed challenges. This might indicate that the standards and guidelines do not focus sufficiently on these challenging factors. We believe there might be a connection between the challenges in organizations' incident management and that the standards and guidelines do not focus on these areas. Hence, basing incident management on standards alone are not satisfactory.

\begin{table}[H]
\begin{center}
\begin{tabular}{| l | l | l |}
\hline
  \textbf{Challenges} & \textbf{Main measures} & \textbf{Secondary measures} \\
  \hline
  Communication & 2b, 3 & 1 \\
  \hline
  Information & 2b & 5 \\
  \hline
  Experience & 2 & 4, 5, 6 \\
  \hline
  Responsibility & 2, 6b & 6c, 6d \\
  \hline
  Employee Involvement & 2c, 4, 5, 6 &  \\
\hline
\end{tabular}
\label{tab:challengesAndMeasures}
\caption{Links Between Observed Challenges and Proposed Measures}
\end{center}
\end{table}

