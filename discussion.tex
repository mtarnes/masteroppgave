\chapter{Discussion}
\label{chp:discussion}
In this chapter the findings from chapter \ref{chp:findings} are discussed and links between research questions and findings are established in sections \ref{sec:discussionCaseA}, \ref{sec:discussionCaseB} and \ref{sec:discussionCaseC}. The research questions are presented in section \ref{sec:objectives}. Further, underlying structures of experiences  are discussed in section \ref{sec:discussionStructures}. These findings are not necessarily  directly linked to the research questions, but emerged from the data. This analysis method is based on the general inductive research approach as described in \ref{sec:qualitativeAnalysis}.  

\section{Case A}
\label{sec:discussionCaseA}
Organization A has general guidelines for incident handling, but does not have one plan that applies to all possible incidents. However, they do have detailed guidelines for the most frequent incident types. In addition to a central contingency plan, they have developed contingency plans for each individual department. Organization A is compliant with ISO/IEC 27035's recommendation of having a policy for incident handling.

SANS and the \acs{ITIL} framework state that having an information security policy is important. Further, the \acs{ITIL} framework emphasizes that employees should have access to and be aware of this policy. Organization A has developed an information security policy. It may, however, not be well enough established throughout the organization. This is supported by the IT security manager's suspicion that not all users have detailed knowledge of the policy. It is further supported by the employee survey where few claimed to be familiar with the policy.

An abuse system and a deviation management system are used for reporting and documenting incidents and vulnerabilities. Having reporting and documentation systems and procedures is recommended by all guidelines presented in section \ref{section:standardsandguidelines}.   The implementation guidance in ISO/IEC 27002 suggests that those reporting information security events should be notified of results after the reported issue has been dealt with. Organization A is compliant with this recommendation as they provide feedback to those reporting security issues. 

The organization has implemented monitoring systems such as IDSs, which is recommended by most relevant standards and guidelines. In addition to technical detection mechanisms, users can be a great source of detecting incidents, and therefore the organization should have available reporting systems. This was highlighted in a presentation at Sikkerhet \& S\aa rbarhet 2013:

\begin{newquote}{Vidar Sandland and Hans Marius Tessem, NorSIS}
``Consider employees as part of the organization's sensor network."
\end{newquote}

This is especially important with regards to social engineering and targeted attacks, which are increasing. Their abuse and deviation management systems facilitate users' reporting of incidents. One interesting observation is that employees seem to lack knowledge and qualifications to be able to recognize incidents, which might indicate that they are not fully utilized as resources for incident detection. We found that employees are unaware that they are required to report incidents, unaware of how to report and under which circumstances reporting is necessary. The majority of the participants in the employee survey said they did not report suspicious e-mails. It should be noted that only a small number of employees are represented in this study and they had mainly received spam and not e-mails targeted specifically at the organization. Among the 15 participants, only one claimed to know which and to whom incidents should be reported. Even though the majority of the participants thought that they would be able to figure out whether incidents should be reported, this finding is alarming, especially if it is representative for the entire group of users. 

According to SANS, NorSIS, the ITIL framework and ISO/IEC, incident prioritization rules should be based on an organizational impact analysis and be part of the incident management preparation phase. One way to evaluate potential organizational impact caused by incidents is to conduct a risk assessment. The organization conducts risk assessments regularly, leading to a categorization scheme which provides the basis for their prioritization. Another important part of the preparation phase highlighted in the standards and guidelines is the establishment of an \ac{IRT}. Organization A does not have their own \ac{IRT}, but has dedicated personnel for incident handling and a crisis team to handle the most severe incidents.

It is stated in the organization's information security policy that information shall be classified by the information owners. This is compliant with ISO/IEC 27002. The fact that so few of the employees were familiar with the information security policy could indicate that they are not aware of their responsibility for classifying information. This is supported by the IT security manager who stated that their information classification is not satisfactory. He believes that employees are not aware of this specific policy requirement. ISO/IEC 27002 emphasizes that classification of information is important to ensure proper protection of information and to identify the need for special handling measures. This could indicate that the organization's information is not sufficiently protected with regards to its sensitivity and value. The organization processes large amounts of sensitive data, and this finding is thus alarming.

The IT security manager stated that it is the management's responsibility to make sure the policy is well established throughout the organization. There may be a lack of management commitment to information security as the policy does not seem to be well established within the organization. This management commitment is highlighted as critical by ISO/IEC 27001. 

The IT security manager emphasized that it is important that users understand the security limitations in the systems they use. The tools employees use for communication, sharing and storing of information might not be as secure as they assume. Several employees mentioned that information security did not concern them. They believed it was not relevant to their work and that they were not exposed to attacks or incidents, despite having access to information and performing their work on a computer. Even though most of the employees in the survey did not know what an information security incident is, they still claimed to be attentive to incidents in their everyday work. These contradictory statements might indicate that information security is not well understood and that employees have an erroneous picture of their own security knowledge and awareness. Organizational culture, motivation, risk awareness and attitude could be contributing factors to this contradiction. Organizational culture and attitude might influence employees to attempt to appear knowledgeable and construct answers thereafter. These findings might indicate a lack of risk awareness among employees and is supported by the IT security manager who said that one issue revealed through rehearsals was employees' understanding of risk. As mentioned in \ref{sec:threatLandscape}, vulnerabilities exist mainly due to lack of employees' understanding of risk and this is therefore an alarming finding. By raising employees' awareness vulnerabilities and thus incidents can be mitigated.

With regards to rehearsals, the organization complies with recommendations in ISO/IEC 27035 and the NorSIS guideline. The organization conducts contingency rehearsals for the crisis team. It is important for the crisis team to gain experience to become better prepared to handle real situations. One interesting observation is that their rehearsals include situations where they know they lack routines. This gives the team an opportunity to train on improvising in situations where there are no predefined plans. Routines developed through such a bottom-up approach might be better established within the team than routines imposed by others, due to the team's participation in the development and implementation.

Employees' awareness and participation in incident management procedures are important according to ISO/IEC 27035. The employee survey indicates that employees are positive to awareness campaigns and found previous campaigns useful. Several employees would like such campaigns to be conducted more often. Even those claiming that most of the content in these campaigns was known, found them useful as reminders. The positive attitude towards learning more about information security, in addition to the wish of having awareness campaigns more often, might indicate that it is a lack of management commitment that is the reason for insufficient understanding and awareness of information security and not employees' attitudes. If that is the case it would be unfortunate as senior management commitment to incident management is highlighted as important in ISO/IEC 27035.

Organization A has individual contingency plans for each department which are initiated when serious incidents occur. If an incident has a particular high impact level or involves several departments, their central contingency plan is initiated. Having an escalation procedure is compliant with ISO/IEC 27035. This standard also specifies that it should be a main activity for the \ac{IRT} to allocate responsibilities. The allocation of responsibilities is performed in the organization by delegating parts of the incident handling to specific sections that have expertise relevant for solving the incident.

The main objectives of the incident management process in ITIL are to restore service as quickly as possible and to limit adverse business impact. Guidelines such as NIST SP 800-61 and ENISA's Good Practice Guide for Incident Management also specifies that recovery of affected systems is important.  The IT security manager however, does not see restoring services as quickly as possible as the top priority as long as the incident is not fully resolved. He is more concerned with making sure that the incident is properly resolved than to rush the restoring of affected systems. It should be noted that the guidelines also emphasize the importance of fully resolving incidents. In ITIL this is part of the problem management process that focuses on root cause analysis. The IT security manager would have liked to perform more measures to ensure that the root cause of incidents is identified and eradicated, but is often not allocated the resourced needed. This may be another indication of a lack of management commitment to information security in the organization.

Organization A is responsible for its own IT operations and does therefore not need to coordinate with external parties during incident handling. Nevertheless, the organizational structure has shown to be a challenge with regards to communication during incident handling. The IT security manager specifically mentioned that the designated contact person changes on a daily basis for some sections. This represents the challenge for the IT security manager who experiences uncertainty with regards to the designated contact person's knowledge and experience which may vary.

Organization A's lessons learned phase is compliant with the recommendations in the majority of the standards and guidelines. They perform reviews of severe incidents to identify root causes and improvements. Further, these improvements are implemented and in specific cases shared with trusted communities and partners. This is recommended in ISO/IEC 27035. 

Organization A has not implemented any specific standard or guideline for incident management, but has based its approach on components from ISO/IEC 27001 and 27002 as well as the \ac{ITIL} framework. Nevertheless, they seem to comply reasonably well with the recommendations in the standards and guidelines presented in this report. They have developed several plans and procedures addressing information security and incident management specifically, but not all of these seem to be well \textit{established} throughout the organization. Unexpected situations where sufficient plans do not exist may occur. Additionally, they do not always have the required staff to respond efficiently. The overall impression is that incidents generally seem to be handled in accordance with their predetermined plans.


\section{Case B}
\label{sec:discussionCaseB}
The organization has developed an information security policy, with intention to define senior management's position with regards to IT security. This might indicate some level of management commitment. Having a security policy is stated to be important by SANS and the ITIL framework, whereas management commitment is highlighted as essential in the ISO/IEC 27000 family of standards. Further, the ITIL framework recommend that employees should have access to and are aware of the information security policy. The organization seems to be compliant with this recommendation as most employees answered that they were to some extent familiar with the organization's information security policy. 

The interviewees provided slightly different definitions of an information security incident. One of the differences in their definitions was that the interviewees from Organization B specified a difference between security breaches, i.e. incidents caused intentionally by employees, and other incidents. The other two interviewees did not specify this difference. This makes sense as the two external suppliers are mainly concerned with the affected systems, whereas it is the organization itself that handles disloyal personnel. They all agreed on what was the worst possible incident Organization B could experience.

The organization has procedures for handling security vulnerabilities. Additionally, Supplier 1 is responsible for securing the organization's network. These are important measures to prevent the occurrence of incidents. Prevention of incidents is stated to be fundamental to the success of an organization's incident response by NIST SP 800-61. 

The organization has routines for incident handling, but the IT manager stated that constructing a holistic plan is challenging. Even though they lack a holistic plan, they do however have contingency plans. Specific contingency plans have been developed in cooperation with Supplier 1 and Supplier 2 for use in crisis situations. For incidents that escalate to a level where they are categorized as crises, the contingency plans are initiated. These plans include guidelines for communication and escalation, which is compliant with recommendations in SANS, NorSIS and ISO/IEC. In NorSIS's guide for incident management it is emphasized that information security should be considered when \acp{SLA} are developed for outsourcing. Supplier 2's plan is developed to ensure fulfilment of the \ac{SLA}. 

Allocating resources to developing detailed plans is not the organization's main focus, as they believe having experienced incident handlers that understand the situation and make decisions thereafter are more important for a successful incident handling than having a detailed plan. This is an interesting observation as standards and guidelines tend to focus on plans and procedures rather than experienced incident handlers even though they do recommend training. 

The supply chain manager highlighted communication as one of the most challenging parts of incident management. It can be challenging to know what information to communicate, when to communicate it and to whom. Making wrong decisions with regards to information could cause delays in the incident handling and may result in serious consequences. Another challenge mentioned by the interviewee from Supplier 2 was the handling of various sources of information. The organization is well aware of these challenges and thus focuses on them in their rehearsals. An example is that they conduct rehearsals in cooperation with several suppliers. The finding that information dissemination is challenging supports the findings discussed in section \ref{relatedwork} where participants in a case study meant that better information dissemination would improve their security procedures which in turn would improve the overall security of their organization. 

A related challenge is knowing who has the responsibility in various situations. The supply chain manager stated this to be especially challenging for minor incidents, because responsibilities are not specified in detail. The interviewee from Supplier 2 also mentioned that not knowing who is responsible for the incident is challenging. He mentioned cases where the owner of an error does not understand that he owns it or does not acknowledges ownership due to costs. Some incidents might be so complex that knowing exactly where they originated, and thus determining who is responsible for handling it is difficult. Nevertheless, contacting the right resources was highlighted as one thing that has worked well in Supplier 2's incident handling for Organization B. The challenge of determining who is responsible in various cases could perhaps be mitigated by improving communication and defining responsibilities more clearly beforehand.

The organization finds it important to scale the incident handling team based on the severity of the specific incident. Based on the situation, the team may include people from both the organization and the suppliers involved in the incident handling, however they try to limit the number of people that has authority to make decisions. Consequently, incidents have to be assessed and decisions of who to involve and inform have to be taken for each individual incident. This could make communication, responsibility allocation and information dissemination even more challenging. 

Supplier 1 keeps track of trends related to security incidents, by monitoring their own internal systems. This is compliant with recommendations for the preparation phase in ISO/IEC 27035 although it should preferably be formalised in an incident management policy. 

The organization is familiar with the ISO/IEC standards, uses the ITIL framework as a guideline for their processes, but focuses on internal documents for incident management. To test their plans and gain experience, the organization performs regular rehearsals. This is compliant with ISO/IEC 27035 and NorSIS's guideline. The organization has seen benefits of conducting rehearsals in cooperation with their suppliers as areas of improvement have been revealed and measures implemented.   

To disseminate information about security best practice to employees, the organization conducts awareness campaigns addressing various topics. One employee mentioned that the security measures proposed in awareness campaigns were too strict and socially unacceptable. Another said he would rather take the risk of a security incident than to comply with the proposed best practice. These statements emphasize the importance of having an appropriate balance between security and usability. This was also discussed in one of the presentations at Sikkerhet \& S\aa rbarhet 2013:

\begin{newquote}{John Arild Amdahl Johansen, Buypass AS}
``Security must \textbf{never} stop business."
\end{newquote}

In the presentation it was stated that if security measures are too complex, users will find a way to circumvent the rules and thus the initial security measures are compromised. However, it should be noted that most employees in the survey found awareness campaigns useful. The two employees who did not find campaigns useful had an IT background which might indicate that employees' impression of campaigns varied with individual background and IT knowledge. 

The organization uses monitoring systems as well as employees as sources for incident detection. This is in accordance with recommendations from most of the standards and guidelines presented in section \ref{section:standardsandguidelines}. However, the employee survey indicated that the reporting routines for employees are not satisfactory. Most of the employees were not sure where to report incidents and only one knew that they are supposed to report incidents to Supplier 1. Many said they would have reported to the local or central IT manager. Their overall uncertainty related to reporting may indicate that reporting routines are not well enough established throughout the organization. Additionally, a few stated that they were not familiar with reporting routines since they had never needed to report anything. This finding is similar to the one in case A where some employees believed information security did not concern them. 

The organization uses a predefined classification scale for categorizing incidents. This is compliant with ISO/IEC 27035 and ENISA's good practice guideline. Incidents are categorized based on impact level. The interviewee from Supplier 2 mentioned that the categorization is further used to prioritize incidents. An incident prioritization based on impact is recommended by ISO/IEC, SANS, NorSIS and the ITIL framework.

All incidents are logged and the root cause of the incident is included in the log. Most of the standards and guidelines discussed in section \ref{section:standardsandguidelines} specifies logging as being important. Additionally, the ITIL framework focuses on root cause analysis in the problem management process.

The organization conducts regular meetings where they discuss serious incidents. They perform trend analysis by looking at incident reports provided by their two main suppliers. These activities are described as essential in ISO/IEC 27035. Additionally they have debriefing meetings after serious incidents. Overall, the organization's lessons learned activities seem to be in accordance with relevant standards and guidelines.

As described in the standards and guidelines, recovery is an important part of incident response. Two related processes in the ITIL framework addressing this are the availability management process and the IT service continuity management process. The organization has tried to ensure a high level of redundancy, which might make this process easier and more efficient. Additionally, it might limit availability related consequences of security incidents. As part of the same phase in the NIST guideline is gathering and handling of electronic evidence. The organization has routines for preservation of electronic evidence, which is also compliant with ISO/IEC 27002.

Even though not all employees knew what an information security incident is or could provide a definition, most of them gave relevant examples. This might indicate a reasonably sound understanding of information security in the organization. However, as in case A most employees stated to be attentive to incidents, even though they also said they did not know what an incident is. This shows that there is still room of improvements related to employees' information security understanding and awareness.

As described in section \ref{relatedwork} it is recommended that incident handlers should have knowledge about the organization's IT systems and services in order to better be able to recognize abnormal behaviour. This type of knowledge is often tacit and its importance is often not recognized. The organization may have difficulties utilizing such knowledge as incident handling is to a large degree outsourced. However, the handlers at Supplier 1 and Supplier 2 may have gained such knowledge as they handle the organization's daily IT operations and application management respectively. %Paper: \cite{werlinger2010preparation}

The organization has not strictly based their plans and procedures on standards or guidelines for incident management, however they are relatively compliant with the standards and guidelines presented in section \ref{section:standardsandguidelines}. Like in case A, it seems like reporting routines are not sufficiently established in the organization as employees showed uncertainty related to their procedures. They have a set of predefined plans, but the importance of having experienced incident handlers becomes extra evident in this case as their incident handling is distributed and scalable. Overall, the impression is that incident handling has been in accordance with predefined plans. However, their plans are quite general and they thus also focus on improvising, i.e. making situation-specific decisions. 

\section{Case C}
\label{sec:discussionCaseC}
The organization has an information security policy which is well known among the participants in the employee survey as they all stated that they were familiar with the policy. In one of the presentations at Sikkerhet \& S\aa rbarhet 2013, Difi\footnote{The Norwegian Agency for Public Management and eGovernment} recommended discussing security during employee appraisals. The IT security manager said that their security handbook is always a topic in the annual employee appraisals. Most of the employees in the survey seemed to have a good understanding of what an information security incident is. This could indicate that information security is somewhat better understood among employees in this organization than in case A and B. There are several findings that could explain this observation. It can be assumed that the annual review of the security handbook ensures that employees are familiar with their security responsibilities. Further, the organization has an overall focus on employees' importance in information security work. As they state in their policy, they believe it is important to have a security-positive environment in the organization. This focus on employees could have increased their security understanding. The organization's core activity is delivery of IT services. Consequently, they have a high focus on information security.

The ISO/IEC 27000 family of standards emphasizes the importance of management commitment both to incident management and information security in general. SANS and NorSIS also mention this in their guidelines. The organization seems to focus on management commitment as the aim of their information security policy is to communicate the management's direction and commitment to information security. It thus seems like the organization in case C has a larger degree of management commitment than the organizations in case A and B. One reason for this might be that they have several ISO/IEC 27001 certifications, as this standard states that management shall provide evidence of its commitment to information security.

Although the organization has several ISO/IEC 27001 certifications, they do not use ISO/IEC 27035 or other standards specific to incident management. They have chosen to base their entire service management, including incident management, on the ITIL framework. The organization has various incident handling plans aimed at handling specific incident types. The organization has very detailed descriptions of their general incident management process. They have specified a separate workflow for major incidents. This is in accordance with recommended practice in the ITIL framework. 
%More discussion?
 
The fact that all of the employees had attended courses or other awareness raising activities, indicates that the organization has a high focus on improving employees' security knowledge and awareness. This observation is supported by statements from the IT security manager, and may also confirm that the organization follows through on their policy objectives. Additionally, employees' attitude towards awareness raising activities shows signs of a security-positive environment.  
 
The employee survey showed some uncertainty with regards to reporting routines in this organization as it did in case A and B. The few employees that did not know where to report claimed to know where to find relevant information. Employees' knowledge of where to find relevant information is positive, but this might not be efficient enough in all situations as it introduces an extra delay. The IT security manager said that employees are advised to report incidents to the \ac{IRT}, however none of the employees mentioned this. Suspicious e-mails was one of the given examples of cases that should be reported. Still, none of the employees had previously reported such e-mails. It should be noted that only two of the employees claimed to have received such e-mails. 

The fact that no one had reported incidents could indicate that employees are not fully utilized as part of the organization's sensor network for detecting incidents. This assumption is supported by one of the employees who stated that they should probably report events more often. Further, this is supported by the interviewee who suspects underreporting. He believes the threshold for reporting is high. This is unfortunate as they try to focus on a security-positive environment. Employees' lack of reporting might indicate that they still have some work to do with regards to the organizational security culture. Nevertheless, the routines for reporting seem to be somewhat better established in this organization than in the other two. 

The organization has a risk framework for reporting vulnerabilities that are subsequently handled. This way incidents can be prevented. The prevention of incidents is an important part of incident management and is stated to be a fundamental factor by NIST. Further, NorSIS specifies preventive measures to be one of the most cost effective ways to perform incident management. 

Incidents are categorized based on their impact level. They base their categorization method on the ITIL framework, which stresses the importance of incident categorization. The organization can analyse categorization information to improve exposed infrastructure and thus decrease the number of incidents. The categorization determines which incident response procedures to initiate, which is compliant with recommendations from the majority of the standards and guidelines discussed in section \ref{section:standardsandguidelines}.

The organization has monitoring systems for incident detection. ISO/IEC 27035 states that organizations should have monitoring systems to aid detection. NIST mentiones monitoring systems as a tool for incident detection and ENISA recommends the use of such systems in addition to reporting. 

The main purposes of their incident management process are in line with the ITIL framework. This also applies to their execution of the process activities where they use incident handlers as well as an \ac{IRT} for assistance and expertise. In addition to the IRT, the organization has a \ac{CIM} team that provides complementary competence. As illustrated in figure \ref{fig:workflowcaseC}, the service desk function is the first line of incident response. This is compliant with recommendations in the ITIL framework. If the service desk is not able to handle the incident internally, specialists are contacted. Further, if the incident is assessed as major, the major incident process is initiated and additional resources must be engaged. In cases where extra dedicated resources are needed, a Task Force is established. This shows that the organization has mature and well established escalation routines.

It is evident in the organization's internal documentation that they believe successful incident management is based on having contingency plans and predefined tasks. The employee we had e-mail correspondence with acknowledged that it would be ideal to have plans and procedures for all possible incidents, but that this might not be practically feasible. He emphasized that incident handlers who compose a set of activities to customize the incident response for specific incidents are key to a successful incident handling. Due to variations in incidents, an experienced incident handler is more important than rigid process adherence.

The organization has routines for handling electronic evidence. They perform duplication of hard disk drives if necessary, but further investigation is conducted by a third party. The existence of routines for handling electronic evidence is compliant with NIST SP 800-61 and ISO/IEC 27002.

Two of the organization's own requirements for their incident management process address incident documentation. The requirements state that all incidents should be registered and all actions recorded. In addition to incidents, experiences and potential improvements are documented. As most of the standards and guidelines discussed in \ref{section:standardsandguidelines} emphasize the importance of logging, it is fair to say that the organization follows best practise.

They conduct rehearsals and occasionally customers or the government participate. Lessons learned activities are conducted after major incidents and rehearsals where they try to identify possible improvements. In accordance with the ITIL framework they initiate the problem management process after major incidents. According to the IT security manager these activities are useful as they often result in concrete measure. The fact that the organization has initiated a project to improve their incident management shows that they are committed to improve their incident management.

Incident handling is distributed among the organization and its customers. Thus, the challenges of collaboration and coordination are evident in this case. As in the other organizations, communication emerges as a challenge. After rehearsals, it has often been recognized that interaction with external parties can be improved. The establishment of more specific communication routines as well as well defined responsibilities might mitigate these challenges.

The organization seems to have a set of well established plans and procedures as well as experienced incident handlers. Additionally, they have several ISO/IEC 27001 certifications and has implemented the ITIL framework for their IT service management. Their incident management is highly compliant with the ITIL framework as well as being relatively compliant with the other standards and guidelines presented in section \ref{section:standardsandguidelines}. The uncertainty among employees with regards to reporting routines might indicate that these routines are not sufficiently established throughout the organization. Nevertheless, our findings indicate that this organization has an overall mature incident management. 

\section{Underlying structures of experiences}
\label{sec:discussionStructures}
\subsection{Communication}
\begin{itemize}
\item \textit{Communication}
\begin{itemize}
\item \textit{\textbf{Who} should be contacted \textbf{when}?}
\item \textit{Who should be contacted when the actual contact person cannot be reached?}
\end{itemize}
\end{itemize}

\subsection{Information}
\textit{Closely linked with communication}

\begin{itemize}
\item \textit{Information}
\begin{itemize}
\item \textit{What information needs to be collected?}
\item \textit{What information should be \textbf{communicated} to \textbf{whom}?}
\item \textit{Who should collect it?}
\item \textit{How can it be collected?}
\end{itemize}
\end{itemize}

\textit{As found in case B, under lessons learned:
The interviewee from Supplier 2 mentioned that that handling and collecting information is a challenge. As an employee talks about information dissemination problems in \cite{ahmad2012incident}.
``it's the sharing, or rather finding of information. The information is there, each day I find a new resource that's got great information."}

\subsection{Experience}

Even though predefined plans exist to handle a specific incident, following the plan might not always be the best solution. If the person dealing with an incidents has long experience, he might take actions that are not according to the plan, but that is the best way to solve the incident. The incident handler's experience is just as important to effective and efficient incident management/response as having plans and procedures. Link this to RQ3.

``By conducting a causal analysis, new knowledge is created which through generalization can be used to facilitate organizational learning."\cite{werlinger2010preparation}
.

\begin{itemize}
\item \textit{Situational Awareness}
\begin{itemize}
\item\textit{ Necessary to make fast and correct decisions in a complex and dynamic environment. It is challenging to know exactly where the incident ``hit" if one does not have situational awareness, thus knowing what to do and who's responsible. This was  seen as a challenge in case B as well as the main background for starting the major ``lessons learned" project in case C.}
\end{itemize}
\end{itemize}

\textit{Another finding is that even though procedures are important, experience may be just as important, as it is impossible to plan detailed for anything, and often one must make decisions on the spot.}

\subsection{Responsibilities}
\textit{Also closely linked with communication. Better routines for communication may make responsibilities clearer?}

\begin{itemize}
\item \textit{Responsibilities}
\begin{itemize}
\item \textit{Who is responsible for the incident?}
\item \textit{Who is responsible for \textbf{fixing} it?}
\item \textit{Who is responsible for covering the costs?}
\end{itemize}
\end{itemize}

\textit{These challenges (especially communication and responsibility) are extra evident in Case B, where there are several suppliers and many people that need to cooperate, but it is also seen in Case A, where there is only one organization that handles its own IT operations involved.}

Another aspect is that users may not be aware that part of the responsibility lies with them. For example users that are owners of information may have a certain responsibility for protecting this information. An example is in case A, where users are responsible for classifying the information they process, but the interviewee thought that many are not aware of this (even though it is stated in the information security policy).

\textbf{Recommendations}
\begin{itemize}
\item Perform rehearsals to gain experience since experience has shown to be just as important as having established plans.
\item Have good plans for communication, since this seems to be challenging.
\item Utilize employees as part of the sensor network, does not seem to be done in these organizations. Better establish reporting routines!
\end{itemize}

\textit{It seems that loss of sensitive information/information important to the organizations' core activities is seen as the worst security incidents. This includes customer-specific information, where disclosure additionally could cause reputational damage.}

\textit{The gap between implemented security measures and the sensitivity and value of information is increasing according to \cite{Morketall2012}. Is this also evident in our findings? Could be in case A}

\textit{Employees should be regarded as part of the sensor network. Not done very well in the organizations, not well established in the organizations.}