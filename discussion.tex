\chapter{Discussion}
\label{chp:discussion}
In this chapter the findings from chapter \ref{chp:findings} are discussed and links between research questions and findings are established. The research questions are presented in section \ref{sec:objectives}. Further, underlying structures of experiences that emerged from the data, but not necessarily directly linked to the research questions, are discussed. This is in compliance with the general inductive research approach as described in \ref{sec:qualitativeAnalysis}.  

\textit{What about management commitment? Is that present in the organizations? Very important according to ISO}

\textit{The gap between implemented security measures and the sensitivity and value of information is increasing according to \cite{Morketall2012}. Is this also evident in our findings? Could be in case A}

\textit{Several standards/guidelines (ISO ENISA) say that you should have monitoring systems in addition to people reporting. As if people are primary and systems secondary. Organizations we talked to seem to see systems as primary and people as secondary}

\section{Case A}
The organization has general guidelines for incident handling, but they do not have one plan that encompasses all possible incidents. They have detailed guidelines for the most frequent incident types. They also have contingency plans for all departments in addition to a central contingency plan. They have a policy for incident handling which is compliant with ISO/IEC 27035. 

The existence of an information security policy is stated to be important by SANS and the \acs{ITIL} framework and is something employees should have access to and be aware of. The organization has an information security policy. It may however not be well enough established throughout the organization. This is both consistent with what the IT security manager said and what was revealed through the employee survey.

The organization has two systems that they use for reporting and documenting incidents and vulnerabilities, an abuse system and a deviation management system. The existence of reporting and documentation systems as well as procedures is recommended by all guidelines presented in section \ref{section:standardsandguidelines}. Additionally, they provide feedback to those reporting security issues. The implementation guidance in ISO/IEC 27002 suggests that those reporting information security events should be notified of results after the reported issue has been dealt with. 

The organization has implemented the recommendation of having monitoring systems such as IDSs. In addition to these technical detection mechanisms, users can be a great source of detecting incidents, and therefore the organization should have available reporting systems. This is especially important with regards to social engineering and targeted attacks, which are increasing. Their abuse and deviation management systems facilitate users' reporting of incidents. 

\begin{newquote}{``Live hacking session", Sikkerhet \& S\aa rbarhet 2013}
``Consider employees as part of the organization's sensor network."
\end{newquote}

Employees seem to lack knowledge and qualifications to be able to recognize incidents and this may indicate that they are not fully utilized as resources in incident detection. Challenges that are evident in the organization are employees actually knowing that they are required to report incidents, how to report and under which circumstances reporting is necessary. The majority of the participants in the employee survey said they did not report suspicious e-mails. It should be noted that only a small number of employees are represented in this study and they had mainly received spam and not e-mails targeted specifically to the organization. Among the 15 participants, only one claimed to know which and to whom cases should be reported. Even though the majority of the participants thought that they would be able to figure out whether incidents should be reported, this finding is alarming, especially if it is representative for the entire group of users. 

According to SANS, NorSIS, the ITIL framework and ISO/IEC, incident prioritization rules should be based on an organizational impact analysis and be part of the incident management preparation phase. One way to evaluate potential organizational impact caused by incidents is to conduct a risk assessment. The organization conducts risk assessments regularly, leading to a categorization scheme which provides the basis for the prioritization. Another important part of the preparation phase highlighted in the standards and guidelines is the establishment of an \ac{IRT}. The organization does not have an \ac{IRT}, but has dedicated personnel for incident handling and a crisis team to handle the most severe incidents.

It is stated in the organization's information security policy that information shall be classified by the information owners. This is compliant with ISO/IEC 27002. The fact that so few of the employees were familiar with the information security policy could indicate that they are not aware of their responsibility for classifying information. This is supported by the IT security manager who stated that their information classification is not satisfactory. He believes that employees are not aware of this specific policy requirement. ISO/IEC 27002 emphasizes that classification of information is important to ensure proper protection of information and the need for special handling measures. This could indicate that the organization's information is not sufficiently protected with regards to its sensitivity and value. The organization processes large amounts of sensitive data, and this finding is thus alarming.

The IT security manager stated that it is the managements responsibility to make sure that the policy is well established throughout the organization and as the policy does not seem to be well established there may be a lack of management commitment to information security. This management commitment is highlighted as critical by ISO/IEC 27001. 

The IT security manager emphasized that it is important that users understand the security limitations in the systems they use. The tools employees use for communication, sharing and storing information might not be as secure as they think. Several employees mentioned that information security did not concern them. They believed it was not relevant to their work and that they were not exposed to attacks or incidents, despite having access to information and performing their work on a computer. Even though most of the employees in the survey did not know what an information security incident is, they still claimed to be attentive to incidents in their everyday work. These contradictory answers might indicate that information security is not well understood and that employees have an erroneous picture of their own security knowledge and awareness. Organizational culture, motivation, risk awareness and attitude could be contributing factors to these contradictory answers. Employees may wish to appear knowledgeable and construct answers thereafter. This can be due to both organizational culture and attitude. These findings might indicate a lack of risk awareness among employees and is supported by the IT security manager who said that one issue revealed through rehearsals was employees' understanding of risk. As mentioned in \ref{sec:threatLandscape}, vulnerabilities exist mainly due to lack of understanding of risk.

With regards to rehearsals, the organization complies with recommendations in ISO/IEC 27035 and the NorSIS guideline. The organization conducts contingency rehearsals for the crisis team. It is important for this team to gain experience with incident handling to become better prepared to handle real situations. One interesting observation is that their rehearsals also include situations where they know they lack established routines. This gives them an opportunity to train on improvising in situations where there are no predefined plans. By using this bottom-up approach for developing new routines, these routines might be better established within the team than routines imposed by others, as the team participates in the development and implementation themselves.

Awareness and participation of all personnel is important according to ISO/IEC 27035. The employee survey indicates that employees are positive to awareness campaigns, and have found previous campaigns useful. Several employees would have liked such campaigns to be conducted more often. Even those claiming that most of the content was known, found it useful as a reminder. This shows a positive attitude to learn more about information security, and might indicate that it is a lack of management commitment and not employees' attitudes that is the reason for insufficient understanding and awareness of information security. If that is the case it would be unfortunate since senior management commitment is highlighted as important in ISO/IEC 27035.

The organization has individual contingency plans for each department that is initiated in case of serious incidents. If an incident has a particular high impact level or involves several departments, their central contingency plan becomes operational. The existence of an escalation routine is compliant with ISO/IEC 27035. This standard also specifies that it should be a main activity for the \ac{IRT} to allocate responsibilities. The allocation of responsibilities is performed in the organization by delegating parts of the incident handling to specific sections with expertise relevant for solving the incident.

The main objectives of the incident management process in ITIL are to restore service as quickly as possible in addition to limit adverse business impact. Guidelines such as NIST SP 800-61 and ENISA's Good Practice Guide for Incident Management also specifies recovery of affected systems to be very important.  The IT security manager however, does not see restoring services as quickly as possible as top priority as long as the incident is not fully resolved yet. He is more concerned with making sure that the incident is properly resolved than to rush the restoring of affected systems. It should be noted that the guidelines also emphasize the importance of fully resolving incidents. In ITIL this is described as a separate process, the problem management process. The IT security manager would have liked to perform more measures to ensure that the root cause of incidents is identified and eradicated, but is often not allocated the resourced needed. This may be another indication of a lack of management commitment to information security in the organization.

The organization does not have to consult with external parties during incident handling, since they are responsible for their own IT operations. Nevertheless, the organizational structure has shown to be a challenge related to communication during incident handling. The IT security manager specifically mentioned that the designated contact person changes on a daily basis for some sections. This implies a challenge for the IT security manager who experiences uncertainty with regards to the designated contact person's knowledge and experience which may vary.

The organization is compliant with the recommendations in the majority of the standards and guidelines for the lessons learned phase. They perform reviews of severe incidents to identify root cause and improvements. Further, these improvements are implemented and in specific cases shared with trusted communities and partners. This is recommended in ISO/IEC 27035. 

The organization has not implemented any specific standard or guideline concerning incident management, but has based its approach on components from ISO/IEC 27001 and 27002 as well as the \ac{ITIL} framework. Nevertheless, they seem to comply reasonably well with the recommendations in the standards and guidelines presented in this report. They have developed several plans and procedures addressing information security and incident management specifically, but not all of these seem to be well \textit{established} throughout the organization. Unexpected situations may occur where sufficient plans do not exist. Even for cases where plans do exist, they do not always have the required staff to respond efficiently. However, incidents generally seem to be handled in accordance with their predetermined plans.


\section{Case B}
The organization has developed an information security policy, with intention to define senior management's position with regards to IT security. This might indicate some level of management commitment. Having a security policy is stated to be important by SANS and the ITIL framework, whereas management commitment is highlighted as essential in the ISO/IEC 27000 family of standards. Further, the ITIL framework recommend that employees should have access to and are aware of the information security policy. The organization seems to be compliant with this recommendation as most employees answered that they were to some extent familiar with the organization's information security policy. 

The interviewees provided slightly different definitions of an information security incident. One of the differences in their definitions was that the interviewees from Organization B specified a difference between security breaches, i.e. incidents caused intentionally by employees, and other incidents. The other two interviewees did not specify this difference. This makes sense as the two external suppliers are mainly concerned with the affected systems, whereas it is the organization itself that handles disloyal personnel. They all agreed on what was the worst possible incident Organization B could experience.

The organization has procedures for handling security vulnerabilities. Additionally, Supplier 1 is responsible for securing the organization's network. These are important measures to prevent the occurrence of incidents. Prevention of incidents is stated to be fundamental to the success of an organization's incident response by NIST SP 800-61. 

The organization has routines for incident handling, but the IT manager stated that constructing a holistic plan is challenging. Even though they lack a holistic plan, they do however have contingency plans. Specific contingency plans have been developed in cooperation with Supplier 1 and Supplier 2 for use in crisis situations. For incidents that escalate to a level where they are categorized as crises, the contingency plans are initiated. These plans include guidelines for communication and escalation, which is compliant with recommendations in SANS, NorSIS and ISO/IEC. In NorSIS's guide for incident management it is emphasized that information security should be considered when \acp{SLA} are developed for outsourcing. Supplier 2's plan is developed to ensure fulfilment of the \ac{SLA}. 

Allocating resources to developing detailed plans is not the organization's main focus, as they believe having experienced incident handlers that understand the situation and make decisions thereafter are more important for a successful incident handling than having a detailed plan. This is an interesting observation as standards and guidelines tend to focus on plans and procedures rather than experienced incident handlers even though they do recommend training. 

The supply chain manager highlighted communication as one of the most challenging parts of incident management. It can be challenging to know what information to communicate, when to communicate it and to whom. Making wrong decisions could cause delays in the incident handling and may get serious consequences. A related challenge is knowing who has the responsibility in various cases. The supply chain manager stated this to be especially challenging for minor incidents, because routines are not specified in detail. The interviewee from Supplier 2 also mentioned that it is challenging when one does not know who is responsible for the incident. In addition to cases where the owner of an error does not understand that he owns it, the case where no-one acknowledges ownership due to costs is mentioned. Some incidents are so complex that knowing exactly where it originated, and thus determining who is responsible is difficult. Nevertheless, contacting the right resources was highlighted as one thing that has worked well in Supplier 2's incident handling for Organization B. Another challenge he mentioned was handling various sources of information. The challenge of determining who is responsible in various cases could perhaps be mitigated by improving communication and defining responsibilities more clearly beforehand. The organization is well aware of these challenges and thus focus on them in their rehearsals. An example is that they conduct rehearsals in cooperation with several suppliers.

The finding that information dissemination is challenging is consistent with findings discussed in section \ref{relatedwork} where participants in a case study meant that a better information dissemination would improve the security of their organization. 

Supplier 1 keeps track of trends related to security incidents, by monitoring their own internal systems. This is compliant with recommendations for the preparation phase in ISO/IEC 27035 although it should preferably be formalised in an incident management policy. 

The organization is familiar with the ISO/IEC standards, uses the ITIL framework as a guideline for implementation, but focuses on internal documents for incident management. To test plans and gain experience, the organization performs rehearsals regularly. This is compliant with ISO/IEC 27035 and NorSIS's guideline. The organization has seen benefits of conducting rehearsals together with their suppliers as areas of improvement have been revealed and measures implemented. Rehearsals has shown to be useful since they have revealed the challenges related to information management, allocation of responsibilities and communication as discussed previously.  

To disseminate information about security best practice to employees, the organization conducts awareness campaigns addressing various topics. One employee mentioned that the security measures proposed in awareness campaigns were too strict and socially unacceptable. Another said he would rather take the risk of a security incident than to comply with the proposed best practice. Here, the importance of having an appropriate balance between security and usability is highlighted. This was also discussed in one of the presentations on Sikkerhet \& S\aa rbarhet 2013:

\begin{newquote}{John Arild Amdahl Johansen}
``Security must \textbf{never} stop business."
\end{newquote}

He stated that if security measures are too complex, users will find a way to circumvent the rules and thus initial the security measures are compromised. However, it should be noted that most employees in the survey found the awareness campaign useful. The two employees who did not find the campaign useful had an IT background which might indicate that employees' impression of the campaign varied with individual background and IT knowledge. 

The organization use monitoring systems as well as employees as sources for incident detection. This is in accordance with recommendations from most of the standards and guidelines presented in section \ref{section:standardsandguidelines}. However, the employee survey indicated that the reporting routines for employees are not satisfactory. Most of the employees were unsure about where to report incidents and only one knew that they are supposed to report to Supplier 1. Many said they would have reported to the local or central IT manager. Their overall uncertainty related to reporting may indicate that their reporting routines are not well enough established throughout the organization. Additionally, a few stated that they were not familiar with reporting routines since they had never needed to report anything. This finding is similar to the one in case A where some employees believed information security did not concern them. 

The organization uses a predefined classification scale for categorizing incidents. This is compliant with ISO/IEC 27035 and ENISA's good practice guideline. Incidents are categorized based on impact. The interviewee from Supplier 2 mentioned that the categorization is further used to prioritize incidents. An incident prioritization based on impact is recommended by ISO/IEC, SANS, NorSIS and the ITIL framework.

All handles incidents are logged and included in the log is the root cause of the incident. Root cause analysis is part of the ITIL problem management process. ISO/IEC 27035, NIST SP 800-61, the SANS Incidents Handler's Handbook and the ITIL framework specifies logging as being important.

The organization is very concerned with scaling the incident handling team based on the severity of the incident in question. Based on the situation, there may be people from both the organization and the suppliers involved in incident handling, but they try to limit the number of people that has authority to make decisions.

They have routines for preservation of electronic evidence. This is compliant with ISO/IEC 27002. 

The organization conducts regular meetings discussing serious incidents. They perform trend analysis by looking at incident reports provided by their two main suppliers. These activities are described as essential in ISO/IEC 27035. Additionally they have debriefing meetings after serious incidents. This is recommended by all standards and guidelines presented in section \ref{section:standardsandguidelines}.

The organization has tried to ensure a high level of redundancy in order to limit availability related consequences of security incident. \textit{Recovery}

Even though not all employees stated to know or could provide a definition of an information security incident, most of them gave relevant examples of incidents. This might indicate an overall satisfactory understanding of information security in the organization. As in case A, most employees stated to be attentive to incidents, even though they also said they did not know what an incident is. 

As described in section \ref{relatedwork} is recommended that incident handlers should have knowledge about the organization's IT systems and services in order to better be able to recognize abnormal behaviour. This type of knowledge is often tacit and the importance of it is often not recognized. The organization may have difficulties utilizing such knowledge as incident handling is to a large degree outsourced. However, the handlers at Supplier 1 and Supplier 2 may have gained such knowledge as well as they handle the organization's daily IT operations and application management respectively. %Paper: \cite{werlinger2010preparation}

\section{Case C}

Reporting. From ISO/IEC 27002, they fulfill 13.1.1 d)

\textit{Under ``sikkerhet og srbarhet"konferansen nevnes det at man skal ta opp infosik so et tema under rlige medarbeidersamtaler, noe som blir gjort av virksomhet C og som nevnes som et effektivt tiltak hos dem. Str litt om det under ``awareness and training".}

\section{Underlying structures of experiences}
\subsection{Communication}
\begin{itemize}
\item \textit{Communication}
\begin{itemize}
\item \textit{\textbf{Who} should be contacted \textbf{when}?}
\item \textit{Who should be contacted when the actual contact person cannot be reached?}
\end{itemize}
\end{itemize}

\subsection{Information}
\textit{Closely linked with communication}

\begin{itemize}
\item \textit{Information}
\begin{itemize}
\item \textit{What information needs to be collected?}
\item \textit{What information should be \textbf{communicated} to \textbf{whom}?}
\item \textit{Who should collect it?}
\item \textit{How can it be collected?}
\end{itemize}
\end{itemize}

\textit{As found in case B, under lessons learned:
The interviewee from Supplier 2 mentioned that that handling and collecting information is a challenge. As an employee talks about information dissemination problems in \cite{ahmad2012incident}.
``it's the sharing, or rather finding of information. The information is there, each day I find a new resource that's got great information."}

\subsection{Experience}

Even though predefined plans exist to handle a specific incident, following the plan might not always be the best solution. If the person dealing with an incidents has long experience, he might take actions that are not according to the plan, but that is the best way to solve the incident. The incident handler's experience is just as important to effective and efficient incident management/response as having plans and procedures. Link this to RQ3.

.

\begin{itemize}
\item \textit{Situational Awareness}
\begin{itemize}
\item\textit{ Necessary to make fast and correct decisions in a complex and dynamic environment. It is challenging to know exactly where the incident ``hit" if one does not have situational awareness, thus knowing what to do and who's responsible. This was  seen as a challenge in case B as well as the main background for starting the major ``lessons learned" project in case C.}
\end{itemize}
\end{itemize}

\textit{Another finding is that even though procedures are important, experience may be just as important, as it is impossible to plan detailed for anything, and often one must make decisions on the spot.}

\subsection{Responsibilities}
\textit{Also closely linked with communication. Better routines for communication may make responsibilities clearer?}

\begin{itemize}
\item \textit{Responsibilities}
\begin{itemize}
\item \textit{Who is responsible for the incident?}
\item \textit{Who is responsible for \textbf{fixing} it?}
\item \textit{Who is responsible for covering the costs?}
\end{itemize}
\end{itemize}

\textit{These challenges (especially communication and responsibility) are extra evident in Case B, where there are several suppliers and many people that need to cooperate, but it is also seen in Case A, where there is only one organization that handles its own IT operations involved.}

Another aspect is that users may not be aware that part of the responsibility lies with them. For example users that are owners of information may have a certain responsibility for protecting this information. An example is in case A, where users are responsible for classifying the information they process, but the interviewee thought that many are not aware of this (even though it is stated in the information security policy).

\textbf{Recommendations}
\begin{itemize}
\item Training to gain experience since experience has shown to be just as important as having established plans.
\item Having good plans for communication, since this seems to be challenging.
\end{itemize}

