\chapter{Discussion}
\label{chp:discussion}
In this chapter the findings from chapter \ref{chp:findings} are discussed and links between research questions and findings are established. The research questions are presented in section \ref{sec:objectives}. Further, underlying structures of experiences that emerged from the data are discussed. These findings are not necessarily directly linked to the research questions. This is in compliance with the general inductive research approach as described in \ref{sec:qualitativeAnalysis}.  



\textit{What about management commitment? Is that present in the organizations? Very important according to ISO}

\textit{The gap between implemented security measures and the sensitivity and value of information is increasing according to \cite{Morketall2012}. Is this also evident in our findings? Could be in case A}

\textit{From the live hacking session at DND S\&S ``Consider employees as part of the organization's sensor network." Employees can be a great source of detecting incidents/attacks. Especially with regards to social engineering and targeted attacks, which are increasing threats. In case A employees are not utilized as resources in incident detection, as they seem to lack knowledge and qualifications to be able to recognize incidents.}

\section{Case A}
\textit{From Case A:}

\textit{Organization A provides feedback to users that are reporting events. This is in accordance with ISO27002.}

\textit{RQ1:}\\
The organization has general guidelines for incident handling, but they do not have one plan that encompasses all possible incidents. They have detailed guidelines for the most frequent incident types. They also have contingency plans for all departments in addition to a central contingency plan. 

The organization has a ``principles for information security" document.
They have specific procedures for .

They have a policy for incident handling. This is compliant with ISO/IEC 27035. This document was not presented, so we did not get the chance to review the content.

Awareness and participation of all personnel is important (ISO/IEC 27035). \textit{To what extent is this present?} They have performed an awareness campaign. 

Additionally they perform rehearsals of contingency plans. Compliant with ISO/IEC 27035.

The existence of an information security policy is stated to be important (SANS). The organization has an information security policy. It may however not be well enough established throughout the organization. This is both consistent with what the IT security manager said and what wee discovered through the employee survey.

It is stated in the organization's information security policy that information shall be classified by the information owners. This is compliant with ISO/IEC 27002. The fact that so few of the employees were familiar with the information security policy could indicate that they are not aware of their responsibility for classifying information. This is supported by the IT security manager who stated that their information classification is not satisfactory. He believes that employees are not aware of this policy requirement. The standard emphasizes that classification of information is important to ensure proper protection of information and the need for special handling measures. This could indicate that the organization's information is not sufficiently protected with regards to its sensitivity and value. The organization processes large amounts of sensitive data, and this finding is thus alarming.

The IT security manager stated that it is the managements responsibility to make sure that the policy is well established throughout the organization and as the policy does not seem to be well established there may be a lack of management commitment to information security. This management commitment is highlighted as critical by ISO/IEC 27001. 

The IT security manager has the impression that users are good at reporting issues such as suspicious e-mails. However, he still believes they have an underreporting of incidents. The majority of the participants in the employee survey said they did not report suspicious e-mails. It should be noted that only a small number of employees are represented in this study and they had mainly received spam and not e-mails targeted specifically to the organization. Among the 15 participants, only one claimed to know which and to whom cases should be reported. Even though the majority of the participants thought that they would be able to figure out whether incidents should be reported, this finding is alarming, especially if it is representative for the entire group of users.

The IT security manager emphasized that it is important that users understand which security limitations the systems they use have. The tools employees use for communication, sharing and storing information might not be as secure as they think. Most of the employees in the survey did not know what an information security incident is, but they still claimed to be attentive to incidents in their everyday work. These contradictory answers might indicate that information security is not well understood and that employees have an erroneous picture of their own security knowledge and awareness. Organizational culture, motivation, risk awareness and attitude could be contributing factors to these contradictory answers. Employees may wish to appear knowledgeable and construct answers thereafter. This can be due to both organizational culture and attitude. Several employees mentioned that information security did not concern them. They believed it was not relevant to their work and that they were not exposed to attacks or incidents, even though they had access to information and performed their work on a computer. This might indicate a lack of risk awareness among employees and is supported by the IT security manager who said that one issue revealed through rehearsals was employees' understanding of risk. As mentioned in \ref{sec:threatLandscape}, vulnerabilities exist mainly due to lack of understanding of risk.

The overall impression is that employees are positive to awareness campaigns, and found previous campaigns useful. Several employees would have liked such campaigns to be conducted more often. Even those claiming that most of the content was known, found it useful as a reminder. This shows a positive attitude to learn more about information security, and might indicate that it is the lack of management commitment and not employees' attitudes that is the reason for insufficient understanding and awareness of information security.

It seems like their reporting system, and incident handling in situations dealing with major incidents work well, but that often they do not have adequate routines. The IT security manager believes this originates in the organizational structure where sufficient resources are not allocated to incident management. 

\textit{This organization has developed several plans and procedures addressing information security and incident management, but not all of these seem to be well \textbf{established} throughout the organization. Some plans and procedures may be more established and ``known" than others. Ref research question number 1}

\textit{RQ2:}\\
The organization has not implemented one specific standard or guideline concerning incident management, but has based its approach on components from ISO/IEC 27001 and 27002 as well as the \ac{ITIL} framework. 

The organization has a team that works with incident handling. This is compliant with recommendations in all standards and guides presented in this report.

\textit{RQ3:}\\

\section{Case B}

\section{Case C}

\textit{Under ``sikkerhet og srbarhet"konferansen nevnes det at man skal ta opp infosik so et tema under rlige medarbeidersamtaler, noe som blir gjort av virksomhet C og som nevnes som et effektivt tiltak hos dem. Str litt om det under ``awareness and training".}

\section{Underlying structures of experiences}
\subsection{Communication}
\begin{itemize}
\item \textit{Communication}
\begin{itemize}
\item \textit{\textbf{Who} should be contacted \textbf{when}?}
\item \textit{Who should be contacted when the actual contact person cannot be reached?}
\end{itemize}
\end{itemize}

\subsection{Information}
\textit{Closely linked with communication}

\begin{itemize}
\item \textit{Information}
\begin{itemize}
\item \textit{What information needs to be collected?}
\item \textit{What information should be \textbf{communicated} to \textbf{whom}?}
\item \textit{Who should collect it?}
\item \textit{How can it be collected?}
\end{itemize}
\end{itemize}

\textit{As found in case B, under lessons learned:
The interviewee from Supplier 2 mentioned that that handling and collecting information is a challenge. As an employee talks about information dissemination problems in \cite{ahmad2012incident}.
``it's the sharing, or rather finding of information. The information is there, each day I find a new resource that's got great information."}

\subsection{Experience}

\begin{itemize}
\item \textit{Situational Awareness}
\begin{itemize}
\item\textit{ Necessary to make fast and correct decisions in a complex and dynamic environment. It is challenging to know exactly where the incident ``hit" if one does not have situational awareness, thus knowing what to do and who's responsible. This was  seen as a challenge in case B as well as the main background for starting the major ``lessons learned" project in case C.}
\end{itemize}
\end{itemize}

\textit{Another finding is that even though procedures are important, experience may be just as important, as it is impossible to plan detailed for anything, and often one must make decisions on the spot.}

\subsection{Responsibilities}
\textit{Also closely linked with communication. Better routines for communication may make responsibilities clearer?}

\begin{itemize}
\item \textit{Responsibilities}
\begin{itemize}
\item \textit{Who is responsible for the incident?}
\item \textit{Who is responsible for \textbf{fixing} it?}
\item \textit{Who is responsible for covering the costs?}
\end{itemize}
\end{itemize}

\textit{These challenges (especially communication and responsibility) are extra evident in Case B, where there are several suppliers and many people that need to cooperate, but it is also seen in Case A, where there is only one organization that handles its own IT operations involved.}

Another aspect is that users may not be aware that part of the responsibility lies with them. For example users that are owners of information may have a certain responsibility for protecting this information. An example is in case A, where users are responsible for classifying the information they process, but the interviewee thought that many are not aware of this (even though it is stated in the information security policy).

