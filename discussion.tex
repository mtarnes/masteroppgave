\chapter{Discussion}
\label{chp:discussion}
In this chapter the findings from chapter \ref{chp:findings} are discussed and links between research questions and findings are established in sections \ref{sec:discussionCaseA}, \ref{sec:discussionCaseB} and \ref{sec:discussionCaseC}. The research questions were presented in section \ref{sec:objectives}. Further, underlying structures of experiences  are discussed in section \ref{sec:discussionStructures}. These findings are not necessarily  directly linked to the research questions, but emerged from the data. This analysis method is based on the general inductive research approach as described in \ref{sec:qualitativeAnalysis}.  

\section{Case A}
\label{sec:discussionCaseA}
Organization A has general guidelines for incident handling, but does not have one plan that applies to all possible incidents. However, they do have detailed guidelines for the most frequent incident types. In addition to a central contingency plan, they have developed contingency plans for each individual department. Organization A is compliant with ISO/IEC 27035's recommendation of having a policy for incident handling.

SANS and the \acs{ITIL} framework state that having an information security policy is important. Further, the \acs{ITIL} framework emphasizes that employees should have access to and be aware of this policy. Organization A has developed an information security policy. It may, however, not be well enough established throughout the organization. This is supported by the employee survey where few claimed to be familiar with the policy. It is also consistent with the fact that IT security manager's believes that not all users have detailed knowledge of the policy. 

An abuse system and a deviation management system are used for reporting and documenting incidents and vulnerabilities. Having reporting and documentation systems and procedures is recommended by all guidelines presented in section \ref{section:standardsandguidelines}.   The implementation guidance in ISO/IEC 27002 suggests that those reporting information security events should be notified of results after the reported issue has been dealt with. Organization A is compliant with this recommendation as they provide feedback to those reporting security issues. 

The organization has implemented monitoring systems such as IDSs, which is recommended by most relevant standards and guidelines. In addition to technical detection mechanisms, users can be a great source of detecting incidents, and therefore the organization should have available reporting systems. This was highlighted in a presentation at the ``Sikkerhet \& S\aa rbarhet 2013" conference:

\begin{newquote}{Vidar Sandland and Hans Marius Tessem, NorSIS}
``Consider employees as part of the organization's sensor network."
\end{newquote}

This is especially important with regards to social engineering and targeted attacks, which are increasing. Their abuse and deviation management systems facilitate users' reporting of incidents. One interesting observation is that employees seem to lack knowledge and qualifications to be able to recognize incidents, which might indicate that they are not fully utilized as resources for incident detection. We found that employees are unaware that they are required to report incidents, unaware of how to report and under which circumstances reporting is necessary. The majority of the participants in the employee survey said they did not report suspicious e-mails. It should be noted that only a small number of employees are represented in this study and they had mainly received spam and not malicious e-mails targeted specifically at the organization. Among the 15 participants, only one claimed to know which and to whom incidents should be reported. Even though the majority of the participants thought that they would be able to figure out whether incidents should be reported, this finding is alarming, especially if it is representative for the entire group of users. 

According to SANS, NorSIS, the ITIL framework and ISO/IEC, incident prioritization rules should be based on an organizational impact analysis and be part of the incident management preparation phase. One way to evaluate potential organizational impact caused by incidents is to conduct a risk assessment. The organization conducts risk assessments regularly, leading to a categorization scheme which provides the basis for their prioritization. Another important part of the preparation phase highlighted in the standards and guidelines is the establishment of an \ac{IRT}. Organization A does not have their own \ac{IRT}, but has dedicated personnel for incident handling and a crisis team to handle the most severe incidents.

It is stated in the organization's information security policy that information shall be classified by the information owners. This is compliant with ISO/IEC 27002. The fact that so few of the employees were familiar with the information security policy could indicate that they are not aware of their responsibility for classifying information. This is supported by the IT security manager who stated that their information classification is not satisfactory. He believes that employees are not aware of this specific policy requirement. ISO/IEC 27002 emphasizes that classification of information is important to ensure proper protection of information and to identify the need for special handling measures. This could indicate that the organization's information is not sufficiently protected with regards to its sensitivity and value. The organization processes large amounts of sensitive data, and this finding is thus alarming.

The IT security manager stated that it is the management's responsibility to make sure the policy is well established throughout the organization. There may be a lack of management commitment to information security as the policy does not seem to be well established within the organization. This management commitment is highlighted as critical by ISO/IEC 27001. 

The IT security manager emphasized that it is important that users understand the security limitations in the systems they use. The tools employees use for communication, sharing and storing of information might not be as secure as they assume. Several employees mentioned that information security did not concern them. They believed it was not relevant to their work and that they were not exposed to attacks or incidents, despite having access to information and performing their work on a computer. Even though most of the employees in the survey did not know what an information security incident is, they still claimed to be attentive to incidents in their everyday work. These contradictory statements might indicate that information security is not well understood and that employees have an erroneous picture of their own security knowledge and awareness. Organizational culture, motivation, risk awareness and attitude could be contributing factors to this contradiction. Organizational culture and attitude might influence employees to attempt to appear knowledgeable and construct answers thereafter. These findings might indicate a lack of risk awareness among employees and is supported by the IT security manager who said that one issue revealed through rehearsals was employees' understanding of risk. As mentioned in \ref{sec:threatLandscape}, vulnerabilities exist mainly due to lack of employees' understanding of risk and this is therefore an alarming finding. By raising employees' awareness vulnerabilities and thus incidents can be mitigated.

With regards to rehearsals, the organization complies with recommendations in ISO/IEC 27035 and the NorSIS guideline. The organization conducts contingency rehearsals for the crisis team. It is important for the crisis team to gain experience to become better prepared to handle real situations. One interesting observation is that their rehearsals include situations where they know they lack routines. This gives the team an opportunity to train on improvising in situations where there are no predefined plans. Routines developed through such a bottom-up approach might be better established within the team than routines imposed by others, due to the team's participation in the development and implementation.

Employees' awareness and participation in incident management procedures are important according to ISO/IEC 27035. The employee survey indicates that employees are positive to awareness campaigns and found previous campaigns useful. Several employees would like such campaigns to be conducted more often. Even those claiming that most of the content in these campaigns was known, found them useful as reminders. The positive attitude towards learning more about information security, in addition to the wish of having awareness campaigns more often, might indicate that it is a lack of management commitment that is the reason for insufficient understanding and awareness of information security and not employees' attitudes. If that is the case it would be unfortunate as senior management commitment to incident management is highlighted as important in ISO/IEC 27035.

Organization A has individual contingency plans for each department which are initiated when serious incidents occur. If an incident has a particular high impact level or involves several departments, their central contingency plan is initiated. Having an escalation procedure is compliant with ISO/IEC 27035. This standard also specifies that it should be a main activity for the \ac{IRT} to allocate responsibilities. The allocation of responsibilities is performed in the organization by delegating parts of the incident handling to specific sections that have expertise relevant for solving the incident.

The main objectives of the incident management process in ITIL are to restore service as quickly as possible and to limit adverse business impact. Guidelines such as NIST SP 800-61 and ENISA's Good Practice Guide for Incident Management also specifies that recovery of affected systems is important.  The IT security manager however, does not see restoring services as quickly as possible as the top priority as long as the incident is not fully resolved. He is more concerned with making sure that the incident is properly resolved than to rush the restoring of affected systems. It should be noted that the guidelines also emphasize the importance of fully resolving incidents. In ITIL this is part of the problem management process that focuses on root cause analysis. The IT security manager would have liked to perform more measures to ensure that the root cause of incidents is identified and eradicated, but is often not allocated the resourced needed. This may be another indication of a lack of management commitment to information security in the organization.

Organization A is responsible for its own IT operations and does therefore not need to coordinate with external parties during incident handling. Nevertheless, the organizational structure has shown to be a challenge with regards to communication during incident handling. The IT security manager specifically mentioned that the designated contact person changes on a daily basis for some sections. This represents the challenge for the IT security manager who experiences uncertainty with regards to the designated contact person's knowledge and experience which may vary.

Organization A's lessons learned phase is compliant with the recommendations in the majority of the standards and guidelines. They perform reviews of severe incidents to identify root causes and improvements. Further, these improvements are implemented and in specific cases shared with trusted communities and partners. This is recommended in ISO/IEC 27035. 

\paragraph{Summary that answers the research questions}
Organization A has not implemented any specific standard or guideline for incident management, but has based its approach on components from ISO/IEC 27001 and 27002 as well as the \ac{ITIL} framework. Nevertheless, they seem to comply reasonably well with the recommendations in the standards and guidelines presented in this report. They have developed several plans and procedures addressing information security and incident management specifically, but not all of these seem to be well \textit{established} throughout the organization. Unexpected situations where sufficient plans do not exist may occur. Additionally, they do not always have the required staff to respond efficiently. The overall impression is that incidents generally seem to be handled in accordance with their predetermined plans.


\section{Case B}
\label{sec:discussionCaseB}
Organization B has developed an information security policy, with intention to define senior management's position IT security position. This might indicate some level of management commitment, which is highlighted as essential in the ISO/IEC 27000 family of standards. Having a security policy is stated to be important by SANS and the ITIL framework. Further, the ITIL framework recommend that employees should have access to and be aware of the information security policy. The organization seems to be compliant with this recommendation as most employees answered that they were to some extent familiar with the organization's information security policy. 

The interviewees from Organization B and its two main suppliers provided slightly different definitions of an information security incident. One variation in their definitions was that the interviewees from Organization B specified a distinction between \textit{security breaches}, i.e. incidents caused intentionally by employees, and other incidents. The interviewees from the suppliers did not specify this distinction. This makes sense as the two external suppliers are mainly concerned with affected systems, whereas it is the organization itself that handles disloyal employees. They all agreed that incidents causing loss of sensitive information is the worst possible incident Organization B can experience. This might indicate that they have the same prioritizations during incident handling.

Even though not all employees knew what an information security incident is or could provide a definition, most of them gave relevant examples. This might indicate a reasonably sound understanding of information security in the organization. However, as in case A most employees stated to be attentive to incidents, even though they also said they did not know what an incident is. This shows that there is still room of improvements related to employees' information security understanding and awareness.

Organization B and its suppliers have implemented various measures to prevent the occurrence of incidents. Prevention of incidents is stated to be fundamental to the success of an organization's incident response by NIST SP 800-61. Organization B has developed procedures for security vulnerability handling, whereas Supplier 1 is responsible for securing the organization's network.  

The IT manager stated that constructing a holistic plan is challenging. Even though they lack a holistic plan, they do have contingency plans. Specific contingency plans have been developed in cooperation with Supplier 1 and Supplier 2 for use in crisis situations. For incidents that escalate to a level where they are categorized as crises, the contingency plans are initiated. These plans include guidelines for communication and escalation. Developing such guidelines is compliant with recommendations in SANS, NorSIS and ISO/IEC. In NorSIS's guide for incident management it is emphasized that information security should be considered when \acp{SLA} are developed for outsourcing. Supplier 2's plan is developed to ensure fulfilment of the \ac{SLA}, and thus complies with the recommendation in NorSIS's guideline. 

Allocating resources for the development of detailed plans is not the organization's main focus. They allocate their resources differently as they believe having experienced incident handlers, that are able to understand the situation and make decisions thereafter, are more important for a successful incident handling than having detailed plans. This is an interesting observation as standards and guidelines tend to focus on plans and procedures rather than experienced incident handlers even though they do recommend training. 

The supply chain manager highlighted information dissemination as one of the most challenging parts of incident management. It can be challenging to know what information to communicate, when to communicate it and to whom. Making wrong decisions with regards to information could cause delays in the incident handling and may result in serious consequences. Another challenge mentioned by the interviewee from Supplier 2 was handling and collecting information from various sources. The organization is well aware of these challenges and thus focuses on them in their rehearsals. An example is that they conduct rehearsals in cooperation with their suppliers. The finding that information dissemination is challenging, supports the findings from one of the case studies discussed in section \ref{relatedwork}. In this case study participants meant that better information dissemination would improve their security procedures, which in turn would improve the overall security of their organization. 

It can be challenging to know who has the responsibility in various situations. The supply chain manager stated this is especially challenging for minor incidents, where responsibilities are not specified in detail. The interviewee from Supplier 2 claimed that not knowing who is responsible for the incident is challenging. He mentioned cases where the owner of an error does not understand that he owns it or does not acknowledge ownership due to potential costs. Some incidents may be so complex that knowing exactly where they originated, and thus determining who is responsible for handling them is difficult. Nevertheless, contacting the responsible person was highlighted as one thing that has worked well in Supplier 2's part of the incident handling. We believe the challenge of determining who is responsible in various cases could be mitigated by improving communication procedures and establishing well defined responsibilities beforehand.

The organization finds it important to scale the incident handling team based on the severity of the specific incident. Based on the situation, the team may include people from both the organization and their suppliers. However, they try to limit the number of people with authority to make decisions. Consequently, incidents have to be assessed and decisions of who to involve and inform have to be taken for each individual incident. This could make communication, responsibility allocation and information dissemination even more challenging. 

Supplier 1 keeps track of trends related to security incidents, by monitoring their internal systems. This is compliant with recommendations for the preparation phase in ISO/IEC 27035 although current vulnerabilities should preferably be formalised in an incident management policy. 

Organization B performs regular rehearsal to test their plans and gain experience. This is compliant with ISO/IEC 27035 and NorSIS's guideline. The organization has seen benefits of conducting rehearsals in cooperation with their suppliers, as areas of improvement have been revealed and measures implemented thereafter.   

To disseminate information to employees about security best practices, Organization B conducts awareness campaigns that address various topics. One employee mentioned that the security measures proposed in one of the awareness campaigns were too strict and socially unacceptable. Another said he would rather take the risk of a security incident than to comply with best practice. These statements emphasize the importance of having an appropriate balance between security and usability. This was also discussed in one of the presentations at the ``Sikkerhet \& S\aa rbarhet 2013" conference:

\begin{newquote}{John Arild Amdahl Johansen, Buypass AS}
``Security must \textbf{never} stop business."
\end{newquote}

In this presentation it was stated that if security measures are too complex, users will find a way to circumvent the rules and thus the initial security measures are compromised. The two employees who did not find the campaigns useful has an IT background which might indicate that employees' impression of such campaigns varies with individual background and IT knowledge. Even though these two employees were familiar with the campaigns' content, their answers indicated a negative attitude towards awareness raising activities, and might imply an unsatisfactory security culture in Organization B. However, it should be noted that most employees in the survey found the awareness campaigns useful.

Organization B uses monitoring systems and employees as sources of incident detection, which is in accordance with recommendations from most of the standards and guidelines presented in section \ref{section:standardsandguidelines}. The employee survey indicated that the knowledge of reporting procedures for employees is not satisfactory, as most of the employees were not sure where to report incidents. Only one knew that they are supposed to report incidents to Supplier 1. Many said they would have reported to the local or central IT manager. Their overall uncertainty related to reporting may indicate that reporting procedures are not well enough established throughout the organization. Additionally, a few stated that they were not familiar with reporting routines as they had never needed to report anything. This attitude is similar to the one found in case A where some employees believed information security did not concern them. 

Organization B uses a predefined classification scale for the categorization of incidents, based on impact level. This is compliant with ISO/IEC 27035 and ENISA's Good Practice Guideline for Incident Management. The interviewee from Supplier 2 mentioned that the categorization is further used to prioritize incidents. Incident prioritization based on impact level is recommended by ISO/IEC, SANS, NorSIS and the ITIL framework.

All incidents are logged and the root cause of the incident is included in the log. Most of the standards and guidelines discussed in section \ref{section:standardsandguidelines} specifies logging as being important. Additionally, the ITIL framework focuses on root cause analysis in the problem management process.

Organization B holds regular meetings where they discuss serious incidents. They perform trend analyses by evaluating incident reports provided by their two main suppliers. These activities are described as essential in ISO/IEC 27035. Additionally, they have review meetings after serious incidents. Overall, the organization's post-incident activities seem to be in accordance with relevant standards and guidelines.

As described in the standards and guidelines, recovery is an important part of incident response. Availability management and IT service continuity management are two processes in the ITIL framework that are related to recovery. Organization B has tried to ensure a high level of redundancy, which makes recovery easier and more efficient. Additionally, it might limit availability related consequences of security incidents. As part of the same phase in the NIST guideline is gathering and handling of electronic evidence. The organization has routines for preservation of electronic evidence, which is compliant with ISO/IEC 27002.

One of the papers discussed in section \ref{relatedwork} recommends incident handlers to obtain knowledge about the organization's IT systems and services in order to better be able to recognize abnormal behaviour. It is stated that this type of knowledge often is tacit and that its importance is often not recognized. Organization B may have difficulties utilizing such knowledge as incident handling is to a large degree outsourced. However, the handlers at Supplier 1 and Supplier 2 may have gained such knowledge as they handle the organization's daily IT operations and application management respectively. %Paper: \cite{werlinger2010preparation}

\paragraph{Summary that answers the research questions}
Organization B has not strictly based plans and procedures on standards or guidelines for incident management, however they are relatively compliant with the standards and guidelines presented in section \ref{section:standardsandguidelines}. Some of their procedures seem to be well established such as their escalation procedures. However, they also have some procedures that are not sufficiently established. As in case A, it seems that reporting procedures are not sufficiently established in the organization as employees showed uncertainty related to these procedures. They have a set of predefined plans, but the importance of having experienced incident handlers becomes extra evident in this case as their incident handling is distributed and their team scalable. Our overall impression is that incident handling has been performed in accordance with predefined plans. However, their plans are quite general and they thus focus on being able to improvise during incident handling, i.e. making situation-specific decisions.

\section{Case C}
\label{sec:discussionCaseC}
Organization C has an information security policy which is reasonably well known among participants in the employee survey. In one of the presentations at the ``Sikkerhet \& S\aa rbarhet 2013" conference, Difi\footnote{The Norwegian Agency for Public Management and eGovernment} recommended discussing security during employee appraisals. The IT security manager said that their security handbook is always a topic in the annual employee appraisals. Further, most of the employees in the survey seemed to have a good understanding of what an information security incident is. This could indicate that information security is somewhat better understood among employees in this organization than in case A and B. There are several findings that could be reasons for this. It can be assumed that the annual review of the security handbook ensures that employees are aware of their individual security responsibilities. Further, the organization focuses on employees' importance in information security work. They believe it is important to have a security-positive environment in the organization, which they have stated in their policy. The focus on employees could have increased the overall security understanding in Organization C. We believe another important factor is that Organization C's core activity is the delivery of IT services. Consequently, they have a high focus on information security.

The ISO/IEC 27000 family of standards emphasizes the importance of management commitment both to incident management and information security in general. SANS and NorSIS also mention this in their guidelines. The organization seems to have some extent of management commitment as the aim of their information security policy is to communicate the management's direction and commitment to information security. One reason for this might be that they have several ISO/IEC 27001 certifications, and this standard states that management shall provide evidence of its commitment to information security. %It thus seems like the organization in case C has a larger degree of management commitment than the organizations in case A and B.

Although the organization has several ISO/IEC 27001 certifications, they do not use ISO/IEC 27035 or other standards specific to incident management. They have chosen to base their entire service management, including incident management, on the ITIL framework. The organization has various incident handling plans aimed at specific incident types. The organization has a detailed description of their general incident management process. They have specified a separate workflow for major incidents. Having separate procedures for major incidents is in accordance with recommended practice in the ITIL framework. 
%More discussion?
 
The fact that all of the employees had attended courses or other awareness raising activities, supports that the organization has a high focus on improving employees' security knowledge and awareness. This observation further supported by statements from the IT security manager, and may also confirm that the organization follows through on their policy objectives. Additionally, employees' attitude towards awareness raising activities shows signs of a security-positive environment.  
 
The employee survey showed some uncertainty with regards to reporting procedures similar to case A and B. The few employees that did not know where to report claimed to know where to find relevant information. Employees' knowledge of where to find relevant information is positive, but this might not be efficient enough in all situations as it introduces an extra delay. The IT security manager said that employees are advised to report incidents to the \ac{IRT}. However, none of the employees mentioned this. Suspicious e-mails was given as one example of cases that should be reported. Still, none of the employees had previously reported such e-mails. It should be noted that only two of the employees claimed to have received such e-mails. 

The organization has monitoring systems for incident detection. ISO/IEC 27035 states that organizations should have monitoring systems to aid detection. NIST mentiones monitoring systems as a tool for incident detection and ENISA recommends the use of such systems in addition to reporting. The fact that none of the participants in the survey had reported incidents could indicate that employees are not fully utilized as part of the organization's sensor network for detecting incidents. This assumption is supported by one of the employees who stated that they should probably report incidents more often. Further, this is supported by the IT security manager who suspects underreporting. He believes the threshold for reporting is high. This is unfortunate as Organization C tries to establish a security-positive environment. Underreporting might indicate that they still have some work to do with regards to achieving this. Nevertheless, reporting procedures seem to be somewhat better established in this organization than in the other two cases. 

Vulnerabilities can be reported though a risk framework and they are subsequently handled. Handling vulnerabilities can aid in incident prevention, which is an important part of incident management and is stated to be a fundamental factor by NIST. Additionally, NorSIS specifies preventive measures to be one of the most cost effective ways to perform incident management. 

Incidents are categorized based on their impact level. Organization C bases their categorization method on the ITIL framework. The categorization determines which incident response procedures to initiate. Categorizing incidents and using the categorization to determine further actions are compliant with recommendations from the majority of the standards and guidelines discussed in section \ref{section:standardsandguidelines}.

The main purposes of their incident management process are in line with the ITIL framework. This also applies to their execution of the process activities where they use incident handlers as well as an \ac{IRT} for assistance and expertise. In addition to the IRT, the organization has a \ac{CIM} team that can provide complementary competence if necessary. As illustrated in figure \ref{fig:workflowcaseC}, the service desk function is the first line of incident response. This is compliant with recommendations in the ITIL framework. If the service desk is not able to handle the incident internally, specialists are contacted. Further, if the incident is assessed as major, the major incident process is initiated and additional resources must be engaged. In cases where extra dedicated resources are needed, a Task Force is established. This shows that the organization has mature and well established escalation routines.

It is evident in the organization's internal documentation that they believe successful incident management is based on contingency plans and predefined tasks. The employee we had e-mail correspondence with acknowledged that it would be ideal to have plans and procedures for all possible incidents, but that this might not be practically feasible. He emphasized that incident handlers who compose a set of predefined activities to customize the incident response for specific incidents are key to a successful incident handling. Due to variations in incidents, an experienced incident handler is more important than rigid process adherence.

Organization C has developed procedures for handling electronic evidence. They perform duplications of hard drives if necessary, but further investigations are conducted by a third party. The existence of routines for handling electronic evidence is compliant with NIST SP 800-61 and ISO/IEC 27002.

Two requirements in Organization C's incident management process address documentation. These requirements state that all incidents should be registered and all actions recorded. In addition, experiences and potential improvements are documented. It is fair to say that the organization follows best practise, as most of the standards and guidelines discussed in \ref{section:standardsandguidelines} emphasize the importance of logging.

Organization C conducts rehearsals and occasionally customers or the government participate. Lessons learned activities where they try to identify possible improvements are conducted after major incidents and rehearsals. According to the IT security manager these activities are useful as they often result in concrete measures. The fact that Organization C has initiated a project to improve their incident management scheme shows their commitment to improve their incident management. This project is allocated extensive resources which again supports the assumption of established management commitment to information security.

Incident handling is distributed among the organization and its customers. Thus, the challenges of collaboration and coordination are evident in this case. As in the other two cases, communication emerges as a challenge. After rehearsals, it has often been recognized that interaction with external parties can be improved. We believe the establishment of more specific communication routines as well as well defined responsibilities might mitigate these challenges.

\paragraph{Summary that answers the research questions}
The organization seems to have a set of well established plans and procedures as well as experienced incident handlers. Additionally, they have several ISO/IEC 27001 certifications and has implemented the ITIL framework for their IT service management. Their incident management is highly compliant with the ITIL framework as well as being relatively compliant with the other standards and guidelines presented in section \ref{section:standardsandguidelines}. It seems like incidents have mostly been handled in accordance with predefined plans, even though they emphasize the importance of experienced incident handlers as well. The uncertainty among employees with regards to reporting routines might indicate that these routines are not sufficiently established throughout the organization. Nevertheless, our findings indicate that this organization has an overall mature incident management. 

\section{Prominent Challenges and Observations}
\label{sec:discussionStructures}
This section discusses challenges and observations that we found prominent in our case study. It is important to note that these findings cannot be directly generalized. However, we find it reasonable to believe that some of these challenges and observations will be evident in other organizations as well, as the participating organizations are large and we assume that they are experienced.

\subsection{Communication}
During our case study we found that communication was regarded as challenging among all of the participants. The organizations had to various extents developed and implemented plans and procedures addressing communication. Successful incident response requires cooperation, thus establishing sound communication procedures for incident management is essential. Communication is further emphasized as one of the most important parts of incident management by NIST. The organizations we studied are large organizations and it is therefore not surprising that several parties are involved in their incident management. Even for Organization A, that does not have to coordinate with external parties during incident handling, incident handlers have to communicate and coordinate across several internal departments and sections. As an example from our study we highlight that the designated contact person in Organization A changes daily for some sections which imposed uncertainty for the IT security manager. 

Communication becomes even more challenging with distributed infrastructures and establishing sound communication procedures is vital. Our impression is that it is important to have available and updated contact lists, but that being able to determine the correct person to contact during incident response is just as important. In some cases, people with special knowledge or authorizations need to be involved. To be able to determine who the correct person is, situations have to be assessed and tacit knowledge about the organization and its employees can be used. This type of knowledge is difficult to document and thus difficult to include in plans. We therefore believe that in order to mitigate communication related challenges, employees involved in incident handling must have experience, which can be gained through rehearsals. 

A presentation given at the ``Sikkerhet \& S\aa rbarhet 2013" conference presented results from an audit performed by The Norwegian Data Protection Authority that highlighted a problem we also found evident in our case study. E-mail is still used for unstructured and informal communication, even though it is not a secure channel. Using insecure communication channels exposes the organization to targeted phishing attacks, e.g. as seen in a recent attack against the large Norwegian telecom corporation Telenor\footnote{More information about this attack can be found in \cite{phisingattack}}. The organizations in our case study, used e-mail for communication not only as a first notification of incidents but also during major incidents. To mitigate the risk of phishing attacks and disclosure of sensitive information, we recommend using secure communication channels where this is practically feasible. 


\subsection{Information}
Collecting information relevant to incident management was pointed out as challenging by participants in our case study. Especially for organizations with distributed infrastructures, there are many sources of information which makes the collection of correct information difficult. This observation supports findings in a case study conducted by Ahmad et al.\cite{ahmad2012incident} where the information security manager stated that the sharing or rather the finding of information was one of the most challenging parts of her job. 

Several of the participants in our study pointed out information dissemination as a challenge in incident handling. Knowing how much information to share can be difficult. On the one hand, too little information could give an erroneous picture of the incident which could in turn lead to wrongs decisions being made or insufficient actions being taken. On the other hand, too much information can be overwhelming and can cause delays in decision making as it has to be structured in order to be useful. It is important to communicate the \textit{right} information to the \textit{right} people. Information about incidents can obviously be quite sensitive and communicating such information to people who are not supposed to receive it can have serious consequences. Additionally, providing unnecessary information can be an annoyance and could at worst be counterproductive.

One employee mentioned that often they were not notified about changes made in the security policy, which is an example of poor information dissemination. However, we believe that employees' knowledge of details in the policy is not essential to a successful incident management as long as they are familiar with procedures and are capable of performing necessary actions. Providing information to employees is important, although this information should be relevant and useful.

We believe the development and establishment of clear information dissemination procedures, that can for instance be based on incident categories, could improve the information dissemination in organizations. If procedures for each predefined incident category exist and are established, it will be easier and more efficient to determine which information to share and with whom.

\subsection{Experience}
\label{sec:experience}
To be able to customize responses to specific incidents, experience is essential. This was highlighted by several of the participants in our study. Developing detailed plans for all possible scenarios is not feasible and probably not useful either. Having well established plans and procedures for incident handling is obviously important. Nevertheless, allocating resources to the development of detailed plans for all potential incidents is unproductive as this is not possible. There could always occur incidents that no one thought of beforehand. 

Experienced incident handlers is key for making rapid and correct decisions in a complex and dynamic environment. One obvious way to gain experience is by handling real incidents. However, organizations cannot wait for incidents to occur to gain experience, and thus rehearsals is a necessity. By conducting rehearsals addressing various types of incidents, plans and procedures can be tested and incident handlers will gain experience at the same time. 

In our opinion, neither experience nor rehearsals are sufficiently highlighted in the standards and guidelines considering how important this is for incident management. The organizations in our study focus on these two factors in their incident management. However, we believe they could benefit from conducting rehearsals more often.

Our impression is that having competent and experienced incident handlers that are both familiar with existing procedures and are capable of handling unexpected scenarios is essential to a successful incident management.

%Sitational awareness?

\subsection{Responsibilities}
It can be challenging to know exactly where an incident originated. Thus, knowing what to do and who is responsible for handling it is difficult. One of the interviewees said that the greatest challenge with incident handling is cases where no one understands that they ``own" the incident and thus no one takes responsibility. This ambiguity of who owns an incident can be due to uncertainty of where it originated. This challenge was also mentioned by another of the interviewees who stated that minor incidents can escalate and have serious consequences if no one takes responsibility. Further, ambiguous responsibilities in combination with costs of handling an incident might lead to a delay in the incident response if no one claims ownership and takes responsibility. 

Even though developing detailed plans for all possible scenario is not feasible, we still believe that having an appropriate detail level in plans addressing responsibilities could be beneficial. As it is in some situations difficult to determine who owns an incident, our best recommendation is improving communication to better be able to determine ownership and responsibilities in situations where this cannot be determined based on a predefined plan.

As emphasized in section \ref{sec:experience}, rehearsals are beneficial. We believe that rehearsals can contribute to revealing grey areas regarding responsibilities. Additionally, rehearsals can make incident handlers more suited to determine where incidents originated. As organization's incident management procedures matures, they become better equipped to determine responsibilities. The supply chain manager in Organization B emphasized that after years of working closely together, their experience and tacit knowledge help them determine who is responsible for handling an incident without this being documented. 

The challenge of determining responsibilities is extra evident in case B, as several suppliers are involved in their incident management. In this specific case, the two suppliers have separate main responsibilities. However, we assume that grey areas may emerge with regards to responsibilities even for this case if new or unexpected incidents occur. NorSIS's guideline emphasizes that responsibilities should be determined in an SLA when (parts of) incident management is outsourced. The standards and guidelines presented in section \ref{section:standardsandguidelines} do not provide specific recommendations for resolving ambiguities with regards to incident ownership and thus responsibilities for specific incidents. We recommend organizations to comply with the NorSIS Guideline for Incident Management, namely determining responsibilities in the \ac{SLA}.


\subsection{Employee Involvement}
When we contacted people for the employee survey we observed an interesting attitude among employees. Several employees seemed reluctant to participate due to their perception of their own lack of knowledge about information security. This was evident in comments such as:

\begin{quote}
\textit{``I don't know if I can help, I don't know anything about information security."}
\end{quote}

We suspect that some of the reluctance was due to employees being scared that the survey would ``reveal" their insufficient knowledge about information security. They seemed somewhat embarrassed about this insufficient knowledge and several said that they should probably have been more familiar with the organization's information security policy. There were however, some employees that admitted lack of knowledge and ``excused" this by saying that information security did not concern them. We find this very alarming as information security concerns \textbf{everyone} and as attacks taking advantage of employees, such as targeted malicious e-mails, is an increasing trend. 

Failing to classify information can lead to the information not being sufficiently secured according to its value. Findings from Organization A showed that information classification is not satisfactory. Employees seem to fail to recognize that the value and sensitivity of the information they process should determine how information should be secured and handled. This could lead to a gap between the sensitivity and value of information and implemented security measures, something which was also highlighted in a recent survey \cite{Morketall2012}.

Employees in our survey seem to have an overall positive attitude towards awareness campaigns. Many of them stated that they wished such campaigns would be performed more often. We believe that, as long as the campaigns are not too extensive, this attitude is consistent throughout organizations. Due to this positivity we also believe that employees can with advantage be more involved in rehearsals. Our findings did not show any user/employee involvement in rehearsals beyond the involvement of incident and crises handlers. If users are trained in reporting procedures and incident detection they can be utilized as part of the sensor network in a larger degree than they are today.

\section{Recommendations}
\label{sec:rec}
This section presents our recommendations based on both challenges and successful practices we found evident in the organizations studied.

\begin{itemize}
\item Use well established standards or guidelines as a basis for incident management, as these are based on years of experience. 
\item Perform rehearsals to gain experience, as experience has shown to be just as important as having established plans.
\begin{itemize}
\item Perform rehearsals both for large and small incidents. Remember that a small incident that is not sufficiently handled could escalate and lead to a more serious incident. Additionally, small incidents can also be valuable for learning.
\item Focus on challenging areas such as information dissemination, communication and division of responsibilities in rehearsals.
\item Perform rehearsals for regular employees, in addition to incident handlers, as they also have an information security responsibility. Typical topics in such a rehearsal would be related to information classification, incident detection (such as malicious e-mails) and reporting procedures. 
\end{itemize}
\item Develop clear and sound plans for communication.
\item Utilize employees as part of the sensor network. Make sure that developed reporting routines are actually \textit{established}.
\item Conduct awareness campaigns with a reasonable regularity, each being of a reasonable length. %(the latter also emphasized by local IT manager in case B, who said their last campaign had a very good length. They got very good feedback on this.)
\begin{itemize}
\item Send awareness campaigns by e-mail to make them easily accessible. A tip is to send them such that they are in the employees' inboxes when they arrive at work in the morning. We believe that people are extra susceptible to campaigns at that time, as other activities will not be disturbed.
\item Focus on improving employees' assessment of the value and sensitivity of information they process such that appropriate security measures can be implemented. 
\item Focus on making sure that employees are aware that information security \textit{does} concern them, such that they get familiar with their responsibilities.
\item Make employees aware of security limitations in the systems they use, such that sensitive information is not unnecessary exposed. Provide examples of how information can be lost or compromised.
\item Focus on making sure employees are attentive to malicious targeted e-mails as well as teaching them to recognize such e-mails.
\item Use incidents caused by employees or incidents that were/could have been detected by employees in awareness-raising activities. Incidents experienced by others, can also be used as examples. Further, we especially recommend using incidents discussed in the media as many employees will be familiar with these. 
\item If the organization does not have the resources to create awareness campaigns themselves, it is possible to buy these from external providers, as successfully done by Organization A and B.
\end{itemize}
\end{itemize}