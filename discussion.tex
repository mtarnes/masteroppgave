\chapter{Discussion}

Challenges (preliminary thoughts):
\begin{itemize}
\item Communication
\begin{itemize}
\item \textit{Who} should be contacted \textit{when}?
\item Who should be contacted when the actual contact person cannot be reached?
\end{itemize}
\item Situational Awareness
\begin{itemize}
\item Necessary to make fast and correct decisions in a complex and dynamic environment. It is challenging to know exactly where the incident ``hit" if one does not have situational awareness, thus knowing what to do and who's responsible. This was  seen as a challenge in case B as well as the main background for starting the major ``lessons learned" project in case C.
\end{itemize}
\item Information
\begin{itemize}
\item What information needs to be gathered?
\item What information should be \textit{communicated} to \textit{whom}?
\item Who should gather it?
\item How can it be gathered?
\end{itemize}
\item Responsibility
\begin{itemize}
\item Who is responsible for the incident?
\item Who is responsible for \textit{fixing} it?
\item Who is responsible for covering the costs?
\end{itemize}
\end{itemize}

These challenges (especially communication and responsibility) are extra evident in Case B, where there are several suppliers and many people that need to cooperate, but it is also seen in Case A, where there is only one organization handling its own IT operations involved.

Another finding is that even though procedures are important, experience may be just as important, as it is impossible to plan detailed for anything, and often one must make decisions on the fly.

It is important to keep the research questions in mind during the analysis:

\begin{itemize}
\item How do organizations perform information security incident management in practice?
\end{itemize}

And:

\begin{itemize}\itemsep-0.1cm
\item What plans and procedures for information security incident management are established in organizations?
\item To what extent are existing standards/guidelines adopted in plans for information security incident management?
\item To what extent have previous information security incidents been handled in accordance with predetermined plans? 
\end{itemize}

From Case A:

"There are many examples of issues that have been revealed through rehearsals. One such example is employees' understanding of risk. Even though various employees may have a certain risk awareness they may not agree on what the actual risk is."
As mentioned in threatlandscape, vulnerabilities exist mainly due to lack of understanding of risk (fra NSM, rapport 2012 \cite{NSMRapport2012}).

As found in case B, under lessons learned:
The interviewee from Supplier 2 mentioned that that handling and collecting information is a challenge. As an employee talks about information dissemination problems in \cite{ahmad2012incident}.
``it's the sharing, or rather finding of information. The information is there, each day I find a new resource that's got great information."

Under ``sikkerhet og sårbarhet"-konferansen nevnes det at man skal ta opp infosik so et tema under årlige medarbeidersamtaler, noe som blir gjort av virksomhet C og som nevnes som et effektivt tiltak hos dem. Står litt om det under ``awareness and training".
