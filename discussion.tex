\chapter{Discussion}
\label{chp:discussion}
In this chapter the findings from chapter \ref{chp:findings} are discussed and links between research questions and findings are established. The research questions are presented in section \ref{sec:objectives}. Further, underlying structures of experiences that emerged from the data are discussed. This is in accordance with the main purposes of an inductive research approach as described in \ref{sec:qualitativeAnalysis}  

\textit{What about management commitment? Is that present in the organizations? Very important according to ISO}

\section{Case A}
\textit{From Case A:}

\textit{"There are many examples of issues that have been revealed through rehearsals. One such example is employees' understanding of risk. Even though various employees may have a certain risk awareness they may not agree on what the actual risk is."
As mentioned in threatlandscape, vulnerabilities exist mainly due to lack of understanding of risk (fra NSM, rapport 2012 \cite{NSMRapport2012}).}

\textit{Organization A provides feedback to users that are reporting events. This is in accordance with ISO27002.}

\textit{RQ1:}\\
The organization has plans and procedures for incident handling, but they do not have one plan that encompasses all possible incidents. They also have contingency plans for all departments in addition to a central contingency plan. Additionally there exist specified workflows for some specific incident-types.

They have a policy for incident handling, which is compliant with ISO/IEC 27035. This document was not presented, so we did not get the chance to review the content.

Awareness and participation of all personnel is important (ISO/IEC 27035). \textit{To what extent is this present?} They have performed an awareness campaign. 

Additionally they perform rehearsals of contingency plans. Compliant with ISO/IEC 27035.

The existence of an information security policy is stated to be important (SANS). The organization has an information security policy. It may however not be well enough established throughout the organization. This is both consistent with what the IT security manager said and what wee discovered through the employee survey.

\textit{RQ2:}\\
The organization has not implemented one specific standard or guideline concerning incident management, but has based its approach on components from ISO/IEC 27001 and 27002 as well as the \ac{ITIL} framework. 

The organization has a team that works with incident handling. This is compliant with recommendations in all standards and guides presented in this report.

\textit{RQ3:}\\

\section{Case B}

\section{Case C}

\textit{Under ``sikkerhet og srbarhet"konferansen nevnes det at man skal ta opp infosik so et tema under rlige medarbeidersamtaler, noe som blir gjort av virksomhet C og som nevnes som et effektivt tiltak hos dem. Str litt om det under ``awareness and training".}

\section{Underlying structures of experiences}
\subsection{Communication}
\begin{itemize}
\item \textit{Communication}
\begin{itemize}
\item \textit{\textbf{Who} should be contacted \textbf{when}?}
\item \textit{Who should be contacted when the actual contact person cannot be reached?}
\end{itemize}
\end{itemize}

\subsection{Information}
\textit{Closely linked with communication}

\begin{itemize}
\item \textit{Information}
\begin{itemize}
\item \textit{What information needs to be collected?}
\item \textit{What information should be \textbf{communicated} to \textbf{whom}?}
\item \textit{Who should collect it?}
\item \textit{How can it be collected?}
\end{itemize}
\end{itemize}

\textit{As found in case B, under lessons learned:
The interviewee from Supplier 2 mentioned that that handling and collecting information is a challenge. As an employee talks about information dissemination problems in \cite{ahmad2012incident}.
``it's the sharing, or rather finding of information. The information is there, each day I find a new resource that's got great information."}

\subsection{Experience}

\begin{itemize}
\item \textit{Situational Awareness}
\begin{itemize}
\item\textit{ Necessary to make fast and correct decisions in a complex and dynamic environment. It is challenging to know exactly where the incident ``hit" if one does not have situational awareness, thus knowing what to do and who's responsible. This was  seen as a challenge in case B as well as the main background for starting the major ``lessons learned" project in case C.}
\end{itemize}
\end{itemize}

\textit{Another finding is that even though procedures are important, experience may be just as important, as it is impossible to plan detailed for anything, and often one must make decisions on the spot.}

\subsection{Responsibilities}
\textit{Also closely linked with communication. Better routines for communication may make responsibilities clearer?}

\begin{itemize}
\item \textit{Responsibilities}
\begin{itemize}
\item \textit{Who is responsible for the incident?}
\item \textit{Who is responsible for \textbf{fixing} it?}
\item \textit{Who is responsible for covering the costs?}
\end{itemize}
\end{itemize}

\textit{These challenges (especially communication and responsibility) are extra evident in Case B, where there are several suppliers and many people that need to cooperate, but it is also seen in Case A, where there is only one organization that handles its own IT operations involved.}

Another aspect is that users may not be aware that part of the responsibility lies with them. For example users that are owners of information may have a certain responsibility for protecting this information. An example is in case A, where users are responsible for classifying the information they process, but the interviewee thought that many are not aware of this (even though it is stated in the information security policy).