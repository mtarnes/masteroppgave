\chapter{Discussion}

Challenges (preliminary thoughts):
\begin{itemize}
\item Communication
\begin{itemize}
\item \textit{Who} should be contacted \textit{when}?
\item Who should be contacted when the actual contact person cannot be reached?
\end{itemize}
\item Situational Awareness
\begin{itemize}
\item Necessary to make fast and correct decisions in a complex and dynamic environment. It is challenging to know exactly where the incident ``hit" if one does not have situational awareness, thus knowing what to do and who's responsible. This was  seen as a challenge in case B as well as the main background for starting the major ``lessons learned" project in case C.
\end{itemize}
\item Information
\begin{itemize}
\item What information needs to be gathered?
\item What information should be \textit{communicated} to \textit{whom}?
\item Who should gather it?
\item How can it be gathered?
\end{itemize}
\item Responsibility
\begin{itemize}
\item Who is responsible for the incident?
\item Who is responsible for \textit{fixing} it?
\item Who is responsible for covering the costs?
\end{itemize}
\end{itemize}

These challenges (especially communication and responsibility) are extra evident in Case B, where there are several suppliers and many people that need to cooperate, but it is also seen in Case A, where there is only one organization handling its own IT operations involved.

Another finding is that even though procedures are important, experience may be just as important, as it is impossible to plan detailed for anything, and often one must make decisions on the spot.