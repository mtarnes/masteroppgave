\chapter{Conclusion and Future Work}
%Standarder er gjerne brukt som grunnlag, men kanskje like saa viktig er egne erfrainger fra ovinger og hendelser pluss at man ikke klarer aa ha detaljerte planer for alt.

In our thesis we have studied how organizations perform information security incident management in practice. We have examined what plans and procedures they have developed and how well these are established in the organizations. Additionally, we examined to what extent existing standards and guidelines were adopted in their plans and whether their practice complied with the standards and guidelines we studied.

An increasing use of digital solutions suggests that organizations today are more exposed to attacks then before. Despite implementing security policies and controls, incidents occur occasionally. This calls for an effective and efficient incident management. There exist literature addressing incident management, but not many studies have been performed of current practice. Through studying incident management in practise and providing recommendations for improvement, we hope that our research can contribute to an increased focus on incident management. Hopefully, organizations considering to implement or improve their incident management may benefit from our recommendations.

We found that the organizations had plans and procedures that are to some extent compliant with standards and guidelines. However, some of these procedures were not well established in the organizations. We highlight reporting procedures in particular, that were not sufficiently established throughout the organizations. In addition to finding answers to our research questions, other finding emerged as we analysed the data. One of the most prominent findings was that the organizations found an experienced incident handler just as important as having detailed plans. In addition, despite the organization in our study being large and experienced in incident handling, some challenges were prominent in all of the cases. These challenges were related to communication, information collection and dissemination, employee involvement and allocation of responsibilities. By observing these challenges we provided a set of recommendations to improve incident management.

Our recommendations address the challenges we found evident in the organizations studied. We recommend using standards and guidelines as a basis for incident management. Further, conducting regular rehearsals to gain experience is essential. Developing clear and sound plans for communication could also improve current practice. We saw that employees could be better utilized as part of the organization's sensor network and thus we emphasize the importance of making sure reporting procedures are well established. Additionally, conducting awareness campaigns have proved to be useful. These recommendations represent the essence of our findings. 

We hope that by conducting this research and providing these recommendations, organizations can become better prepared to respond to information security incidents in the future.


\section{Future work}
It would have been interesting to implement our recommendations in the studied organizations and perhaps also others, to see if they have a positive effect on their incident management. As we have only studied three organizations, our results are not generalizable and thus it would be interesting to conduct the same study with a larger number of organizations. Then, one could see if our findings also apply to organizations in general. Such a study could reveal more challenges and thus lead to more recommendations. Such a study could also be supplemented by a quantitative study, to see whether these challenges are evident in the majority of organizations. By including many organizations, one could also compare industries. This could lead to both general and industry specific recommendations.
