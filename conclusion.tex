\chapter{Conclusion and Future Work}
\label{chp:conclusion}
In our thesis we have studied how three large organizations perform information security incident management in practice. We have examined what plans and procedures they have developed and how well these are established. Additionally, we have examined to what extent existing standards and guidelines are adopted in the organizations' plans and whether their practices comply with the standards and guidelines we studied.

We found that the organizations have plans and procedures that are to some extent compliant with standards and guidelines. However, some of these procedures were not well established throughout the organizations. We highlight reporting procedures in particular, as procedures that were not sufficiently established. In addition to finding answers to our research questions, other findings emerged as we analysed the data. One observation was that the organizations found an experienced incident handler just as important for incident response as having detailed plans. Despite the organizations in our study being large and experienced in incident handling, some challenges were prominent in all of the cases. These challenges were related to communication, information collection and dissemination, employee involvement and allocation of responsibilities. 

By evaluating the challenges we developed a set of recommendations for improving incident management practices. We recommend using standards and guidelines as a basis for incident management. Further, conducting regular rehearsals to gain experience is essential. The development of clear and sound plans for communication could also improve current practice. We saw that employees could be better utilized as part of organizations' sensor networks and thus we emphasize the importance of making sure reporting procedures are well established. Additionally, conducting awareness campaigns has proven to be useful. 

We hope that by conducting this research and providing these recommendations, we can contribute to organizations becoming better prepared to respond to information security incidents in the future. 

We believe it is valuable to continue the research on incident management as recent reports and surveys have indicated that threats are changing and increasing. It would be interesting to implement our recommendations in the studied organizations and perhaps other organizations as well, to see whether they can have a positive effect on incident management. As we have only studied a limited number of organizations, our results are not generalizable and thus it would be interesting to conduct the same study with a larger number of organizations. This can verify whether our findings apply to organizations in general. Such a study can reveal more challenges and thus lead to more recommendations. It can further be supplemented by a quantitative study, to see whether these challenges are evident in the majority of organizations. By including a large number of organizations, one can compare industries as well. This can lead to both general and industry specific recommendations.
